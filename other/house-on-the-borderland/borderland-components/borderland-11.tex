% File borderland-11.txt
% Version 2017/09/18
% In the original, this was Chapter XVII.

\clearpage
\label{ch:11}

\begin{ChapterStart}
\null\null
\ChapterTitle{11. The Slowing Rotation}
\end{ChapterStart}

It might have been a million years later, that I perceived, beyond possibility of doubt, that the fiery sheet that lit the world, was indeed darkening.

Another vast space went by, and the whole enormous flame had sunk to a deep, copper color. Gradually, it darkened, from copper to copper-red, and from this, at times, to a deep, heavy, purplish tint, with, in it, a strange loom of blood.

Although the light was decreasing, I could perceive no diminishment in the apparent speed of the sun. It still spread itself in that dazzling veil of speed.

The world, so much of it as I could see, had assumed a dreadful shade of gloom, as though, in very deed, the last day of the worlds approached.

The sun was dying; of that there could be little doubt; and still the earth whirled onward, through space and all the aeons. At this time, I remember, an extraordinary sense of bewilderment took me. I found myself, later, wandering, mentally, amid an odd chaos of fragmentary modern theories and the old Biblical story of the world’s ending.

Then, for the first time, there flashed across me, the memory that the sun, with its system of planets, was, and had been, traveling through space at an incredible speed. Abruptly, the question rose---\textit{Where?} For a very great time, I pondered this matter; but, finally, with a certain sense of the futility of my puzzlings, I let my thoughts wander to other things. I grew to wondering, how much longer the house would stand. Also, I queried, to myself, whether I should be doomed to stay, bodiless, upon the earth, through the dark-time that I knew was coming. From these thoughts, I fell again to speculations upon the possible direction of the sun’s journey through space.... And so another great while passed.

Gradually, as time fled, I began to feel the chill of a great winter. Then, I remembered that, with the sun dying, the cold must be, necessarily, extraordinarily intense. Slowly, slowly, as the aeons slipped into eternity, the earth sank into a heavier and redder gloom. The dull flame in the firmament took on a deeper tint, very somber and turbid.

Then, at last, it was borne upon me that there was a change. The fiery, gloomy curtain of flame that hung quaking overhead, and down away into the Southern sky, began to thin and contract; and, in it, as one sees the fast vibrations of a jarred harp-string, I saw once more the sun-stream quivering, giddily, North and South.

Slowly, the likeness to a sheet of fire, disappeared, and I saw, plainly, the slowing beat of the sun-stream. Yet, even then, the speed of its swing was inconceivably swift. And all the time, the brightness of the fiery arc grew ever duller. Underneath, the world loomed dimly---an indistinct, ghostly region.

Overhead, the river of flame swayed slower, and even slower; until, at last, it swung to the North and South in great, ponderous beats, that lasted through seconds. A long space went by, and now each sway of the great belt lasted nigh a minute; so that, after a great while, I ceased to distinguish it as a visible movement; and the streaming fire ran in a steady river of dull flame, across the deadly-looking sky.

An indefinite period passed, and it seemed that the arc of fire became less sharply defined. It appeared to me to grow more attenuated, and I thought blackish streaks showed, occasionally. Presently, as I watched, the smooth onward-flow ceased; and I was able to perceive that there came a momentary, but regular, darkening of the world. This grew until, once more, night descended, in short, but periodic, intervals upon the wearying earth.

Longer and longer became the nights, and the days equaled them; so that, at last, the day and the night grew to the duration of seconds in length, and the sun showed, once more, like an almost invisible, coppery-red colored ball, within the glowing mistiness of its flight. Corresponding to the dark lines, showing at times in its trail, there were now distinctly to be seen on the half-visible sun itself, great, dark belts.

Year after year flashed into the past, and the days and nights spread into minutes. The sun had ceased to have the appearance of a tail; and now rose and set---a tremendous globe of a glowing copper-bronze hue; in parts ringed with blood-red bands; in others, with the dusky ones, that I have already mentioned. These circles---both red and black---were of varying thicknesses. For a time, I was at a loss to account for their presence. Then it occurred to me, that it was scarcely likely that the sun would cool evenly all over; and that these markings were due, probably, to differences in temperature of the various areas; the red representing those parts where the heat was still fervent, and the black those portions which were already comparatively cool.

It struck me, as a peculiar thing, that the sun should cool in evenly defined rings; until I remembered that, possibly, they were but isolated patches, to which the enormous rotatory speed of the sun had imparted a belt-like appearance. The sun, itself, was very much greater than the sun I had known in the old-world days; and, from this, I argued that it was considerably nearer.

At nights, the moon\footnote{No further mention is made of the moon. From what is said here, it is evident that our satellite had greatly increased its distance from the earth. Possibly, at a later age it may even have broken loose from our attraction. I cannot but regret that no light is shed on this point.---\allsmcp{WHH}} still showed; but small and remote; and the light she reflected was so dull and weak that she seemed little more than the small, dim ghost of the olden moon, that I had known.

The days and nights lengthened out, until they equaled a space somewhat less than one of the old-earth hours; the sun rising and setting like a great, ruddy bronze disk, crossed with ink-black bars. About this time, I found myself, able once more, to see the gardens, with clearness. For the world had now grown very still, and changeless. Yet, I am not correct in saying, ‘gardens’; for there were no gardens---nothing that I knew or recognized. In place thereof, I looked out upon a vast plain, stretching away into distance. A little to my left, there was a low range of hills. Everywhere, there was a uniform, white covering of snow, in places rising into hummocks and ridges.

It was only now, that I recognized how really great had been the snowfall. In places it was vastly deep, as was witnessed by a great, upleaping, wave-shaped hill, away to my right; though it is not impossible, that this was due, in part, to some rise in the surface of the ground. Strangely enough, the range of low hills to my left---already mentioned---was not entirely covered with the universal snow; instead, I could see their bare, dark sides showing in several places. And everywhere and always there reigned an incredible death-silence and desolation. The immutable, awful quiet of a dying world.

All this time, the days and nights were lengthening, perceptibly. Already, each day occupied, maybe, some two hours from dawn to dusk. At night, I had been surprised to find that there were very few stars overhead, and these small, though of an extraordinary brightness; which I attributed to the peculiar, but clear, blackness of the nighttime.

Away to the North, I could discern a nebulous sort of mistiness; not unlike, in appearance, a small portion of the Milky Way. It might have been an extremely remote star-cluster; or---the thought came to me suddenly---perhaps it was the sidereal universe that I had known, and now left far behind, forever---a small, dimly glowing mist of stars, far in the depths of space.

Still, the days and nights lengthened, slowly. Each time, the sun rose duller than it had set. And the dark belts increased in breadth.

About this time, there happened a fresh thing. The sun, earth, and sky were suddenly darkened, and, apparently, blotted out for a brief space. I had a sense, a certain awareness (I could learn little by sight), that the earth was enduring a very great fall of snow. Then, in an instant, the veil that had obscured everything, vanished, and I looked out, once more. A marvelous sight met my gaze. The hollow in which this house, with its gardens, stands, was brimmed with snow.\footnote{Conceivably, frozen air.---\allsmcp{WHH}} It lipped over the sill of my window. Everywhere, it lay, a great level stretch of white, which caught and reflected, gloomily, the somber coppery glows of the dying sun. The world had become a shadowless plain, from horizon to horizon.

I glanced up at the sun. It shone with an extraordinary, dull clearness. I saw it, now, as one who, until then, had seen it, only through a partially obscuring medium. All about it, the sky had become black, with a clear, deep blackness, frightful in its nearness, and its unmeasured deep, and its utter unfriendliness. For a great time, I looked into it, newly, and shaken and fearful. It was so near. Had I been a child, I might have expressed some of my sensation and distress, by saying that the sky had lost its roof.

Later, I turned, and peered about me, into the room. Everywhere, it was covered with a thin shroud of the all-pervading white. I could see it but dimly, by reason of the somber light that now lit the world. It appeared to cling to the ruined walls; and the thick, soft dust of the years, that covered the floor knee-deep, was nowhere visible. The snow must have blown in through the open framework of the windows. Yet, in no place had it drifted; but lay everywhere about the great, old room, smooth and level. Moreover, there had been no wind these many thousand years. But there was the snow.\footnote{Previous footnote would explain the snow (?) within the room.---\allsmcp{WHH}}

And all the earth was silent. And there was a cold, such as no living man can ever have known.

The earth was now illuminated, by day, with a most doleful light, beyond my power to describe. It seemed as though I looked at the great plain, through the medium of a bronze-tinted sea.

It was evident that the earth’s rotatory movement was departing, steadily.

The end came, all at once. The night had been the longest yet; and when the dying sun showed, at last, above the world’s edge, I had grown so wearied of the dark, that I greeted it as a friend. It rose steadily, until about twenty degrees above the horizon. Then, it stopped suddenly, and, after a strange retrograde movement, hung motionless---a great shield in the sky.\footnote{I am confounded that neither here, nor later on, does the Recluse make any further mention of the continued north and south movement (apparent, of course,) of the sun from solstice to solstice.---\allsmcp{WHH}} Only the circular rim of the sun showed bright---only this, and one thin streak of light near the equator.

Gradually, even this thread of light died out; and now, all that was left of our great and glorious sun, was a vast dead disk, rimmed with a thin circle of bronze-red light.

\clearpage
