% File borderland-07.txt
% Version 2017/09/18
% In the original, this was Chapter XIII.

\clearpage
\label{ch:07}

\begin{ChapterStart}
\null\null
\ChapterTitle{7. The Trap in the Great Cellar}
\end{ChapterStart}

It was not until a couple of days later, that I managed to get across to the ravine. There, I found that, in my few weeks’ absence, there had been wrought a wondrous change. Instead of the three-parts filled ravine, I looked out upon a great lake, whose placid surface, reflected the light, coldly. The water had risen to within half a dozen feet of the ravine edge. Only in one part was the lake disturbed, and that was above the place where, far down under the silent waters, yawned the entrance to the vast, underground Pit. Here, there was a continuous bubbling; and, occasionally, a curious sort of sobbing gurgle would find its way up from the depth. Beyond these, there was nothing to tell of the things that were hidden beneath. As I stood there, it came to me how wonderfully things had worked out. It was completely shut off and concealed from human curiosity forever.

Is it not possible that it has, all along, held a deeper significance, a hint---could one but have guessed---of the Pit that lies far down in the earth, beneath this old house? Under this house! Even now, the idea is strange and terrible to me. For I have proved, beyond doubt, that the Pit yawns right below the house, which is evidently supported, somewhere above the center of it, upon a tremendous, arched roof, of solid rock.

It happened in this wise, that, having occasion to go down to the cellars, the thought occurred to me to pay a visit to the great vault, where the trap is situated; and see whether everything was as I had left it.

Reaching the place, I walked slowly up the center, until I came to the trap. There it was, with the stones piled upon it, just as I had seen it last. I had a lantern with me, and the idea came to me, that now would be a good time to investigate whatever lay under the great, oak slab. Placing the lantern on the floor, I tumbled the stones off the trap, and, grasping the ring, pulled the door open. As I did so, the cellar became filled with the sound of a murmurous thunder, that rose from far below. At the same time, a damp wind blew up into my face, bringing with it a load of fine spray. Therewith, I dropped the trap, hurriedly, with a half frightened feeling of wonder.

For a moment, I stood puzzled. Then, a sudden thought possessed me, and I raised the ponderous door, with a feeling of excitement. Leaving it standing upon its end, I seized the lantern, and, kneeling down, thrust it into the opening. As I did so, the moist wind and spray drove in my eyes, making me unable to see, for a few moments. Even when my eyes were clear, I could distinguish nothing below me, save darkness, and whirling spray.

Seeing that it was useless to expect to make out anything, with the light so high, I felt in my pockets for a piece of twine, with which to lower it further into the opening. Even as I fumbled, the lantern slipped from my fingers, and hurtled down into the darkness. For a brief instant, I watched its fall, and saw the light shine on a tumult of white foam, some eighty or a hundred feet below me. Then it was gone. My sudden surmise was correct, and now, I knew the cause of the wet and noise. The great cellar was connected with the Pit, by means of the trap, which opened right above it; and the moisture, was the spray, rising from the water, falling into the depths.

In an instant, I had an explanation of certain things, that had hitherto puzzled me. These thoughts flashed through my brain, as I stood in the dark, searching my pockets for matches. I had the box in my hand now, and, striking a light, I stepped to the trap door, and closed it. Then, I piled the stones back upon it; after which, I made my way out from the cellars.

And so, I suppose the water goes on, thundering down into that bottomless hell-pit. Sometimes, I have an inexplicable desire to go down to the great cellar, open the trap, and gaze into the impenetrable, spray-damp darkness. At times, the desire becomes almost overpowering, in its intensity. It is not mere curiosity, that prompts me; but more as though some unexplained influence were at work. Still, I never go; and intend to fight down the strange longing, and crush it; even as I would the unholy thought of self-destruction.

This idea of some intangible force being exerted, may seem reasonless. Yet, my instinct warns me, that it is not so. In these things, reason seems to me less to be trusted than instinct.

One thought there is, in closing, that impresses itself upon me, with ever growing insistence. It is, that I live in a very strange house; a very awful house. And I have begun to wonder whether I am doing wisely in staying here. Yet, if I left, where could I go, and still obtain the solitude, and the sense of her presence, that alone make my old life bearable?\footnote{An apparently unmeaning interpolation. I can find no previous reference in the MS. to this matter. It becomes clearer, however, in the light of succeeding incidents.---\allsmcp{WHH}}

\clearpage
