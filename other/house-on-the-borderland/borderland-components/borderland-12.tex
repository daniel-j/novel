% File borderland-12.txt
% Version 2017/09/18
% In the original, this was Chapter XVIII.

\clearpage
\label{ch:12}

\begin{ChapterStart}
\null\null
\ChapterTitle{12. The Green Star}
\end{ChapterStart}

The world was held in a savage gloom---cold and intolerable. Outside, all was quiet---quiet! From the dark room behind me, came the occasional, soft thud of falling matter---fragments of rotting stone.\footnote{At this time the sound-carrying atmosphere must have been either incredibly attenuated, or---more probably---nonexistent. In the light of this, it cannot be supposed that these, or any other, noises would have been apparent to living ears---to hearing, as we, in the material body, understand that sense.---\allsmcp{WHH}} So time passed, and night grasped the world, wrapping it in wrappings of impenetrable blackness.

There was no night-sky, as we know it. Even the few straggling stars had vanished, conclusively. I might have been in a shuttered room, without a light; for all that I could see. Only, in the impalpableness of gloom, opposite, burnt that vast, encircling hair of dull fire. Beyond this, there was no ray in all the vastitude of night that surrounded me; save that, far in the North, that soft, mistlike glow still shone.

Silently, years moved on. What period of time passed, I shall never know. It seemed to me, waiting there, that eternities came and went, stealthily; and still I watched. I could see only the glow of the sun’s edge, at times; for now, it had commenced to come and go---lighting up a while, and again becoming extinguished.

All at once, during one of these periods of life, a sudden flame cut across the night---a quick glare that lit up the dead earth, shortly; giving me a glimpse of its flat lonesomeness. The light appeared to come from the sun---shooting out from somewhere near its center, diagonally. A moment, I gazed, startled. Then the leaping flame sank, and the gloom fell again. But now it was not so dark; and the sun was belted by a thin line of vivid, white light. I stared, intently. Had a volcano broken out on the sun? Yet, I negatived the thought, as soon as formed. I felt that the light had been far too intensely white, and large, for such a cause.

Another idea there was, that suggested itself to me. It was, that one of the inner planets had fallen into the sun---becoming incandescent, under that impact. This theory appealed to me, as being more plausible, and accounting more satisfactorily for the extraordinary size and brilliance of the blaze, that had lit up the dead world, so unexpectedly.

Full of interest and emotion, I stared, across the darkness, at that line of white fire, cutting the night. One thing it told to me, unmistakably: the sun was yet rotating at an enormous speed.\footnote{I can only suppose that the time of the earth’s yearly journey had ceased to bear its present \textit{relative} proportion to the period of the sun’s rotation.---WHH} Thus, I knew that the years were still fleeting at an incalculable rate; though so far as the earth was concerned, life, and light, and time, were things belonging to a period lost in the long gone ages.

After that one burst of flame, the light had shown, only as an encircling band of bright fire. Now, however, as I watched, it began to sink into a ruddy tint, and, later, to a dark, copper-red color; much as the sun had done. Presently, it sank to a deeper hue; and, in a still further space of time, it began to fluctuate; having periods of glowing, and anon, dying. Thus, after a great while, it disappeared.

Long before this, the smoldering edge of the sun had deadened into blackness. And so, in that supremely future time, the world, dark and intensely silent, rode on its gloomy orbit around the ponderous mass of the dead sun.

My thoughts, at this period, can be scarcely described. At first, they were chaotic and wanting in coherence. But, later, as the ages came and went, my soul seemed to imbibe the very essence of the oppressive solitude and dreariness, that held the earth.

With this feeling, there came a wonderful clearness of thought, and I realized, despairingly, that the world might wander for ever, through that enormous night. For a while, the unwholesome idea filled me, with a sensation of overbearing desolation; so that I could have cried like a child. In time, however, this feeling grew, almost insensibly, less, and an unreasoning hope possessed me. Patiently, I waited.

From time to time, the noise of dropping particles, behind in the room, came dully to my ears. Once, I heard a loud crash, and turned, instinctively, to look; forgetting, for the moment, the impenetrable night in which every detail was submerged. In a while, my gaze sought the heavens; turning, unconsciously, toward the North. Yes, the nebulous glow still showed. Indeed, I could have almost imagined that it looked somewhat plainer. For a long time, I kept my gaze fixed upon it; feeling, in my lonely soul, that its soft haze was, in some way, a tie with the past. Strange, the trifles from which one can suck comfort! And yet, had I but known---But I shall come to that in its proper time.

For a very long space, I watched, without experiencing any of the desire for sleep, that would so soon have visited me in the old-earth days. How I should have welcomed it; if only to have passed the time, away from my perplexities and thoughts.

Several times, the comfortless sound of some great piece of masonry falling, disturbed my meditations; and, once, it seemed I could hear whispering in the room, behind me. Yet it was utterly useless to try to see anything. Such blackness, as existed, scarcely can be conceived. It was palpable, and hideously brutal to the sense; as though something dead, pressed up against me---something soft, and icily cold.

Under all this, there grew up within my mind, a great and overwhelming distress of uneasiness, that left me, but to drop me into an uncomfortable brooding. I felt that I must fight against it; and, presently, hoping to distract my thoughts, I turned to the window, and looked up toward the North, in search of the nebulous whiteness, which, still, I believed to be the far and misty glowing of the universe we had left. Even as I raised my eyes, I was thrilled with a feeling of wonder; for, now, the hazy light had resolved into a single, great star, of vivid green.

As I stared, astonished, the thought flashed into my mind; that the earth must be traveling toward the star; not away, as I had imagined. Next, that it could not be the universe the earth had left; but, possibly, an outlying star, belonging to some vast star-cluster, hidden in the enormous depths of space. With a sense of commingled awe and curiosity, I watched it, wondering what new thing was to be revealed to me.

For a while, vague thoughts and speculations occupied me, during which my gaze dwelt insatiably upon that one spot of light, in the otherwise pitlike darkness. Hope grew up within me, banishing the oppression of despair, that had seemed to stifle me. Wherever the earth was traveling, it was, at least, going once more toward the realms of light. Light! One must spend an eternity wrapped in soundless night, to understand the full horror of being without it.

Slowly, but surely, the star grew upon my vision, until, in time, it shone as brightly as had the planet Jupiter, in the old-earth days. With increased size, its color became more impressive; reminding me of a huge emerald, scintillating rays of fire across the world.

Years fled away in silence, and the green star grew into a great splash of flame in the sky. A little later, I saw a thing that filled me with amazement. It was the ghostly outline of a vast crescent, in the night; a gigantic new moon, seeming to be growing out of the surrounding gloom. Utterly bemused, I stared at it. It appeared to be quite close---comparatively; and I puzzled to understand how the earth had come so near to it, without my having seen it before.

The light, thrown by the star, grew stronger; and, presently, I was aware that it was possible to see the earthscape again; though indistinctly. Awhile, I stared, trying to make out whether I could distinguish any detail of the world’s surface, but I found the light insufficient. In a little, I gave up the attempt, and glanced once more toward the star. Even in the short space, that my attention had been diverted, it had increased considerably, and seemed now,\linebreak to my bewildered sight, about a quarter of the size of the full moon. The light it threw, was extraordinarily powerful; yet its color was so abominably unfamiliar, that such of the world as I could see, showed unreal; more as though I looked out upon a landscape of shadow, than aught else.

All this time, the great crescent was increasing in brightness, and began, now, to shine with a perceptible shade of green. Steadily, the star increased in size and brilliancy, until it showed, fully as large as half a full moon; and, as it grew greater and brighter, so did the vast crescent throw out more and more light, though of an ever deepening hue of green. Under the combined blaze of their radiances, the wilderness that stretched before me, became steadily more visible. Soon, I seemed able to stare across the whole world, which now appeared, beneath the strange light, terrible in its cold and awful, flat dreariness.

It was a little later, that my attention was drawn to the fact, that the great star of green flame, was slowly sinking out of the North, toward the East. At first, I could scarcely believe that I saw aright; but soon there could be no doubt that it was so. Gradually, it sank, and, as it fell, the vast crescent of glowing green, began to dwindle and dwindle, until it became a mere arc of light, against the livid colored sky. Later it vanished, disappearing in the self-same spot from which I had seen it slowly emerge.

By this time, the star had come to within some thirty degrees of the hidden horizon. In size it could now have rivaled the moon at its full; though, even yet, I could not distinguish its disk. This fact led me to conceive that it was, still, an extraordinary distance away; and, this being so, I knew that its size must be huge, beyond the conception of man to understand or imagine.

Suddenly, as I watched, the lower edge of the star vanished---cut by a straight, dark line. A minute---or a century---passed, and it dipped lower, until the half of it had disappeared from sight. Far away out on the great plain, I saw a monstrous shadow blotting it out, and advancing swiftly. Only a third of the star was visible now. Then, like a flash, the solution of this extraordinary phenomenon revealed itself to me. The star was sinking behind the enormous mass of the dead sun. Or rather, the sun---obedient to its attraction---was rising toward it, with the earth following in its trail.\footnote{A careful reading of the MS. suggests that, either the sun is traveling on an orbit of great eccentricity, or else that it was approaching the green star on a lessening orbit. And at this moment, I conceive it to be finally torn directly from its oblique course, by the gravitational pull of the immense star.---WHH} As these thoughts expanded in my mind, the star vanished; being completely hidden by the tremendous bulk of the sun. Over the earth there fell, once more, the brooding night.

With the darkness, came an intolerable feeling of loneliness and dread. For the first time, I thought of the Pit, and its inmates. After that, there rose in my memory the still more terrible Thing, that had haunted the shores of the Sea of Sleep, and lurked in the shadows of this old building. Where were they? I wondered---and shivered with miserable thoughts. For a time, fear held me, and I prayed, wildly and incoherently, for some ray of light with which to dispel the cold blackness that enveloped the world.

How long I waited, it is impossible to say---certainly for a very great period. Then, all at once, I saw a loom of light shine out ahead. Gradually, it became more distinct. Suddenly, a ray of vivid green, flashed across the darkness. At the same moment, I saw a thin line of livid flame, far in the night. An instant, it seemed, and it had grown into a great clot of fire; beneath which, the world lay bathed in a blaze of emerald green light. Steadily it grew, until, presently, the whole of the green star had come into sight again. But now, it could be scarcely called a star; for it had increased to vast proportions, being incomparably greater than the sun had been in the olden time.

Then, as I stared, I could see the edge of the lifeless sun, glowing like a great crescent-moon. Slowly, its lighted surface, broadened out to me, until half of its diameter was visible; and the star began to drop away on my right. Time passed, and the earth moved on, slowly traversing the tremendous face of the dead sun.\footnote{It will be noticed here that the earth was “\textit{slowly} traversing the tremendous face of the dead sun.” No explanation is given of this, and we must conclude, either that the speed of time had slowed, or else that the earth was actually progressing on its orbit at a rate, slow, when measured by existing standards. A careful study of the MS. however, leads me to conclude that the speed of time had been steadily decreasing for a very considerable period.---WHH}

Gradually, as the earth traveled forward, the star fell still more to the right; until, at last, it shone on the back of the house, sending a flood of broken rays, in through the skeleton-like walls. Glancing upward, I saw that much of the ceiling had vanished, enabling me to see that the upper storeys were even more decayed. The roof had, evidently, gone entirely; and I could see the green effulgence of the Starlight shining in, slantingly.

\clearpage
