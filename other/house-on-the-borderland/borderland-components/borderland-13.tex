% File borderland-13.txt
% Version 2017/09/18
% In the original, this was Chapter XIX.

\clearpage
\label{ch:13}

\begin{ChapterStart}
\null\null
\ChapterTitle{13. The End of the Solar System}
\end{ChapterStart}

From the abutment, where once had been the windows, through which I had watched that first, fatal dawn, I could see that the sun was hugely greater, than it had been, when first the Star lit the world. So great was it, that its lower edge seemed almost to touch the far horizon. Even as I watched, I imagined that it drew closer. The radiance of green that lit the frozen earth, grew steadily brighter.

Thus, for a long space, things were. Then, on a sudden, I saw that the sun was changing shape, and growing smaller, just as the moon would have done in past time. In a while, only a third of the illuminated part was turned toward the earth. The Star bore away on the left.

Gradually, as the world moved on, the Star shone upon the front of the house, once more; while the sun showed, only as a great bow of green fire. An instant, it seemed, and the sun had vanished. The Star was still fully visible. Then the earth moved into the black shadow of the sun, and all was night---Night, black, starless, and intolerable.

Filled with tumultuous thoughts, I watched across the night---waiting. Years, it may have been, and then, in the dark house behind me, the clotted stillness of the world was broken. I leapt from the window, out on to the frozen world. I have a confused notion of having run awhile; and, after that, I just waited---waited. Time moved onward. I was conscious of little, save a sensation of cold and hopelessness and fear.

An age, it seemed, and there came a glow, that told of the coming light. It grew, tardily. Then---with a loom of unearthly glory---the first ray from the Green Star, struck over the edge of the dark sun, and lit the world. It fell upon a great, ruined structure, some two hundred yards away. It was the house.

The world moved out into the light of the Star, and I saw that, now, it seemed to stretch across a quarter of the heavens. The glory of its livid light was so tremendous, that it appeared to fill the sky with quivering flames. Then, I saw the sun. It was so close that half of its diameter lay below the horizon; and, as the world circled across its face, it seemed to tower right up into the sky, a stupendous dome of emerald colored fire.

Years appeared to pass, slowly. The earth had almost reached the center of the sun’s disk. The light from the Green \textit{Sun}---as now it must be called---shone through the interstices, that gapped the mouldered walls of the old house, giving them the appearance of being wrapped in green flames.

Suddenly, up from the center of the roofless house, shot a vast column of blood-red flame. I saw the little, twisted towers and turrets flash into fire; yet still preserving their twisted crookedness. The beams of the Green Sun, beat upon the house, and intermingled with its glows; so that it appeared a furnace of red and green fire.

Fascinated, I watched, until an overwhelming sense of coming danger, drew my attention. I glanced up, and, at once, it was borne upon me, that the sun was closer; so close, in fact, that it seemed to overhang the world. Then---I know not how---I was caught up into strange heights---floating like a bubble in the awful effulgence.

Far below me, I saw the earth, with the burning house leaping into an ever growing mountain of flame, ’round about it, the ground appeared to be glowing; and, in places, heavy wreaths of yellow smoke ascended from the earth. It seemed as though the world were becoming ignited from that one plague-spot of fire. Then the ground seemed to cave in, suddenly, and the house, disappeared into the depths of the earth, sending a strange, blood colored cloud into the heights. I remembered the hell Pit under the house.

In a while, I looked ’round. The huge bulk of the sun, rose high above me. The distance between it and the earth, grew rapidly less. Suddenly, the earth appeared to shoot forward. In a moment, it had traversed the space between it and the sun. I heard no sound; but, out from the sun’s face, gushed an ever-growing tongue of dazzling flame. It seemed to leap, almost to the distant Green Sun---shearing through the emerald light, a very cataract of blinding fire. It reached its limit, and sank; and, on the sun, glowed a vast splash of burning white---the grave of the earth.

The sun was very close to me, now. Presently, I found that I was rising higher; until, at last, I rode above it, in the emptiness. The Green Sun was now so huge that its breadth seemed to fill up all the sky, ahead. I looked down, and noted that the sun was passing directly beneath me.

A year may have gone by---or a century---and I was left, suspended, alone. The sun showed far in front---a black, circular mass, against the molten splendor of the great, Green Orb. Near one edge, I observed that a lurid glow had appeared, marking the place where the earth had fallen. By this, I knew that the long-dead sun was still revolving, though with great slowness.

Afar to my right, I seemed to catch, at times, a faint glow of whitish light. For a great time, I was uncertain whether to put this down to fancy or not. Thus, for a while, I stared, with fresh wonderings; until, at last, I knew that it was no imaginary thing; but a reality. It grew brighter; and, presently, there slid out of the green, a pale globe of softest white. It came nearer, and I saw that it was apparently surrounded by a robe of gently glowing clouds. Time passed....

I glanced toward the diminishing sun. It showed, only as a dark blot on the face of the Green Sun. As I watched, I saw it grow smaller, steadily, as though rushing toward the superior orb, at an immense speed. Intently, I stared. What would happen? I was conscious of extraordinary emotions, as I realized that it would strike the Green Sun. It grew no bigger than a pea, and I looked, with my whole soul, to witness the final end of our System---that system which had borne the world through so many aeons, with its multitudinous sorrows and joys; and now---

Suddenly, something crossed my vision, cutting from sight all vestige of the spectacle I watched with such soul-interest. What happened to the dead sun, I did not see; but I have no reason---in the light of that which I saw afterward---to disbelieve that it fell into the strange fire of the Green Sun, and so perished.

And then, suddenly, an extraordinary question rose in my mind, whether this stupendous globe of green fire might not be the vast Central Sun---the great sun, ’round which our universe and countless others revolve. I felt confused. I thought of the probable end of the dead sun, and another suggestion came, dumbly---Do the dead stars make the Green Sun their grave? The idea appealed to me with no sense of grotesqueness; but rather as something both possible and probable.

\clearpage
