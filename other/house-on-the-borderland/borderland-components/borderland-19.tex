% File borderland-19.txt
% Version 2017/09/18
% In the original, this was Chapter XXV.

\clearpage
\label{ch:19}

\begin{ChapterStart}
\null\null
\ChapterTitle{19. The Thing from the Arena}
\end{ChapterStart}

This morning, early, I went through the gardens; but found everything as usual. Near the door, I examined the path, for footprints; yet, here again, there was nothing to tell me whether, or not, I dreamed last night.

It was only when I came to speak to the dog, that I discovered tangible proof, that something did happen. When I went to his kennel, he kept inside, crouching up in one corner, and I had to coax him, to get him out. When, finally, he consented to come, it was in a cowed and subdued manner. As I patted him, my attention was attracted to a greenish patch, on his left flank. On examining it, I found, that the fur and skin had been apparently, burnt off; for the flesh showed, raw and scorched. The shape of the mark was curious, reminding me of the imprint of a large talon or hand.

I stood up, thoughtful. My gaze wandered toward the study window. The rays of the rising sun, shimmered on the smoky patch in the lower corner, causing it to fluctuate from green to red, oddly. Ah! that was undoubtedly another proof; and, suddenly, the horrible Thing I saw last night, rose in my mind. I looked at the dog, again. I knew the cause, now, of that hateful looking wound on his side---I knew, also, that, what I had seen last night, had been a real happening. And a great discomfort filled me. Pepper! And now this poor animal ...! I glanced at the dog again, and noticed that he was licking at his wound.

‘Poor brute!’ I muttered, and bent to pat his head. At that, he got upon his feet, nosing and licking my hand, wistfully.

Presently, I left him, having other matters to which to attend.

After dinner, I went to see him, again. He seemed quiet, and disinclined to leave his kennel. He has refused all food today.

The day has passed, uneventfully enough. After tea, I went, again, to have a look at the dog. He seemed moody, and somewhat restless; yet persisted in remaining in his kennel. Before locking up, for the night, I moved his kennel out, away from the wall, so that I shall be able to watch it from the small window, tonight. The thought came to me, to bring him into the house for the night; but consideration has decided me, to let him remain out. I cannot say that the house is, in any degree, less to be feared than the gardens. Pepper was in the house, and yet....

It is now two o’clock. Since eight, I have watched the kennel, from the small, side window in my study. Yet, nothing has occurred, and I am too tired to watch longer. I will go to bed....

During the night, I was restless. This is unusual for me; but, toward morning, I obtained a few hours’ sleep.

I rose early, and, after breakfast, visited the dog. He was quiet; but morose, and refused to leave his kennel. I wish there was some horse doctor near here; I would have the poor brute looked to. All day, he has taken no food; but has shown an evident desire for water---lapping it up, greedily. I was relieved to observe this.

The evening has come, and I am in my study. I intend to follow my plan of last night, and watch the kennel. The door, leading into the garden, is bolted, securely. I am consciously glad there are bars to the windows....

Night:---Midnight has gone. The dog has been silent, up to the present. Through the side window, on my left, I can make out, dimly, the outlines of the kennel. For the first time, the dog moves, and I hear the rattle of his chain. I look out, quickly. As I stare, the dog moves again, restlessly, and I see a small patch of luminous light, shine from the interior of the kennel. It vanishes; then the dog stirs again, and, once more, the gleam comes. I am puzzled. The dog is quiet, and I can see the luminous thing, plainly. It shows distinctly. There is something familiar about the shape of it. For a moment,\linebreak I wonder; then it comes to me, that it is not unlike the four fingers and thumb of a hand. Like a hand! And I remember the contour of that fearsome wound on the dog’s side. It must be the wound I see. It is luminous at night---Why? The minutes pass. My mind is filled with this fresh thing....

Suddenly, I hear a sound, out in the gardens. How it thrills through me. It is approaching. Pad, pad, pad. A prickly sensation traverses my spine, and seems to creep across my scalp. The dog moves in his kennel, and whimpers, frightenedly. He must have turned ’round; for, now, I can no longer see the outline of his shining wound.

Outside, the gardens are silent, once more, and I listen, fearfully. A minute passes, and another; then I hear the padding sound, again. It is quite close, and appears to be coming down the graveled path. The noise is curiously measured and deliberate. It ceases outside the door; and I rise to my feet, and stand motionless. From the door, comes a slight sound---the latch is being slowly raised. A singing noise is in my ears, and I have a sense of pressure about the head---

The latch drops, with a sharp click, into the catch. The noise startles me afresh; jarring, horribly, on my tense nerves. After that, I stand, for a long while, amid an ever-growing quietness. All at once, my knees begin to tremble, and I have to sit, quickly.

An uncertain period of time passes, and, gradually, I begin to shake off the feeling of terror, that has possessed me. Yet, still I sit. I seem to have lost the power of movement. I am strangely tired, and inclined to doze. My eyes open and close, and, presently, I find myself falling asleep, and waking, in fits and starts.

It is some time later, that I am sleepily aware that one of the candles is guttering. When I wake again, it has gone out, and the room is very dim, under the light of the one remaining flame. The semi-darkness troubles me little. I have lost that awful sense of dread, and my only desire seems to be to sleep---sleep.

Suddenly, although there is no noise, I am awake---wide awake. I am acutely conscious of the nearness of some mystery, of some overwhelming Presence. The very air seems pregnant with terror.\linebreak I sit huddled, and just listen, intently. Still, there is no sound. Nature, herself, seems dead. Then, the oppressive stillness is broken by a little eldritch scream of wind, that sweeps ’round the house, and dies away, remotely.

I let my gaze wander across the half-lighted room. By the great clock in the far corner, is a dark, tall shadow. For a short instant, I stare, frightenedly. Then, I see that it is nothing, and am, momentarily, relieved.

In the time that follows, the thought flashes through my brain, why not leave this house---this house of mystery and terror? Then, as though in answer, there sweeps up, across my sight, a vision of the wondrous Sea of Sleep,---the Sea of Sleep where she and I have been allowed to meet, after the years of separation and sorrow; and I know that I shall stay on here, whatever happens.

Through the side window, I note the somber blackness of the night. My glance wanders away, and ’round the room; resting on one shadowy object and another. Suddenly, I turn, and look at the window on my right; as I do so, I breathe quickly, and bend forward, with a frightened gaze at something outside the window, but close to the bars. I am looking at a vast, misty swine-face, over which fluctuates a flamboyant flame, of a greenish hue. It is the Thing from the arena.\footnote{Previously encountered in ch. 2.---Anon. Ed.} The quivering mouth seems to drip with a continual, phosphorescent slaver. The eyes are staring straight into the room, with an inscrutable expression. Thus, I sit rigidly---frozen.

The Thing has begun to move. It is turning, slowly, in my direction. Its face is coming ’round toward me. It sees me. Two huge, inhumanly human, eyes are looking through the dimness at me. I am cold with fear; yet, even now, I am keenly conscious, and note, in an irrelevant way, that the distant stars are blotted out by the mass of the giant face.

A fresh horror has come to me. I am rising from my chair, without the least intention. I am on my feet, and something is impelling me toward the door that leads out into the gardens.\linebreak I wish to stop; but cannot. Some immutable power is opposed to my will, and I go slowly forward, unwilling and resistant. My glance flies ’round the room, helplessly, and stops at the window. The great swine-face has disappeared, and I hear, again, that stealthy pad, pad, pad. It stops outside the door---the door toward which I am being compelled....

There succeeds a short, intense silence; then there comes a sound. It is the rattle of the latch, being slowly lifted. At that, I am filled with desperation. I will not go forward another step.\linebreak I make a vast effort to return; but it is, as though I press back, upon an invisible wall. I groan out loud, in the agony of my fear, and the sound of my voice is frightening. Again comes that rattle, and I shiver, clammily. I try---aye, fight and struggle, to hold back, \textit{back}; but it is no use....

I am at the door, and, in a mechanical way, I watch my hand go forward, to undo the topmost bolt. It does so, entirely without my volition. Even as I reach up toward the bolt, the door is violently shaken, and I get a sickly whiff of mouldy air, which seems to drive in through the interstices of the doorway. I draw the bolt back, slowly, fighting, dumbly, the while. It comes out of its socket, with a click, and I begin to shake, aguishly. There are two more; one at the bottom of the door; the other, a massive affair, is placed about the middle.

For, perhaps a minute, I stand, with my arms hanging slackly, by my sides. The influence to meddle with the fastenings of the door, seems to have gone. All at once, there comes the sudden rattle of iron, at my feet. I glance down, and realize, with an unspeakable terror, that my foot is pushing back the lower bolt. An awful sense of helplessness assails me.... The bolt comes out of its hold, with a slight, ringing sound and I stagger on my feet, grasping at the great, central bolt, for support. A minute passes, an eternity; then another------My God, help me! I am being forced to work upon the last fastening. \textit{I will not!} Better to die, than open to the Terror, that is on the other side of the door. Is there no escape ...? God help me, I have jerked the bolt half out of its socket! My lips emit a hoarse scream of terror, the bolt is three parts drawn, now, and still my unconscious hands work toward my doom. Only a fraction of steel, between my soul and That. Twice, I scream out in the supreme agony of my fear; then, with a mad effort, I tear my hands away. My eyes seem blinded. A great blackness is falling upon me. Nature has come to my rescue. I feel my knees giving. There is a loud, quick thudding upon the door, and I am falling, falling....

I must have lain there, at least a couple of hours. As I recover, I am aware that the other candle has burnt out, and the room is in an almost total darkness. I cannot rise to my feet, for I am cold, and filled with a terrible cramp. Yet my brain is clear, and there is no longer the strain of that unholy influence.

Cautiously, I get upon my knees, and feel for the central bolt. I find it, and push it securely back into its socket; then the one at the bottom of the door. By this time, I am able to rise to my feet, and so manage to secure the fastening at the top. After that, I go down upon my knees, again, and creep away among the furniture, in the direction of the stairs. By doing this, I am safe from observation from the window.

I reach the opposite door, and, as I leave the study, cast one nervous glance over my shoulder, toward the window. Out in the night, I seem to catch a glimpse of something impalpable; but it may be only a fancy. Then, I am in the passage, and on the stairs.

Reaching my bedroom, I clamber into bed, all clothed as I am, and pull the bedclothes over me. There, after awhile, I begin to regain a little confidence. It is impossible to sleep; but I am grateful for the added warmth of the bedclothes. Presently, I try to think over the happenings of the past night; but, though I cannot sleep, I find that it is useless, to attempt consecutive thought. My brain seems curiously blank.

Toward morning, I begin to toss, uneasily. I cannot rest, and, after awhile, I get out of bed, and pace the floor. The wintry dawn is beginning to creep through the windows, and shows the bare discomfort of the old room. Strange, that, through all these years, it has never occurred to me how dismal the place really is.

After a time, I go to the window, and, opening it, look out. The sun is now above the horizon, and the air, though cold, is sweet and crisp. Gradually, my brain clears, and a sense of security, for the time being, comes to me. Somewhat happier, I go down stairs, and out into the garden, to have a look at the dog.

As I approach the kennel, I am greeted by the same mouldy stench that assailed me at the door last night. Shaking off a momentary sense of fear, I call to the dog; but he takes no heed, and, after calling once more, I throw a small stone into the kennel. At this, he moves, uneasily, and I shout his name, again; but do not go closer.

In a little the poor beast rises, and shambles out lurching queerly. In the daylight he stands swaying from side to side, and blinking stupidly. I look and note that the horrid wound is larger, much larger, and seems to have a whitish, fungoid appearance. It is impossible to tell what may be the matter with him; and it is well to be cautious.

The day slips away, uneventfully; and night comes on. I have determined to repeat my experiment of last night. I cannot say that it is wisdom; yet my mind is made up. Still, however, I have taken precautions; for I have driven stout nails in at the back of each of the three bolts, that secure the door, opening from the study into the gardens. This will, at least, prevent a recurrence of the danger I ran last night.

From ten to about two-thirty, I watch; but nothing occurs; and, finally, I stumble off to bed, where I am soon asleep.

\clearpage
