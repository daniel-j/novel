% File borderland-06.txt
% Version 2017/09/18
% In the original, this was Chapter XII.
% Heavily edited here.

\clearpage
\label{ch:06}

\begin{ChapterStart}
\null\null
\ChapterTitle{6. The Subterranean Pit}
\end{ChapterStart}

Another week came and went, during which I spent a great deal of my time about the ravine. I had come to the conclusion a few days earlier, that the arched hole, in the angle of the great rift, was the place through which the Swine-things had made their exit, from some unholy place in the bowels of the world. How near the probable truth this went, I was to learn later.

It may be easily understood, that I was tremendously curious, though in a frightened way, to know to what infernal place that hole led; though, so far, the idea had not struck me, seriously, of making an investigation. I was far too much imbued with a sense of horror of the Swine-creatures, to think of venturing, willingly, where there was any chance of coming into contact with them.

Gradually, however, as time passed, this feeling grew insensibly less; so that when, a few days later, the thought occurred to me that it might be possible to clamber down and have a look into the hole, I was not so exceedingly averse to it, as might have been imagined. Still, I do not think, even then, that I really intended to try any such foolhardy adventure. And yet, such is the pertinacity of human curiosity, that, at last, my chief desire was but to discover what lay beyond that gloomy entrance.

Slowly, as the days slid by, my fear of the Swine-things became an emotion of the past---more an unpleasant, incredible memory, than aught else.

Thus, a day came, when, throwing thoughts and fancies adrift, I procured a rope from the house, and, having made it fast to a stout tree, at the top of the rift, and some little distance back from the ravine edge, let the other end down into the cleft, until it dangled right across the mouth of the dark hole.

Then, cautiously, and with many misgivings as to whether it was not a mad act that I was attempting, I climbed slowly down, using the rope as a support, until I reached the hole. Here, still holding on to the rope, I stood, and peered in. All was perfectly dark, and not a sound came to me. Yet, a moment later, it seemed that I could hear something. I held my breath, and listened; but all was silent as the grave, and I breathed freely once more. At the same instant, I heard the sound again. It was like a noise of labored breathing---deep and sharp-drawn. For a short second, I stood, petrified; not able to move. But now the sounds had ceased again, and I could hear nothing.

As I stood there, anxiously, my foot dislodged a pebble, which fell inward, into the dark, with a hollow chink. At once, the noise was taken up and repeated a score of times; each succeeding echo being fainter, and seeming to travel away from me, as though into remote distance. Then, as the silence fell again, I heard that stealthy breathing. For each respiration I made, I could hear an answering breath. The sounds appeared to be coming nearer; and then, I heard several others; but fainter and more distant. Why I did not grip the rope, and spring up out of danger, I cannot say. It was as though I had been paralyzed. I broke out into a profuse sweat, and tried to moisten my lips with my tongue. My throat had gone suddenly dry, and I coughed, huskily. It came back to me, in a dozen, horrible, throaty tones, mockingly. I peered, helplessly, into the gloom; but still nothing showed. I had a strange, choky sensation, and again I coughed, dryly. Again the echo took it up, rising and falling, grotesquely, and dying slowly into a muffled silence.

Then, suddenly, a thought came to me, and I held my breath. The other breathing stopped. I breathed again, and, once more, it re-commenced. But now, I no longer feared. I knew that the strange sounds were not made by any lurking Swine-creature; but were simply the echo of my own respirations.

Yet, I had received such a fright, that I was glad to scramble up the rift, and haul up the rope. I was far too shaken and nervous to think of entering that dark hole then, and so returned to the house. I felt more myself next morning; but even then, I could not summon up sufficient courage to explore the place.

All this time, the water in the ravine had been creeping slowly up, and now stood but a little below the opening. At the rate at which it was rising, it would be level with the floor in less than another week; and I realized that, unless I carried out my investigations soon, I should probably never do so at all; as the water would rise and rise, until the opening, itself, was submerged.

It may have been that this thought stirred me to act; but, whatever it was, a couple of days later, saw me standing at the top of the cleft, fully equipped for the task.

This time, I was resolved to conquer my shirking, and go right through with the matter. With this intention, I had brought, in addition to the rope, a bundle of candles, meaning to use them as a torch; also my double-barreled shotgun. In my belt, I had a heavy horse-pistol, loaded with buckshot.

As before, I fastened the rope to the tree. Then, having tied my gun across my shoulders, with a piece of stout cord, I lowered myself over the edge of the ravine. At this movement, Pepper, who had been eyeing my actions, watchfully, rose to his feet, and ran to me, with a half bark, half wail, it seemed to me, of warning. But I was resolved on my enterprise, and bade him lie down. I would much have liked to take him with me; but this was next to impossible, in the existing circumstances. As my face dropped level with the ravine edge, he licked me, right across the mouth; and then, seizing my sleeve between his teeth, began to pull back, strongly. It was very evident that he did not want me to go. Yet, having made up my mind, I had no intention of giving up the attempt; and, with a sharp word to Pepper, to release me, I continued my descent, leaving the poor old fellow at the top, barking and crying like a forsaken pup.

Carefully, I lowered myself from projection to projection. I knew that a slip might mean a wetting.

Reaching the entrance, I let go the rope, and untied the gun from my shoulders. Then, with a last look at the sky---which I noticed was clouding over, rapidly---I went forward a couple of paces, so as to be shielded from the wind, and lit one of the candles. Holding it above my head, and grasping my gun, firmly, I began to move on, slowly, throwing my glances in all directions.

For the first minute, I could hear the melancholy sound of Pepper’s howling, coming down to me. Gradually, as I penetrated further into the darkness, it grew fainter; until, in a little while, I could hear nothing. The path tended downward somewhat, and to the left. Thence it kept on, still running to the left, until I found that it was leading me right in the direction of the house.

Very cautiously, I moved onward, stopping, every few steps, to listen. I had gone, perhaps, a hundred yards, when, suddenly, it seemed to me that I caught a faint sound, somewhere along the passage behind. With my heart thudding heavily, I listened. The noise grew plainer, and appeared to be approaching, rapidly. I could hear it distinctly, now. It was the soft padding of running feet. In the first moments of fright, I stood, irresolute; not knowing whether to go forward or backward. Then, with a sudden realization of the best thing to do, I backed up to the rocky wall on my right, and, holding the candle above my head, waited---gun in hand---cursing my foolhardy curiosity, for bringing me into such a strait.

I had not long to wait, but a few seconds, before two eyes reflected back from the gloom, the rays of my candle. I raised my gun, using my right hand only, and aimed quickly. Even as I did so, something leapt out of the darkness, with a blustering bark of joy that woke the echoes, like thunder. It was Pepper. How he had contrived to scramble down the cleft, I could not conceive. As I brushed my hand, nervously, over his coat, I noticed that he was dripping; and concluded that he must have tried to follow me, and fallen into the water; from which he would not find it very difficult to climb.

Having waited a minute, or so, to steady myself, I proceeded along the way, Pepper following, quietly. I was curiously glad to have the old fellow with me. He was company, and, somehow, with him at my heels, I was less afraid. Also, I knew how quickly his keen ears would detect the presence of any unwelcome creature, should there be such, amid the darkness that wrapped us.

For some minutes we went slowly along; the path still leading straight toward the house. Soon, I concluded, we should be standing right beneath it, did the path but carry far enough. I led the way, cautiously, for another fifty yards, or so. Then, I stopped, and held the light high; and reason enough I had to be thankful that I did so; for there, not three paces forward, the path vanished, and, in place, showed a hollow blackness, that sent sudden fear through me.

Very cautiously, I crept forward, and peered down; but could see nothing. Then, I crossed to the left of the passage, to see whether there might be any continuation of the path. Here, right against the wall, I found that a narrow track, some three feet wide, led onward. Carefully, I stepped on to it; but had not gone far, before I regretted venturing thereon. For, after a few paces, the already narrow way, resolved itself into a mere ledge, with, on the one side the solid, unyielding rock, towering up, in a great wall, to the unseen roof, and, on the other, that yawning chasm.

To my great relief, a little further on, the track suddenly broadened out again to its original breadth. Gradually, as I went onward, I noticed that the path trended steadily to the right, and so, after some minutes, I discovered that I was not going forward; but simply circling the huge abyss. I had, evidently, come to the end of the great passage.

Five minutes later, I stood on the spot from which I had started; having been completely ‘round, what I guessed now to be a vast Pit, the mouth of which must be at least a hundred yards across.

For some little time, I stood there, lost in perplexing thought. ‘What does it all mean?’ was the cry that had begun to reiterate through my brain.

A sudden idea struck me, and I searched ‘round for a piece of stone. Presently, I found a bit of rock, about the size of a small loaf. Sticking the candle upright in a crevice of the floor, I went back from the edge, somewhat, and, taking a short run, launched the stone forward into the chasm---my idea being to throw it far enough to keep it clear of the sides. Then, I stooped forward, and listened; but, though I kept perfectly quiet, for at least a full minute, no sound came back to me from out of the dark.

I knew, then, that the depth of the hole must be immense; for the stone, had it struck anything, was large enough to have set the echoes of that weird place, whispering for an indefinite period. Even as it was, the cavern had given back the sounds of my footfalls, multitudinously. The place was awesome, and I would willingly have retraced my steps, and left the mysteries of its solitudes unsolved; only, to do so, meant admitting defeat.

Then, a thought came, to try to get a view of the abyss. It occurred to me that, if I placed my candles ‘round the edge of the hole, I should be able to get, at least, some dim sight of the place.

I found, on counting, that I had brought fifteen candles, in the bundle---my first intention having been, as I have already said, to make a torch of the lot. These, I proceeded to place ‘round the Pit mouth, with an interval of about twenty yards between each.

Having completed the circle, I stood in the passage, and endeavored to get an idea of how the place looked. But I discovered, immediately, that they were totally insufficient for my purpose. They did little more than make the gloom visible. One thing they did, however, and that was, they confirmed my opinion of the size of the opening; and, although they showed me nothing that I wanted to see; yet the contrast they afforded to the heavy darkness, pleased me, curiously. It was as though fifteen tiny stars shone through the subterranean night.

Then, even as I stood, Pepper gave a sudden howl, that was taken up by the echoes, and repeated with ghastly variations, dying away, slowly. With a quick movement, I held aloft the one candle that I had kept, and glanced down at the dog; at the same moment, I seemed to hear a noise, like a diabolical chuckle, rise up from the hitherto, silent depths of the Pit. I started; then, I recollected that it was, probably, the echo of Pepper’s howl.

Pepper had moved away from me, up the passage, a few steps; he was nosing along the rocky floor; and I thought I heard him lapping. I went toward him, holding the candle low. As I moved, I heard my boot go sop, sop; and the light was reflected from something that glistened, and crept past my feet, swiftly toward the Pit. I bent lower, and looked; then gave vent to an expression of surprise. From somewhere, higher up the path, a stream of water was running quickly in the direction of the great opening, and growing in size every second.

Again, Pepper gave vent to that deep-drawn howl, and, running at me, seized my coat, and attempted to drag me up the path toward the entrance.

And now I understood the cause of the catastrophe. It was raining heavily, literally in torrents. The surface of the lake was level with the bottom of the opening---nay! more than level, it was above it. Evidently, the rain had swollen the lake, and caused this premature rise; for, at the rate the ravine had been filling, it would not have reached the entrance for a couple more days.

Luckily, the rope by which I had descended, was streaming into the opening, upon the inrushing waters. Seizing the end, I knotted it securely ‘round Pepper’s body, then, summoning up the last remnant of my strength, I commenced to swarm up the side of the cliff. I reached the ravine edge, in the last stage of exhaustion. Yet, I had to make one more effort, and haul Pepper into safety.

Slowly and wearily, I hauled on the rope. Once or twice, it seemed that I should have to give up; for Pepper is a weighty dog, and I was utterly done. Yet, to let go, would have meant certain death to the old fellow, and the thought spurred me to greater exertions. I have but a very hazy remembrance of the end. I recall pulling, through moments that lagged strangely. I have also some recollection of seeing Pepper’s muzzle, appearing over the ravine edge, after what seemed an indefinite period of time. Then, all grew suddenly dark.

\clearpage

