% File borderland-09.txt
% Version 2017/09/18
% In the original, this was Chapter XV.

\clearpage
\label{ch:09}

\begin{ChapterStart}
\null\null
\ChapterTitle{9. The Noise in the Night}
\end{ChapterStart}

And now, I come to the strangest of all the strange happenings that have befallen me in this house of mysteries. It occurred quite lately---within the month; and I have little doubt but that what I saw was in reality the end of all things. However, to my story.

I do not know how it is; but, up to the present, I have never been able to write these things down, directly they happened. It is as though I have to wait a time, recovering my just balance, and digesting---as it were---the things I have heard or seen. No doubt, this is as it should be; for, by waiting, I see the incidents more truly, and write of them in a calmer and more judicial frame of mind.

It is now the end of November. My story relates to what happened in the first week of the month.

It was night, about eleven o’clock. Pepper and I kept one another company in the study---that great, old room of mine, where I read and work. I was reading, curiously enough, the Bible. I have begun, in these later days, to take a growing interest in that great and ancient book. Suddenly, a distinct tremor shook the house, and there came a faint and distant, whirring buzz, that grew rapidly into a far, muffled screaming. It reminded me, in a queer, gigantic way, of the noise that a clock makes, when the catch is released, and it is allowed to run down. The sound appeared to come from some remote height---somewhere up in the night. There was no repetition of the shock. I looked across at Pepper. He was sleeping peacefully.

Gradually, the whirring noise decreased, and there came a long silence.

All at once, a glow lit up the end window, which protrudes far out from the side of the house, so that, from it, one may look both East and West. I felt puzzled, and, after a moment’s hesitation, walked across the room, and pulled aside the blind. As I did so, I saw the Sun rise, from behind the horizon. It rose with a steady, perceptible movement. I could see it travel upward. In a minute, it seemed, it had reached the tops of the trees, through which I had watched it. Up, up---It was broad daylight now. Behind me, I was conscious of a sharp, mosquito-like buzzing. I glanced ’round, and knew that it came from the clock. Even as I looked, it marked off an hour. The minute hand was moving ’round the dial, faster than an ordinary second-hand. The hour hand moved quickly from space to space. I had a numb sense of astonishment. A moment later, so it seemed, the two candles went out, almost together. I turned swiftly back to the window; for I had seen the shadow of the window-frames, traveling along the floor toward me, as though a great lamp had been carried up past the window.

I saw now, that the sun had risen high into the heavens, and was still visibly moving. It passed above the house, with an extraordinary sailing kind of motion. As the window came into shadow, I saw another extraordinary thing. The fine-weather clouds were not passing, easily, across the sky---they were scampering, as though a hundred-mile-an-hour wind blew. As they passed, they changed their shapes a thousand times a minute, as though writhing with a strange life; and so were gone. And, presently, others came, and whisked away likewise.

To the West, I saw the sun, drop with an incredible, smooth, swift motion. Eastward, the shadows of every seen thing crept toward the coming greyness. And the movement of the shadows was visible to me---a stealthy, writhing creep of the shadows of the wind-stirred trees. It was a strange sight.

Quickly, the room began to darken. The sun slid down to the horizon, and seemed, as it were, to disappear from my sight, almost with a jerk. Through the greyness of the swift evening, I saw the silver crescent of the moon, falling out of the Southern sky, toward the West. The evening seemed to merge into an almost instant night. Above me, the many constellations passed in a strange, ‘noiseless’ circling, Westward. The moon fell through that last thousand fathoms of the night-gulf, and there was only the starlight....

About this time, the buzzing in the corner ceased; telling me that the clock had run down. A few minutes passed, and I saw the Eastward sky lighten. A grey, sullen morning spread through all the darkness, and hid the march of the stars. Overhead, there moved, with a heavy, everlasting rolling, a vast, seamless sky of grey clouds---a cloud-sky that would have seemed motionless, through all the length of an ordinary earth-day. The sun was hidden from me; but, from moment to moment, the world would brighten and darken, brighten and darken, beneath waves of subtle light and shadow....

The light shifted ever Westward, and the night fell upon the earth. A vast rain seemed to come with it, and a wind of a most extraordinary loudness---as though the howling of a nightlong gale, were packed into the space of no more than a minute.

This noise passed, almost immediately, and the clouds broke; so that, once more, I could see the sky. The stars were flying Westward, with astounding speed. It came to me now, for the first time, that, though the noise of the wind had passed, yet a constant ‘blurred’ sound was in my ears. Now that I noticed it, I was aware that it had been with me all the time. It was the world-noise.

And then, even as I grasped at so much comprehension, there came the Eastward light. No more than a few heartbeats, and the sun rose, swiftly. Through the trees, I saw it, and then it was above the trees. Up---up, it soared and all the world was light. It passed, with a swift, steady swing to its highest altitude, and fell thence, Westward. I saw the day roll visibly over my head. A few light clouds flittered Northward, and vanished. The sun went down with one swift, clear plunge, and there was about me, for a few seconds, the darker growing grey of the gloaming.

Southward and Westward, the moon was sinking rapidly. The night had come, already. A minute it seemed, and the moon fell those remaining fathoms of dark sky. Another minute, or so, and the Eastward sky glowed with the coming dawn. The sun leapt upon me with a frightening abruptness, and soared ever more swiftly toward the zenith. Then, suddenly, a fresh thing came to my sight. A black thundercloud rushed up out of the South, and seemed to leap all the arc of the sky, in a single instant. As it came, I saw that its advancing edge flapped, like a monstrous black cloth in the heaven, twirling and undulating rapidly, with a horrid suggestiveness. In an instant, all the air was full of rain, and a hundred lightning flashes seemed to flood downward, as it were in one great shower. In the same second of time, the world-noise was drowned in the roar of the wind, and then my ears ached, under the stunning impact of the thunder.

And, in the midst of this storm, the night came; and then, within the space of another minute, the storm had passed, and there was only the constant ‘blur’ of the world-noise on my hearing. Overhead, the stars were sliding quickly Westward; and something, mayhaps the particular speed to which they had attained, brought home to me, for the first time, a keen realization of the knowledge that it was the world that revolved. I seemed to see, suddenly, the world---a vast, dark mass---revolving visibly against the stars.

The dawn and sun seemed to come together, so greatly had the speed of the world-revolution increased. The sun drove up, in one long, steady curve; passed its highest point, swept down into the Western sky, and disappeared. I was scarcely conscious of evening, so brief was it. Then I was watching the flying constellations, and the Westward hastening moon. In but a space of seconds, so it seemed, it was sliding swiftly downward through the night-blue, and then was gone. And, almost directly, came the morning.

And now there seemed to come a strange acceleration. The sun made one clean, clear sweep through the sky, and disappeared behind the Westward horizon, and the night came and went with a like haste.

As the succeeding day, opened and closed upon the world, I was aware of a sweat of snow, suddenly upon the earth. The night came, and, almost immediately, the day. In the brief leap of the sun, I saw that the snow had vanished; and then, once more, it was night.

Thus matters were; and, even after the many incredible things that I have seen, I experienced all the time a most profound awe. To see the sun rise and set, within a space of time to be measured by seconds; to watch (after a little) the moon leap---a pale, and ever growing orb---up into the night sky, and glide, with a strange swiftness, through the vast arc of blue; and, presently, to see the sun follow, springing out of the Eastern sky, as though in chase; and then again the night, with the swift and ghostly passing of starry constellations, was all too much to view believingly. Yet, so it was---the day slipping from dawn to dusk, and the night sliding swiftly into day, ever rapidly and more rapidly.

The last three passages of the sun had shown me a snow-covered earth, which, at night, had seemed, for a few seconds, incredibly weird under the fast-shifting light of the soaring and falling moon. Now, however, for a little space, the sky was hidden, by a sea of swaying, leaden-white clouds, which lightened and blackened, alternately, with the passage of day and night.

The clouds rippled and vanished, and there was once more before me, the vision of the swiftly leaping sun, and nights that came and went like shadows.

Faster and faster, spun the world. And now each day and night was completed within the space of but a few seconds; and still the speed increased.

It was a little later, that I noticed that the sun had begun to have the suspicion of a trail of fire behind it. This was due, evidently, to the speed at which it, apparently, traversed the heavens. And, as the days sped, each one quicker than the last, the sun began to assume the appearance of a vast, flaming comet\footnote{The Recluse uses this as an illustration, evidently in the sense of the popular conception of a comet.---\allsmcp{WHH}} flaring across the sky at short, periodic intervals. At night, the moon presented, with much greater truth, a comet-like aspect; a pale, and singularly clear, fast traveling shape of fire, trailing streaks of cold flame. The stars showed now, merely as fine hairs of fire against the dark.

Once, I turned from the window, and glanced at Pepper. In the flash of a day, I saw that he slept, quietly, and I moved once more to my watching.

The sun was now bursting up from the Eastern horizon, like a stupendous rocket, seeming to occupy no more than a second or two in hurling from East to West. I could no longer perceive the passage of clouds across the sky, which seemed to have darkened somewhat. The brief nights, appeared to have lost the proper darkness of night; so that the hair-like fire of the flying stars, showed but dimly. As the speed increased, the sun began to sway very slowly in the sky, from South to North, and then, slowly again, from North to South.

So, amid a strange confusion of mind, the hours passed.

All this while had Pepper slept. Presently, feeling lonely and distraught, I called to him, softly; but he took no notice. Again, I called, raising my voice slightly; still he moved not. I walked over to where he lay, and touched him with my foot, to rouse him. At the action, gentle though it was, he fell to pieces. That is what happened; he literally and actually crumbled into a mouldering heap of bones and dust.

For the space of, perhaps a minute, I stared down at the shapeless heap, that had once been Pepper. I stood, feeling stunned. What can have happened? I asked myself; not at once grasping the grim significance of that little hill of ash. Then, as I stirred the heap with my foot, it occurred to me that this could only happen in a great space of time. Years---and years.

Outside, the weaving, fluttering light held the world. Inside, I stood, trying to understand what it meant---what that little pile of dust and dry bones, on the carpet, meant. But I could not think, coherently.

I glanced away, ’round the room, and now, for the first time, noticed how dusty and old the place looked. Dust and dirt everywhere; piled in little heaps in the corners, and spread about upon the furniture. The very carpet, itself, was invisible beneath a coating of the same, all pervading, material. As I walked, little clouds of the stuff rose up from under my footsteps, and assailed my nostrils, with a dry, bitter odor that made me wheeze, huskily.

Suddenly, as my glance fell again upon Pepper’s remains, I stood still, and gave voice to my confusion---questioning, aloud, whether the years were, indeed, passing; whether this, which I had taken to be a form of vision, was, in truth, a reality. I paused. A new thought had struck me. Quickly, but with steps which, for the first time, I noticed, tottered, I went across the room to the great pier-glass, and looked in. It was too covered with grime, to give back any reflection, and, with trembling hands, I began to rub off the dirt. Presently, I could see myself. The thought that had come to me, was confirmed. Instead of the great, hale man, who scarcely looked fifty, I was looking at a bent, decrepit man, whose shoulders stooped, and whose face was wrinkled with the years of a century. The hair---which a few short hours ago had been nearly coal black---was now silvery white. Only the eyes were bright. Gradually, I traced, in that ancient man, a faint resemblance to my self of other days.

I turned away, and tottered to the window. I knew, now, that I was old, and the knowledge seemed to confirm my trembling walk. For a little space, I stared moodily out into the blurred vista of changeful landscape. Even in that short time, a year passed, and, with a petulant gesture, I left the window. As I did so, I noticed that my hand shook with the palsy of old age; and a short sob choked its way through my lips.

For a little while, I paced, tremulously, between the window and the table; my gaze wandering hither and thither, uneasily. How dilapidated the room was. Everywhere lay the thick dust. The fender was a shape of rust. The chains that held the brass clock-weights, had rusted through long ago, and now the weights lay on the floor beneath; themselves two cones of verdigris.

As I glanced about, it seemed to me that I could see the very furniture of the room rotting and decaying before my eyes. Nor was this fancy, on my part; for, all at once, the bookshelf, along the sidewall, collapsed, with a cracking and rending of rotten wood, precipitating its contents upon the floor, and filling the room with a smother of dusty atoms.

How tired I felt. As I walked, it seemed that I could hear my dry joints, creak and crack at every step. All had happened so quickly and suddenly. This must be, indeed, the beginning of the end of all things! It occurred to me, to go to look for her; but I felt too weary. And then, she had been so queer about these happenings, of late. Of late! I repeated the words, and laughed, feebly---mirthlessly, as the realization was borne in upon me that I spoke of a time, half a century gone. Half a century! It might have been twice as long!

I moved slowly to the window, and looked out once more across the world. I can best describe the passage of day and night, at this period, as a sort of gigantic, ponderous flicker. Moment by moment, the acceleration of time continued; so that, at nights now, I saw the moon, only as a swaying trail of palish fire, that varied from a mere line of light to a nebulous path, and then dwindled again, disappearing periodically.

The flicker of the days and nights quickened. The days had grown perceptibly darker, and a queer quality of dusk lay, as it were, in the atmosphere. The nights were so much lighter, that the stars were scarcely to be seen, saving here and there an occasional hair-like line of fire, that seemed to sway a little, with the moon.

Quicker, and ever quicker, ran the flicker of day and night; and, suddenly it seemed, I was aware that the flicker had died out, and, instead, there reigned a comparatively steady light, which was shed upon all the world, from an eternal river of flame that swung up and down, North and South, in stupendous, mighty swings.

The sky was now grown very much darker, and there was in the blue of it a heavy gloom, as though a vast blackness peered through it upon the earth. Yet, there was in it, also, a strange and awful clearness, and emptiness. Periodically, I had glimpses of a ghostly track of fire that swayed thin and darkly toward the sun-stream; vanished and reappeared. It was the scarcely visible moon-stream.

Looking out at the landscape, I was conscious again, of a blurring ‘flitter,’ that came either from the light of the ponderous, swinging sun-stream, or was the result of the incredibly rapid changes of the earth’s surface. And every few moments, so it seemed, the snow would lie suddenly upon the world, and vanish as abruptly, as though an invisible giant ‘flitted’ a white sheet off and on the earth.

Time fled, and the weariness that was mine, grew insupportable. I turned from the window, and walked once across the room, the heavy dust deadening the sound of my footsteps. Each step that I took, seemed a greater effort than the one before. An intolerable ache, knew me in every joint and limb, as I trod my way, with a weary uncertainty.

By the opposite wall, I came to a weak pause, and wondered, dimly, what was my intent. I looked to my left, and saw my old chair. The thought of it brought a faint sense of comfort to my bewildered wretchedness. Yet, because I was so weary and old and tired, I would scarcely brace my mind to do anything but stand, and wish myself past those few yards. I rocked, as I stood. The floor, even, seemed a place for rest; but the dust lay so thick and sleepy and black. I turned, with a great effort of will, and made toward my chair. I reached it, with a groan of thankfulness. I sat down.

Everything about me appeared to be growing dim. It was all so strange and unthought of. Last night, I was a comparatively strong, though elderly man; and now, only a few hours later---! I looked at the little dust-heap that had once been Pepper. Hours! and I laughed, a feeble, bitter laugh; a shrill, cackling laugh, that shocked my dimming senses.

For a while, I must have dozed. Then I opened my eyes, with a start. Somewhere across the room, there had been a muffled noise of something falling. I looked, and saw, vaguely, a cloud of dust hovering above a pile of \textit{débris}. Nearer the door, something else tumbled, with a crash. It was one of the cupboards; but I was tired, and took little notice. I closed my eyes, and sat there in a state of drowsy, semi-unconsciousness. Once or twice---as though coming through thick mists---I heard noises, faintly. Then I must have slept.

\clearpage
