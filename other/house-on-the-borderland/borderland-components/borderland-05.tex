% File borderland-05.txt
% Version 2017/09/18
% In the original, this was Chapter IX.
% Intervening adventure with swine-creatures omitted.

\clearpage
\label{ch:05}

\begin{ChapterStart}
\null\null
\ChapterTitle{5. The Cellars}
\end{ChapterStart}


At last, what with being tired and cold, and the uneasiness that possessed me, I resolved to take a walk through the house; first calling in at the study, for a glass of brandy to warm me. This, I did, and, while there, I examined the door, carefully; but found all as I had left it the night before.

The day was just breaking, as I left the tower; though it was still too dark in the house to be able to see without a light, and I took one of the study candles with me on my ’round. By the time I had finished the ground floor, the daylight was creeping in, wanly, through the barred windows. My search had shown me nothing fresh. Everything appeared to be in order, and I was on the point of extinguishing my candle, when the thought suggested itself to me to have another glance ’round the cellars.

For, perhaps, the half of a minute, I hesitated. I would have been very willing to forego the task---as, indeed, I am inclined to think any man well might---for of all the great, awe-inspiring rooms in this house, the cellars are the hugest and weirdest. Great, gloomy caverns of places, unlit by any ray of daylight. Yet, I would not shirk the work. I felt that to do so would smack of sheer cowardice. Besides, as I reassured myself, the cellars were really the most unlikely places in which to come across anything dangerous; considering that they can be entered, only through a heavy oaken door, the key of which, I carry always on my person.

It is in the smallest of these places that I keep my wine; a gloomy hole close to the foot of the cellar stairs; and beyond which, I have seldom proceeded. Indeed, save for the rummage ’round, already mentioned, I doubt whether I had ever, before, been right through the cellars.

As I unlocked the great door, at the top of the steps, I paused, nervously, a moment, at the strange, desolate smell that assailed my nostrils. Then, throwing the barrel of my weapon forward, I descended, slowly, into the darkness of the underground regions.

Reaching the bottom of the stairs, I stood for a minute, and listened. All was silent, save for a faint drip, drip of water, falling, drop-by-drop, somewhere to my left. As I stood, I noticed how quietly the candle burnt; never a flicker nor flare, so utterly windless was the place.

Quietly, I moved from cellar to cellar. I had but a very dim memory of their arrangement. The impressions left by my first search were blurred. I had recollections of a succession of great cellars, and of one, greater than the rest, the roof of which was upheld by pillars; beyond that my mind was hazy, and predominated by a sense of cold and darkness and shadows. Now, however, it was different; for, although nervous, I was sufficiently collected to be able to look about me, and note the structure and size of the different vaults I entered.

Of course, with the amount of light given by my candle, it was not possible to examine each place, minutely, but I was enabled to notice, as I went along, that the walls appeared to be built with wonderful precision and finish; while here and there, an occasional, massive pillar shot up to support the vaulted roof.

Thus, I came, at last, to the great cellar that I remembered. It is reached, through a huge, arched entrance, on which I observed strange, fantastic carvings, which threw queer shadows under the light of my candle. As I stood, and examined these, thoughtfully, it occurred to me how strange it was, that I should be so little acquainted with my own house. Yet, this may be easily understood, when one realizes the size of this ancient pile.

Holding the light high, I passed on into the cellar, and, keeping to the right, paced slowly up, until I reached the further end.\linebreak I walked quietly, and looked cautiously about, as I went. But, so far as the light showed, I saw nothing unusual.

At the top, I turned to the left, still keeping to the wall, and so continued, until I had traversed the whole of the vast chamber. As I moved along, I noticed that the floor was composed of solid rock, in places covered with a damp mould, in others bare, or almost so, save for a thin coating of light-grey dust.

I had halted at the doorway. Now, however, I turned, and made my way up the center of the place; passing among the pillars, and glancing to right and left, as I moved. About halfway up the cellar, I stubbed my foot against something that gave out a metallic sound. Stooping quickly, I held the candle, and saw that the object I had kicked, was a large, metal ring. Bending lower, I cleared the dust from around it, and, presently, discovered that it was attached to a ponderous trap door, black with age.

Feeling excited, and wondering to where it could lead, I laid my gun on the floor, and, sticking the candle in the trigger guard, took the ring in both hands, and pulled. The trap creaked loudly---the sound echoing, vaguely, through the huge place---and opened, heavily.

Propping the edge on my knee, I reached for the candle, and held it in the opening, moving it to right and left; but could see nothing. I was puzzled and surprised. There were no signs of steps, nor even the appearance of there ever having been any. Nothing; save an empty blackness. I might have been looking down into a bottomless, sideless well. Then, even as I stared, full of perplexity, I seemed to hear, far down, as though from untold depths, a faint whisper of sound. I bent my head, quickly, more into the opening, and listened, intently. It may have been fancy; but I could have sworn to hearing a soft titter, that grew into a hideous, chuckling, faint and distant. Startled, I leapt backward, letting the trap fall, with a hollow clang, that filled the place with echoes. Even then, I seemed to hear that mocking, suggestive laughter; but this, I knew, must be my imagination. The sound, I had heard, was far too slight to penetrate through the cumbrous trap.

For a full minute, I stood there, quivering---glancing, nervously, behind and before; but the great cellar was silent as a grave, and, gradually, I shook off the frightened sensation. With a calmer mind, I became again curious to know into what that trap opened; but could not, then, summon sufficient courage to make a further investigation. One thing I felt, however, was that the trap ought to be secured. This, I accomplished by placing upon it several large pieces of ‘dressed’ stone, which I had noticed in my tour along the East wall.

Then, after a final scrutiny of the rest of the place, I retraced my way through the cellars, to the stairs, and so reached the daylight, with an infinite feeling of relief, that the uncomfortable task was accomplished.

\clearpage
