% !TeX TS-program = LuaLaTeX
% !TeX encoding = UTF-8
\documentclass{novel} % v. 1.40.1 or later
% Note that just about all of the setup uses defaults!
\SetTitle{The House on the Borderland}
% Abriged Version
\SetAuthor{William Hope Hodgson}
\SetPDFX[CGATSTR001]{X-1a:2001} % US Web Coated SWOP v2. Profile not embedded.

\begin{document}

% The House on the Borderland, abridged version
% Version 2017/09/18
% This is a demonstration of LuaLaTeX `novel' document class.


% ----------------------------------------------------------------------------


\frontmatter % Sets page to lowercase roman i.


% Half-Title, page i
\thispagestyle{empty}
% \nbs is normal baseline skip.
\vspace*{6\nbs} % Note asterisk. Without asterisk, does not force space, when at top of page.
{\centering\charscale[1.4]{The House on the Borderland}\par}
\cleartorecto % So that next page ii is blank.


% Title, page iii
\thispagestyle{empty}
\vspace*{4\nbs}
{\centering\charscale[4]{The House on}\par}
\vspace{2\nbs} % Could use asterisk; doesn't matter mid-page, for this command.
{\centering\charscale[4]{the Borderland}\par}
\vspace{2\nbs}
% In that era, books sometimes had elaborate subtitles, such as the following.
% In our era, this is rarely done.
\begin{center}
\textit{From the Manuscript discovered in 1877\\
by Messrs. Tonnison and Berreggnog\\
in the Ruins that lie to the South of\\
the Village of Kraighten, in the West of Ireland.\\
Set out here, with Notes.}\par
\end{center}
\vspace{3\nbs}
{\centering\charscale[2.5]{William Hope Hodgson}\par}
\vspace{3\nbs}
{\centering\textsc{ABRIDGED VERSION 2017/09/18}\par}
\vfill % forces remainder to bottom of page
{\centering A LuaLaTeX Typesetting Demonstration\par}
\clearpage


% Copyright, page iv
\thispagestyle{empty}
\null
\vfill
\begin{legalese}
The main text of \textit{The House on the Borderland}, by William Hope Hodgson, is in the Public Domain of the United States of America because it was published in the U.S.A. prior to January 1, 1923. It is in the Public Domain of its creator’s home jurisdiction of the United Kingdom, because more than seventy years have passed since death of its creator in \lnum{1918}.\par
\null
Accompanying Cover Artwork, Foreword, and (to the extent covered by copyright law) book design and TeX markup, are Copyright ©\lnum{2017} Anonymous Editor, located in California, United States of America. These copyrights are hereby licensed under the LaTeX Project Public License, v. \lnum{1.3c}.\par
\null
This is a work of fiction. Places, events, and characters were created in the imagination of William Hope Hodgson.\par
\null
\allsmcp{ABRIDGED VERSION 2017/09/18.} This is a demonstration file for the LuaLaTeX \textit{novel} document class. It is not an authoritative edition. Substantial portions of the original work have been omitted, and portions of the included work have been edited.\par
\null
\allsmcp{DISCLAIMER:} Although the resulting PDF files are labeled as print-ready, there is \allsmcp{NO WARRANTY, EXPRESS OR IMPLIED} that they are actually suitable for printing or other publication.\par
\end{legalese}
\clearpage


% Dedication, page v.
% Although the poem, Shoon of the Dead, appears to be part of the original Dedication,
% I have moved it to a separate Epigraph.
% "Shoon" is an archaic word for "shoes."
\thispagestyle{empty}
\vspace*{6\nbs}
{\centering\charscale[1.2]{To My Father}\par}
\vspace{0.5\nbs}
{\centering\textit{(Whose feet tread the lost aeons)}\par}
\clearpage % Not \cleartorecto here, because Epigraph appears on verso.


% Epigraph, page vi
% This poem was apparently part of the original Dedication.
% I have makde it a separate Epigraph, due to its length,
% and also as a demonstration of how an Epigraph may sometimes be used verso instead of recto,
% particularly when facing a table of contents.
\thispagestyle{empty}
% Although the header does not yet appear, I reset the head texts now, lest they be forgotten.
\SetVersoHeadText{\textsc{William\, Hope\, Hodgson}}
\SetRectoHeadText{\textsc{The\, House\, on\, the\, Borderland}}
\vspace*{6\nbs}
\begin{adjustwidth}{7em}{0em}
\noindent Open the door,\\
And listen!\\
Only the wind’s muffled roar,\\
And the glisten\\
Of tears ’round the moon.\\
And, in fancy, the tread\\
Of vanishing shoon---\\
Out in the night with the Dead.\par
\null
\noindent Hush! And hark\\
To the sorrowful cry\\
Of the wind in the dark.\\
Hush and hark, without murmur or sigh,\\
To shoon that tread the lost aeons:\\
To the sound that bids you to die.\\
Hush and hark! Hush and Hark!\\
Shoon of the Dead\par
\end{adjustwidth}
\clearpage


% Table of Contents, page vii
\thispagestyle{empty}
\vspace*{2\nbs}
{\centering\charscale[1.4]{Contents}\par}
\null
\begin{toc}[0]{3em}
\tocitem*{\textsc{Foreword}}{ix}
\vspace{0.5\nbs}
\tocitem*{\textsc{Author's Note}}{xi}
\vspace{0.5\nbs}
\tocitem*{\textsc{Introduction.} The Finding of the Manuscript}{1}
\vspace{0.5\nbs}
\tocitem*[1]{The Plain of Silence}{13}
\tocitem*[2]{The House in the Arena}{18}
\tocitem*[3]{The Earth}{24}
\tocitem*[4]{The Thing in the Ravine}{27}
\tocitem*[5]{The Cellars}{33}
\tocitem*[6]{The Subterranean Pit}{37}
\tocitem*[7]{The Trap in the Great Cellar}{44}
\tocitem*[8]{The Sea of Sleep}{47}
\tocitem*[9]{The Noise in the Night}{51}
\tocitem*[10]{The Awakening}{60}
\tocitem*[11]{The Slowing Rotation}{65}
\tocitem*[12]{The Green Star}{71}
\tocitem*[13]{The End of the Solar System}{78}
\tocitem*[14]{The Celestial Globes}{82}
\tocitem*[15]{The Dark Sun}{85}
\tocitem*[16]{The Dark Nebula}{89}
\tocitem*[17]{Pepper}{93}
\tocitem*[18]{The Footsteps in the Garden}{94}
\tocitem*[19]{The Thing From the Arena}{97}
\tocitem*[20]{The Luminous Speck}{104}
\vspace{0.5\nbs}
\tocitem*{\textsc{Conclusion}}{107}
\end{toc}
\cleartorecto % So that next page viii is blank


% Foreword, pages ix-x
\thispagestyle{empty} % See also the use of this command at end of 2-page Foreword.
% A Preface or Foreword does not have author/title in the header.
% If it goes on for more than a page, or sometimes more than 2 pages,
% then the header will show Preface or Foreword verso/recto.
% Although this was not necessary in this document, due to having
% a short Foreword, it is general practice for longer material.
% In that case, use \SetVersoHeadText and \SetRectoHeadText.
% Then, remember to change them again!
\begin{ChapterStart}
\null\null
\ChapterTitle{Foreword}
\end{ChapterStart}
This abridged edition of \textit{The House on the Borderland} demonstrates the typesetting capabilities of LuaLaTeX, using the \textit{novel} document class. It is not an exact transcription of the original text, and should not be used as a reference for critical study.\par
Hodgson’s complete novel has several early chapters dealing with swine-creatures, who lurk in the Pit near his house. In the interest of keeping this edition brief, I have removed most of the adventures involving those creatures, leaving only a brief, early encounter. I have kept the one essential swine-thing that appears in the fantastic dimension and materialize later in the story. Although this removes some of the action adventure, I believe that the remaining story is sufficiently coherent.\par
The name “Pit” was originally used to mean both the large ravine, and the deep subterranean pit that lies underneath the house and has an entrance toward the ravine. For greater clarity, I have changed “Pit” to “ravine” when it does not refer to the subterranean pit, or to the demolished remains.\par
The Recluse was accompanied by his elderly sister, who appears from time to time in the original story, usually as a background element. I have removed her, and all references to her and her cat, as they were not essential. I believe the result places tighter focus in the Recluse and his frame of mind.\par
The original has portions that were supposed to be damaged or illegible, due to weathering of the manuscript prior to its discovery. Fragments were printed, along with notes to that effect. I did not think that the story was in any way advanced by their presence, other than as a pretense, so I removed them.\par
The poem “Grief” appeared at the end of the plain text version. I do not know its position or purpose in the original print. Thus, I have taken the liberty of placing it at the end of the chapter where the Recluse first mentions his lost love, as if it were part of the diary entry there. I believe it is consistent with the tone and flow at that point.\par
In Hodgson’s era, toward the tail end of Romanticism, many words were often used when few would have sufficed; an emotion rarely passed without being re-stated. From time to time throughout the book, I have deleted extraneous or redundant material, for the sake of directness and typeset appearance. The deletions may have been a few words, paragraphs, or entire chapters. In the remaining material, my own editing consists largely of necessary transitions, clarifications, and grammatical corrections due to splicing.\par
\null
\stake\hfill---Anonymous Editor\par
\stake\hfill California, September 2017\par
\thispagestyle{empty} % Retroactively blanks the header for this second page of Foreword.
\cleartorecto


% Author's Note (WHH), page xii.
% This page was "Author's Introduction to the Manuscript" in the plain-text version that I used.
% I have not seen the original book.
% It is somewhat problematic, because it is signed by WHH as author, yet it is fictional.
% Since it pretends to be a factual note by Hodgson, concerning material that came into his hands from someone else, I have decided to place it front matter, and re-title it as "Author's Note." This was a judgement call.
% At the time I write this, extra inter-word space must be added when small caps are used in headers. This can be achieved using \, to add a half-space. In the future, this behavior may change. It is not a property of novel package, but of an included package.
\begin{ChapterStart}[7]
\null\null
\ChapterTitle{Author’s Note}
\end{ChapterStart}
Many are the hours in which I have pondered upon the story that is set forth in the following pages. I trust that my instincts are not awry when they prompt me to leave the account, in simplicity, as it was handed to me.\par
And the MS. itself---You must picture me, when first it was given into my care, turning it over, curiously, and making a swift, jerky examination. A small book it is; but thick, and all, save the last few pages, filled with a quaint but legible handwriting, and writ very close. I have the queer, faint, pit-water smell of it in my nostrils now as I write, and my fingers have subconscious memories of the soft, “cloggy” feel of the long-damp pages.\par
I read, and, in reading, lifted the Curtains of the Impossible that blind the mind, and looked out into the unknown. Amid stiff, abrupt sentences I wandered; and, presently, I had no fault to charge against their abrupt tellings; for, better far than my own ambitious phrasing, is this mutilated story capable of bringing home all that the old Recluse, of the vanished house, had striven to tell.\par
Of the simple, stiffly given account of weird and extraordinary matters, I will say little. It lies before you. The inner story must be uncovered, personally, by each reader, according to ability and desire. And even should any fail to see, as now I see, the shadowed picture and conception of that to which one may well give the accepted titles of Heaven and Hell; yet can I promise certain thrills, merely taking the story as a story.\par
\null
\stake\hfill---William Hope Hodgson\par
\stake\hfill December 17, 1907\par


% -------------------------------------------------------------------------------------


\mainmatter % Includes \cleartorecto from novel v. 1.40.1.


% Advantage of components: Can compile in pieces during editing.
% Use \input, not \include.
% Comment-out lines that are not used during partial compilation.
% File borderland-00-introduction.txt.
% Version 2017/09/18
% According to the book's lengthy subtitle, the manuscript was discovered by "Messrs. Tonnison and Berreggnog."
% Therefore, Hodgson was not the discoverer, but he acquired the manuscript from someone else.
% In the following Introduction (originally Chapter I) we deduce that the voice must be that of Bereggnog, since Tonnison is separately mentioned.
% I have added "As Told by Mr. Bereggnog" to clarify this, and added "Begin Manuscript" at the end of the Introduction, so that the change of first-person narrator is clear.
% Since this Introduction is entirely fictional and and integral part of the story, its header shows author/title rather than Introduction.

\cleartorecto % this must begin recto

\label{ch:intro}

\begin{ChapterStart}
\null
\ChapterTitle{Introduction}
\null
\ChapterSubtitle{The Finding of the Manuscript}
\null
\ChapterDeco{As Told by Mr. Berreggnog}
\end{ChapterStart}


Right away in the west of Ireland lies a tiny hamlet called Kraighten. It is situated, alone, at the base of a low hill. Far around there spreads a waste of bleak and totally inhospitable country; where, here and there at great intervals, one may come upon the ruins of some long desolate cottage---unthatched and stark. The whole land is bare and unpeopled, the very earth scarcely covering the rock that lies beneath it, and with which the country abounds, in places rising out of the soil in wave-shaped ridges.

Yet, in spite of its desolation, my friend Tonnison and I had elected to spend our vacation there. He had stumbled on the place by mere chance the year previously, during the course of a long walking tour, and discovered the possibilities for the angler in a small and unnamed river that runs past the outskirts of the little village.

I have said that the river is without name; I may add that no map that I have hitherto consulted has shown either village or stream. They seem to have entirely escaped observation: indeed, they might never exist for all that the average guide tells one. Possibly this can be partly accounted for by the fact that the nearest railway station (Ardrahan) is some forty miles distant.

It was early one warm evening when my friend and I arrived in Kraighten. We had reached Ardrahan the previous night, sleeping there in rooms hired at the village post office, and leaving in good time on the following morning, clinging insecurely to one of the typical jaunting cars.

It had taken us all day to accomplish our journey over some of the roughest tracks imaginable, with the result that we were thoroughly tired and somewhat bad tempered. However, the tent had to be erected and our goods stowed away before we could think of food or rest. And so we set to work, with the aid of our driver, and soon had the tent up upon a small patch of ground just outside the little village, and quite near to the river.

Then, having stored all our belongings, we dismissed the driver, as he had to make his way back as speedily as possible, and told him to come across to us at the end of a fortnight. We had brought sufficient provisions to last us for that space of time, and water we could get from the stream. Fuel we did not need, as we had included a small oil-stove among our outfit, and the weather was fine and warm.

It was Tonnison’s idea to camp out instead of getting lodgings in one of the cottages. As he put it, there was no joke in sleeping in a room with a numerous family of healthy Irish in one corner and the pigsty in the other, while overhead a ragged colony of roosting fowls distributed their blessings impartially, and the whole place so full of peat smoke that it made a fellow sneeze his head off just to put it inside the doorway.

Tonnison had got the stove lit now and was busy cutting slices of bacon into the frying pan; so I took the kettle and walked down to the river for water. On the way, I had to pass close to a little group of the village people, who eyed me curiously, but not in any unfriendly manner, though none of them ventured a word.

As I returned with my kettle filled, I went up to them and, after a friendly nod, to which they replied in like manner, I asked them casually about the fishing; but, instead of answering, they just shook their heads silently, and stared at me. I repeated the question, addressing more particularly a great, gaunt fellow at my elbow; yet again I received no answer. Then the man turned to a comrade and said something rapidly in a language that I did not understand; and, at once, the whole crowd of them fell to jabbering in what, after a few moments, I guessed to be pure Irish. At the same time they cast many glances in my direction. For a minute, perhaps, they spoke among themselves thus; then the man I had addressed faced ’round at me and said something. By the expression of his face I guessed that he, in turn, was questioning me; but now I had to shake my head, and indicate that I did not comprehend what it was they wanted to know; and so we stood looking at one another, until I heard Tonnison calling to me to hurry up with the kettle. Then, with a smile and a nod, I left them, and all in the little crowd smiled and nodded in return, though their faces still betrayed their puzzlement.

It was evident, I reflected as I went toward the tent, that the inhabitants of these few huts in the wilderness did not know a word of English; and when I told Tonnison, he remarked that he was aware of the fact, and, more, that it was not at all uncommon in that part of the country, where the people often lived and died in their isolated hamlets without ever coming in contact with the outside world.

“I wish we had got the driver to interpret for us before he left,” I remarked, as we sat down to our meal. “It seems so strange for the people of this place not even to know what we’ve come for.”

Tonnison grunted an assent, and thereafter was silent for a while.

Later, having satisfied our appetites somewhat, we began to talk, laying our plans for the morrow; then, after a smoke, we closed the flap of the tent, and prepared to turn in.

“I suppose there’s no chance of those fellows outside taking anything?” I asked, as we rolled ourselves in our blankets.

Tonnison said that he did not think so, at least while we were about; and, as he went on to explain, we could lock up everything, except the tent, in the big chest that we had brought to hold our provisions. I agreed to this, and soon we were both asleep.

Next morning, early, we rose and went for a swim in the river; after which we dressed and had breakfast. Then we roused out our fishing tackle and overhauled it, by which time, our breakfasts having settled somewhat, we made all secure within the tent and strode off in the direction my friend had explored on his previous visit.

During the day we fished happily, working steadily upstream, and by evening we had one of the prettiest creels of fish that I had seen for a long while. Returning to the village, we made a good feed off our day’s spoil, after which, having selected a few of the finer fish for our breakfast, we presented the remainder to the group of villagers who had assembled at a respectful distance to watch our doings. They seemed wonderfully grateful, and heaped mountains of what I presumed to be Irish blessings upon our heads.

Thus we spent several days, having splendid sport, and first-rate appetites to do justice upon our prey. We were pleased to find how friendly the villagers were inclined to be, and that there was no evidence of their having ventured to meddle with our belongings during our absences.

It was on a Tuesday that we arrived in Kraighten, and it would be on the Sunday following that we made a great discovery. Hitherto we had always gone up-stream; on that day, however, we laid aside our rods, and, taking some provisions, set off for a long ramble in the opposite direction. The day was warm, and we trudged along leisurely enough, stopping about mid-day to eat our lunch upon a great flat rock near the riverbank. Afterward we sat and smoked awhile, resuming our walk only when we were tired of inaction.

For perhaps another hour we wandered onward, chatting quietly and comfortably on this and that matter, and on several occasions stopping while my companion---who is something of an artist---made rough sketches of striking bits of the wild scenery.

And then, without any warning whatsoever, the river we had followed so confidently, came to an abrupt end---vanishing into the earth.

“Good Lord!” I said, “who ever would have thought of this?”

And I stared in amazement; then I turned to Tonnison. He was looking, with a blank expression upon his face, at the place where the river disappeared.

In a moment he spoke.

“Let us go on a bit; it may reappear again---anyhow, it is worth investigating.”

I agreed, and we went forward once more, though rather aimlessly; for we were not at all certain in which direction to prosecute our search. For perhaps a mile we moved onward; then Tonnison, who had been gazing about curiously, stopped and shaded his eyes.

“See!” he said, after a moment, “isn’t that mist or something, over there to the right---away in a line with that great piece of rock?” And he indicated with his hand.

I stared, and, after a minute, seemed to see something, but could not be certain, and said so.

“Anyway,” my friend replied, “we’ll just go across and have a glance.” And he started off in the direction he had suggested, I following. Presently, we came among bushes, and, after a time, out upon the top of a high, boulder-strewn bank, from which we looked down into a wilderness of bushes and trees.

“Seems as though we had come upon an oasis in this desert of stone,” muttered Tonnison, as he gazed interestedly. Then he was silent, his eyes fixed; and I looked also; for up from somewhere about the center of the wooded lowland there rose high into the quiet air a great column of hazelike spray, upon which the sun shone, causing innumerable rainbows.

“How beautiful!” I exclaimed.

“Yes,” answered Tonnison, thoughtfully. “There must be a waterfall, or something, over there. Perhaps it’s our river come to light again. Let’s go and see.”

Down the sloping bank we made our way, and entered among the trees and shrubberies. The bushes were matted, and the trees overhung us, so that the place was disagreeably gloomy; though not dark enough to hide from me the fact that many of the trees were fruit trees, and that, here and there, one could trace indistinctly, signs of a long departed cultivation. Thus it came to me that we were making our way through the riot of a great and ancient garden. I said as much to Tonnison, and he agreed that there certainly seemed reasonable grounds for my belief.

What a wild place it was, so dismal and somber! Somehow, as we went forward, a sense of the silent loneliness and desertion of the old garden grew upon me, and I felt shivery. One could imagine things lurking among the tangled bushes; while, in the very air of the place, there seemed something uncanny. I think Tonnison was conscious of this also, though he said nothing.

Suddenly, we came to a halt. Through the trees there had grown upon our ears a distant sound. Tonnison bent forward, listening. I could hear it more plainly now; it was continuous and harsh---a sort of droning roar, seeming to come from far away. I experienced a queer, indescribable, little feeling of nervousness. What sort of place was it into which we had got? I looked at my companion, to see what he thought of the matter; and noted that there was only puzzlement in his face; and then, as I watched his features, an expression of comprehension crept over them, and he nodded his head.

“That’s a waterfall,” he exclaimed, with conviction. “I know the sound now.” And he began to push vigorously through the bushes, in the direction of the noise.

As we went forward, the sound became plainer continually, showing that we were heading straight toward it. Steadily, the roaring grew louder and nearer, until it appeared, as I remarked to Tonnison, almost to come from under our feet---and still we were surrounded by the trees and shrubs.

“Take care!” Tonnison called to me. “Look where you’re going.” And then, suddenly, we came out from among the trees, on to a great open space, where, not six paces in front of us, yawned the mouth of a tremendous chasm, from the depths of which the noise appeared to rise, along with the continuous, mistlike spray that we had witnessed from the top of the distant bank.

For quite a minute we stood in silence, staring in bewilderment at the sight; then my friend went forward cautiously to the edge of the abyss. I followed, and, together, we looked down through a boil of spray at a monster cataract of frothing water that burst, spouting, from the side of the chasm, nearly a hundred feet below.

“Good Lord!” said Tonnison.

I was silent, and rather awed. The sight was so unexpectedly grand and eerie; though this latter quality came more upon me later.

Presently, I looked up and across to the further side of the chasm. There, I saw something towering up among the spray: it looked like a fragment of a great ruin, and I touched Tonnison on the shoulder. He glanced ’round, with a start, and I pointed toward the thing. His gaze followed my finger, and his eyes lighted up with a sudden flash of excitement, as the object came within his field of view.

“Come along,” he shouted above the uproar. “We’ll have a look at it. There’s something queer about this place; I feel it in my bones.” And he started off, ’round the edge of the craterlike abyss. As we neared this new thing, I saw that I had not been mistaken in my first impression. It was undoubtedly a portion of some ruined building; yet now I made out that it was not built upon the edge of the chasm itself, as I had at first supposed; but perched almost at the extreme end of a huge spur of rock that jutted out some fifty or sixty feet over the abyss. In fact, the jagged mass of ruin was literally suspended in midair.

Arriving opposite it, we walked out on to the projecting arm of rock, and I must confess to having felt an intolerable sense of terror as I looked down from that dizzy perch into the unknown depths below us---into the deeps from which there rose ever the thunder of the falling water and the shroud of rising spray.

Reaching the ruin, we clambered ’round it cautiously, and, on the further side, came upon a mass of fallen stones and rubble. The ruin itself seemed to me, as I proceeded now to examine it minutely, to be a portion of the outer wall of some prodigious structure, it was so thick and substantially built; yet what it was doing in such a position I could by no means conjecture. Where was the rest of the house, or castle, or whatever there had been?

I went back to the outer side of the wall, and thence to the edge of the chasm, leaving Tonnison rooting systematically among the heap of stones and rubbish on the outer side. Then I commenced to examine the surface of the ground, near the edge of the abyss, to see whether there were not left other remnants of the building to which the fragment of ruin evidently belonged. But though I scrutinized the earth with the greatest care, I could see no signs of anything to show that there had ever been a building erected on the spot, and I grew more puzzled than ever.

Then, I heard a cry from Tonnison; he was shouting my name, excitedly, and without delay I hurried along the rocky promontory to the ruin. I wondered whether he had hurt himself, and then the thought came, that perhaps he had found something.

I reached the crumbled wall and climbed ’round. There I found Tonnison standing within a small excavation that he had made among the \textit{débris}: he was brushing the dirt from something that looked like a book, much crumpled and dilapidated; and opening his mouth, every second or two, to bellow my name. As soon as he saw that I had come, he handed his prize to me, telling me to put it into my satchel so as to protect it from the damp, while he continued his explorations. This I did, first, however, running the pages through my fingers, and noting that they were closely filled with neat, old-fashioned writing which was quite legible, save in one portion, where many of the pages were almost destroyed, being muddied and crumpled, as though the book had been doubled back at that part. This, I found out from Tonnison, was actually as he had discovered it, and the damage was due, probably, to the fall of masonry upon the opened part. Curiously enough, the book was fairly dry, which I attributed to its having been so securely buried among the ruins.

Having put the volume away safely, I turned-to and gave Tonnison a hand with his self-imposed task of excavating; yet, though we put in over an hour’s hard work, turning over the whole of the upheaped stones and rubbish, we came upon nothing more than some fragments of broken wood, that might have been parts of a desk or table; and so we gave up searching, and went back along the rock, once more to the safety of the land.

The next thing we did was to make a complete tour of the tremendous chasm, which we were able to observe was in the form of an almost perfect circle, save for where the ruin-crowned spur of rock jutted out, spoiling its symmetry.

The abyss was, as Tonnison put it, like nothing so much as a gigantic well or pit going sheer down into the bowels of the earth.

For some time longer, we continued to stare about us, and then, noticing that there was a clear space away to the north of the chasm, we bent our steps in that direction.

Here, distant from the mouth of the mighty pit by some hundreds of yards, we came upon a great lake of silent water---silent, that is, save in one place where there was a continuous bubbling and gurgling.

Now, being away from the noise of the spouting cataract, we were able to hear one another speak, without having to shout at the tops of our voices, and I asked Tonnison what he thought of the place---I told him that I didn’t like it, and that the sooner we were out of it the better I should be pleased.

He nodded in reply, and glanced at the woods behind furtively. I asked him if he had seen or heard anything. He made no answer; but stood silent, as though listening, and I kept quiet also.

Suddenly, he spoke.

“Hark!” he said, sharply. I looked at him, and then away among the trees and bushes, holding my breath involuntarily. A minute came and went in strained silence; yet I could hear nothing, and I turned to Tonnison to say as much; and then, even as I opened my lips to speak, there came a strange wailing noise out of the wood on our left.... It appeared to float through the trees, and there was a rustle of stirring leaves, and then silence.

All at once, Tonnison spoke, and put his hand on my shoulder. “Let us get out of here,” he said, and began to move slowly toward where the surrounding trees and bushes seemed thinnest. As I followed him, it came to me suddenly that the sun was low, and that there was a raw sense of chilliness in the air.

Tonnison said nothing further, but kept on steadily. We were among the trees now, and I glanced around, nervously; but saw nothing, save the quiet branches and trunks and the tangled bushes. Onward we went, and no sound broke the silence, except the occasional snapping of a twig under our feet, as we moved forward. Yet, in spite of the quietness, I had a horrible feeling that we were not alone; and I kept so close to Tonnison that twice I kicked his heels clumsily, though he said nothing. A minute, and then another, and we reached the confines of the wood coming out at last upon the bare rockiness of the countryside. Only then was I able to shake off the haunting dread that had followed me among the trees.

Once, as we moved away, there seemed to come again a distant sound of wailing, and I said to myself that it was the wind---yet the evening was breathless.

Presently, Tonnison began to talk.

“Look you,” he said with decision, “I would not spend the night in \textit{that} place for all the wealth that the world holds. There is something unholy---diabolical---about it. It came to me all in a moment, just after you spoke. It seemed to me that the woods were full of vile things---you know!”

“Yes,” I answered, and looked back toward the place; but it was hidden from us by a rise in the ground.

“There’s the book,” I said, and I put my hand into the satchel.

“You’ve got it safely?” he questioned, with a sudden access of anxiety.

“Yes,” I replied.

“Perhaps,” he continued, “we shall learn something from it when we get back to the tent. We had better hurry, too; we’re a long way off still, and I don’t fancy, now, being caught out here in the dark.”

It was two hours later when we reached the tent; and, without delay, we set to work to prepare a meal; for we had eaten nothing since our lunch at midday.

Supper over, we cleared the things out of the way, and lit our pipes. Then Tonnison asked me to get the manuscript out of my satchel. This I did, and then, as we could not both read from it at the same time, he suggested that I should read the thing out loud. “And mind,” he cautioned, knowing my propensities, “don’t go skipping half the book.”

Yet, had he but known what it contained, he would have realized how needless such advice was, for once at least. And there seated in the opening of our little tent, I began the strange tale of \textit{The House on the Borderland} (for such was the title of the MS.); this is told in the following pages.

\null\null
{\centering\textit{Begin Manuscript...}\par}

\clearpage % the following material begins with \cleartorecto

% File borderland-01.txt
% Version 2017/09/18
% Now begins the manuscript. The first-person narrator is the Recluse.
% In the original, this was Chapter II.

\cleartorecto % this must begin recto
\label{ch:01}

\begin{ChapterStart}
\null\null
\ChapterTitle{1. The Plain of Silence}
\end{ChapterStart}

I am an old man. I live here in this ancient house, surrounded by huge, unkempt gardens.

The peasantry, who inhabit the wilderness beyond, say that I am mad. That is because I will have nothing to do with them. I live here alone---no servants---I hate them. I have one friend, a dog; yes, I would sooner have old Pepper than the rest of Creation together. He, at least, understands me---and has sense enough to leave me alone when I am in my dark moods.

I have decided to start a kind of diary; it may enable me to record some of the thoughts and feelings that I cannot express to anyone; but, beyond this, I am anxious to make some record of the strange things that I have heard and seen, during many years of loneliness, in this weird old building.

For a couple of centuries, this house has had a reputation, a bad one, and, until I bought it, for more than eighty years no one had lived here; consequently, I got the old place at a ridiculously low figure.

I am not superstitious; but I have ceased to deny that things happen in this old house---things that I cannot explain; and, therefore, I must needs ease my mind, by writing down an account of them, to the best of my ability; though, should this, my diary, ever be read when I am gone, the readers will but shake their heads, and be the more convinced that I was mad.

This house, how ancient it is! though its age strikes one less, perhaps, than the quaintness of its structure, which is curious and fantastic to the last degree. Little curved towers and pinnacles, with outlines suggestive of leaping flames, predominate; while the body of the building is in the form of a circle.

I have heard that there is an old story, told amongst the country people, to the effect that the devil built the place. However, that is as may be. True or not, I neither know nor care, save as it may have helped to cheapen it, ere I came.

I must have been here some ten years before I saw sufficient to warrant any belief in the stories, current in the neighborhood, about this house. It is true that I had, on at least a dozen occasions, seen, vaguely, things that puzzled me, and, perhaps, had felt more than I had seen. Then, as the years passed, bringing age upon me, I became often aware of something unseen, yet unmistakably present, in the empty rooms and corridors. Still, it was as I have said many years before I saw any real manifestations of the so-called supernatural.

It was not Halloween. If I were telling a story for amusement’s sake, I should probably place it on that night of nights; but this is a true record of my own experiences, and I would not put pen to paper to amuse anyone. No. It was after midnight on the morning of the twenty-first day of January. I was sitting reading, as is often my custom, in my study. Pepper lay, sleeping, near my chair.

Without warning, the flames of the two candles went low, and then shone with a ghastly green effulgence. I looked up, quickly, and as I did so I saw the lights sink into a dull, ruddy tint; so that the room glowed with a strange, heavy, crimson twilight that gave the shadows behind the chairs and tables a double depth of blackness; and wherever the light struck, it was as though luminous blood had been splashed over the room.

Down on the floor, I heard a faint, frightened whimper, and something pressed itself in between my two feet. It was Pepper, cowering under my dressing gown. Pepper, usually as brave as a lion!

It was this movement of the dog’s, I think, that gave me the first twinge of \textit{real} fear. I had been considerably startled when the lights burnt first green and then red; but had been momentarily under the impression that the change was due to some influx of noxious gas into the room. Now, however, I saw that it was not so; for the candles burned with a steady flame, and showed no signs of going out, as would have been the case had the change been due to fumes in the atmosphere.

I did not move. I felt distinctly frightened; but could think of nothing better to do than wait. For perhaps a minute, I kept my glance about the room, nervously. Then I noticed that the lights had commenced to sink, very slowly; until presently they showed minute specks of red fire, like the gleamings of rubies in the darkness. Still, I sat watching; while a sort of dreamy indifference seemed to steal over me; banishing altogether the fear that had begun to grip me.

Away in the far end of the huge old-fashioned room, I became conscious of a faint glow. Steadily it grew, filling the room with gleams of quivering green light; then they sank quickly, and changed---even as the candle flames had done---into a deep, somber crimson that strengthened, and lit up the room with a flood of awful glory.

The light came from the end wall, and grew ever brighter until its intolerable glare caused my eyes acute pain, and involuntarily I closed them. It may have been a few seconds before I was able to open them. The first thing I noticed was that the light had decreased, greatly; so that it no longer tried my eyes. Then, as it grew still duller, I was aware, all at once, that, instead of looking at the redness, I was staring through it, and through the wall beyond.

Gradually, as I became more accustomed to the idea, I realized that I was looking out on to a vast plain, lit with the same gloomy twilight that pervaded the room. The immensity of this plain scarcely can be conceived. In no part could I perceive its confines. It seemed to broaden and spread out, so that the eye failed to perceive any limitations. Slowly, the details of the nearer portions began to grow clear; then, in a moment almost, the light died away, and the vision---if vision it were---faded and was gone.

Suddenly, I became conscious that I was no longer in the chair. Instead, I seemed to be hovering above it, and looking down at a dim something, huddled and silent. In a little while, a cold blast struck me, and I was outside in the night, floating, like a bubble, up through the darkness. As I moved, an icy coldness seemed to enfold me, so that I shivered.

After a time, I looked to right and left, and saw the intolerable blackness of the night, pierced by remote gleams of fire. Onward, outward, I drove. Once, I glanced behind, and saw the earth, a small crescent of blue light, receding away to my left. Further off, the sun, a splash of white flame, burned vividly against the dark.

An indefinite period passed. Then, for the last time, I saw the earth---an enduring globule of radiant blue, swimming in an eternity of ether. And there I, a fragile flake of soul dust, flickered silently across the void, from the distant blue, into the expanse of the unknown.

A great while seemed to pass over me, and now I could nowhere see anything. I had passed beyond the fixed stars and plunged into the huge blackness that waits beyond. All this time I had experienced little, save a sense of lightness and cold discomfort. Now however the atrocious darkness seemed to creep into my soul, and I became filled with fear and despair. What was going to become of me? Where was I going? Even as the thoughts were formed, there grew against the impalpable blackness that wrapped me a faint tinge of blood. It seemed extraordinarily remote, and mistlike; yet, at once, the feeling of oppression was lightened, and I no longer despaired.

Slowly, the distant redness became plainer and larger; until, as I drew nearer, it spread out into a great, somber glare---dull and tremendous. Still, I fled onward, and, presently, I had come so close, that it seemed to stretch beneath me, like a great ocean of somber red. I could see little, save that it appeared to spread out interminably in all directions.

In a further space, I found that I was descending upon it; and, soon, I sank into a great sea of sullen, red-hued clouds. Slowly, I emerged from these, and there, below me, I saw the stupendous plain that I had seen from my room in this house that stands upon the borders of the Silences.

Presently, I landed, and stood, surrounded by a great waste of loneliness. The place was lit with a gloomy twilight that gave an impression of indescribable desolation.

Afar to my right, within the sky, there burnt a gigantic ring of dull-red fire, from the outer edge of which were projected huge, writhing flames, darted and jagged. The interior of this ring was black, black as the gloom of the outer night. I comprehended, at once, that it was from this extraordinary sun that the place derived its doleful light.

From that strange source of light, I glanced down again to my surroundings. Everywhere I looked, I saw nothing but the same flat weariness of interminable plain. Nowhere could I descry any signs of life; not even the ruins of some ancient habitation.

Gradually, I found that I was being borne forward, floating across the flat waste. For what seemed an eternity, I moved onward. I was unaware of any great sense of impatience; though some curiosity and a vast wonder were with me continually. Always, I saw around me the breadth of that enormous plain; and, always, I searched for some new thing to break its monotony; but there was no change---only loneliness, silence, and desert.

Presently, in a half-conscious manner, I noticed that there was a faint mistiness, ruddy in hue, lying over its surface. Still, when I looked more intently, I was unable to say that it was really mist; for it appeared to blend with the plain, giving it a peculiar unrealness, and conveying to the senses the idea of unsubstantiality.

Gradually, I began to weary with the sameness of the thing. Yet, it was a great time before I perceived any signs of the place, toward which I was being conveyed.

At first, I saw it, far ahead, like a long hillock on the surface of the Plain. Then, as I drew nearer, I perceived that I had been mistaken; for, instead of a low hill, I made out, now, a chain of great mountains, whose distant peaks towered up into the red gloom, until they were almost lost to sight.

\clearpage

% File borderland-02.txt
% Version 2017/09/18
% In the original, this was Chapter III.

\clearpage
\label{ch:02}

\begin{ChapterStart}
\null\null
\ChapterTitle{2. The House in the Arena}
\end{ChapterStart}

And so, after a time, I came to the mountains. Then, the course of my journey was altered, and I began to move along their bases, until, all at once, I saw that I had come opposite to a vast rift, opening into the mountains. Through this, I was borne, moving at no great speed. On either side of me, huge, scarped walls of rocklike substance rose sheer. Far overhead, I discerned a thin ribbon of red, where the mouth of the chasm opened, among inaccessible peaks. Within, was gloom, deep and somber, and chilly silence. For a while, I went onward steadily, and then, at last, I saw, ahead, a deep, red glow, that told me I was near upon the further opening of the gorge.

A minute came and went, and I was at the exit of the chasm, staring out upon an enormous amphitheatre of mountains. Yet, of the mountains, and the terrible grandeur of the place, I recked nothing; for I was confounded with amazement to behold, at a distance of several miles and occupying the center of the arena, a stupendous structure built apparently of green jade. Yet, in itself, it was not the discovery of the building that had so astonished me; but the fact, which became every moment more apparent, that in no particular, save in color and its enormous size, did the lonely structure vary from this house in which I live.

For a while, I continued to stare, fixedly. Even then, I could scarcely believe that I saw aright. In my mind, a question formed, reiterating incessantly: ‘What does it mean?’ ‘What does it mean?’ and I was unable to make answer, even out of the depths of my imagination. I seemed capable only of wonder and fear. For a time longer, I gazed, noting continually some fresh point of resemblance that attracted me. At last, wearied and sorely puzzled, I turned from it, to view the rest of the strange place on to which I had intruded.

Hitherto, I had been so engrossed in my scrutiny of the House, that I had given only a cursory glance ’round. Now, as I looked, I began to realize upon what sort of a place I had come. The arena, for so I have termed it, appeared a perfect circle of about ten to twelve miles in diameter, the House, as I have mentioned before, standing in the center. The surface of the place, like to that of the Plain, had a peculiar, misty appearance, that was yet not mist.

From a rapid survey, my glance passed quickly upward along the slopes of the circling mountains. How silent they were. I think that this same abominable stillness was more trying to me than anything that I had so far seen or imagined. I was looking up, now, at the great crags, towering so loftily. Up there, the impalpable redness gave a blurred appearance to everything.

And then, as I peered, curiously, a new terror came to me; for away up among the dim peaks to my right, I had descried a vast shape of blackness, giantlike. It grew upon my sight. It had an enormous equine head, with gigantic ears, and seemed to peer steadfastly down into the arena. There was that about the pose that gave me the impression of an eternal watchfulness---of having warded that dismal place, through unknown eternities. Slowly, the monster became plainer to me; and then, suddenly, my gaze sprang from it to something further off and higher among the crags. For a long minute, I gazed, fearfully. I was strangely conscious of something not altogether unfamiliar---as though something stirred in the back of my mind. The thing was black, and had four grotesque arms. The features showed indistinctly, ’round the neck, I made out several light-colored objects. Slowly, the details came to me, and I realized, coldly, that they were skulls. Further down the body was another circling belt, showing less dark against the black trunk. Then, even as I puzzled to know what the thing was, a memory slid into my mind, and straightway, I knew that I was looking at a monstrous representation of Kali, the Hindu goddess of death.

Other remembrances of my old student days drifted into my thoughts. My glance fell back upon the huge beast-headed Thing. Simultaneously, I recognized it for the ancient Egyptian god Set, or Seth, the Destroyer of Souls. With the knowledge, there came a great sweep of questioning---‘Two of the---!’ I stopped, and endeavored to think. Things beyond my imagination peered into my frightened mind. I saw, obscurely. ‘The old gods of mythology!’ I tried to comprehend to what it was all pointing. My gaze dwelt, flickeringly, between the two. ‘If---’

An idea came swiftly, and I turned, and glanced rapidly upward, searching the gloomy crags, away to my left. Something loomed out under a great peak, a shape of greyness. I wondered I had not seen it earlier, and then remembered I had not yet viewed that portion. I saw it more plainly now. It was, as I have said, grey. It had a tremendous head; but no eyes. That part of its face was blank.

Now, I saw that there were other things up among the mountains. Further off, reclining on a lofty ledge, I made out a livid mass, irregular and ghoulish. It seemed without form, save for an unclean, half-animal face, that looked out, vilely, from somewhere about its middle. And then I saw others---there were hundreds of them. They seemed to grow out of the shadows. Several I recognized almost immediately as mythological deities; others were strange to me, utterly strange, beyond the power of a human mind to conceive.

On each side, I looked, and saw more, continually. The mountains were full of strange things---Beast-gods, and Horrors so atrocious and bestial that possibility and decency deny any further attempt to describe them. And I---I was filled with a terrible sense of overwhelming horror and fear and repugnance; yet, spite of these, I wondered exceedingly. Was there then, after all, something in the old heathen worship, something more than the mere deifying of men, animals, and elements? The thought gripped me---was there?

What were they, those Beast-gods, and the others? At first, they had appeared to me just sculptured Monsters placed indiscriminately among the inaccessible peaks and precipices of the surrounding mountains. Now, as I scrutinized them with greater intentness, my mind began to reach out to fresh conclusions. There was something about them, an indescribable sort of silent vitality that suggested, to my broadening consciousness, a state of life-in-death---a something that was by no means life, as we understand it; but rather an inhuman form of existence, that well might be likened to a deathless trance---a condition in which it was possible to imagine their continuing, eternally. ‘Immortal!’ the word rose in my thoughts unbidden; and, straightway, I grew to wondering whether this might be the immortality of the gods.

And then, in the midst of my wondering and musing, something happened. Until then, I had been staying just within the shadow of the exit of the great rift. Now, without volition on my part, I drifted out of the semi-darkness and began to move slowly across the arena---toward the House. At this, I gave up all thoughts of those prodigious Shapes above me---and could only stare, frightenedly, at the tremendous structure toward which I was being conveyed so remorselessly. Yet, though I searched earnestly, I could discover nothing that I had not already seen, and so became gradually calmer.

Presently, I had reached a point more than halfway between the House and the gorge. All around was spread the stark loneliness of the place, and the unbroken silence. Steadily, I neared the great building. Then, all at once, something caught my vision, something that came ’round one of the huge buttresses of the House, and so into full view. It was a gigantic thing, and moved with a curious lope, going almost upright, after the manner of a man. It was quite unclothed, and had a remarkable luminous appearance. Yet it was the face that attracted and frightened me the most. It was the face of a swine.

Silently, intently, I watched this horrible creature, and forgot my fear, momentarily, in my interest in its movements. It was making its way, cumbrously ’round the building, stopping as it came to each window to peer in and shake at the bars, with which---as in this house---they were protected; and whenever it came to a door, it would push at it, fingering the fastening stealthily. Evidently, it was searching for an ingress into the House.

I had come now to within less than a quarter of a mile of the great structure, and still I was compelled forward. Abruptly, the Thing turned and gazed hideously in my direction. It opened its mouth, and, for the first time, the stillness of that abominable place was broken, by a deep, booming note that sent an added thrill of apprehension through me. Then, immediately, I became aware that it was coming toward me, swiftly and silently. In an instant, it had covered half the distance that lay between. And still, I was borne helplessly to meet it. Only a hundred yards, and the brutish ferocity of the giant face numbed me with a feeling of unmitigated horror. I could have screamed, in the supremeness of my fear; and then, in the very moment of my extremity and despair, I became conscious that I was looking down upon the arena, from a rapidly increasing height. I was rising, rising. In an inconceivably short while, I had reached an altitude of many hundred feet. Beneath me, the spot that I had just left, was occupied by the foul swine-creature. It had gone down on all fours and was snuffing and rooting, like a veritable hog, at the surface of the arena. A moment and it rose to its feet, clutching upward, with an expression of desire upon its face such as I have never seen in this world.

Continually, I mounted higher. A few minutes, it seemed, and I had risen above the great mountains---floating, alone, afar in the redness. At a tremendous distance below, the arena showed, dimly; with the mighty House looking no larger than a tiny spot of green. The swine-thing was no longer visible.

Presently, I passed over the mountains, out above the huge breadth of the plain. Far away, on its surface, in the direction of the ring-shaped sun, there showed a confused blur. I looked toward it, indifferently. It reminded me, somewhat, of the first glimpse I had caught of the mountain-amphitheatre.

With a sense of weariness, I glanced upward at the immense ring of fire. What a strange thing it was! Then, as I stared, out from the dark center, there spurted a sudden flare of extraordinary vivid fire. Compared with the size of the black center, it was as naught; yet, in itself, stupendous. With awakened interest, I watched it carefully, noting its strange boiling and glowing. Then, in a moment, the whole thing grew dim and unreal, and so passed out of sight. Much amazed, I glanced down to the Plain from which I was still rising. Thus, I received a fresh surprise. The Plain---everything had vanished, and only a sea of red mist was spread far below me. Gradually as I stared this grew remote, and died away into a dim far mystery of red against an unfathomable night. A while, and even this had gone, and I was wrapped in an impalpable, lightless gloom.

\clearpage

% File borderland-03.txt
% Version 2017/09/18
% In the original, this was Chapter IV.

\clearpage
\label{ch:03}

\begin{ChapterStart}
\null\null
\ChapterTitle{3. The Earth}
\end{ChapterStart}

Thus I was, and only the memory that I had lived through the dark, once before, served to sustain my thoughts. A great time passed---ages. And then a single star broke its way through the darkness. It was the first of one of the outlying clusters of this universe. Presently, it was far behind, and all about me shone the splendor of the countless stars. Later, years it seemed, I saw the sun, a clot of flame. Around it, I made out presently several remote specks of light---the planets of the Solar system. And so I saw the earth again, blue and unbelievably minute. It grew larger, and became defined.

A long space of time came and went, and then at last I entered into the shadow of the world---plunging headlong into the dim and holy earth night. Overhead were the old constellations, and there was a crescent moon. Then, as I neared the earth’s surface, a dimness swept over me, and I appeared to sink into a black mist.

For a while, I knew nothing. I was unconscious. Gradually, I became aware of a faint, distant whining. It became plainer. A desperate feeling of agony possessed me. I struggled madly for breath, and tried to shout. A moment, and I got my breath more easily. I was conscious that something was licking my hand. Something damp swept across my face. I heard a panting, and then again the whining. It seemed to come to my ears, now, with a sense of familiarity, and I opened my eyes. All was dark; but the feeling of oppression had left me. I was seated, and something was whining piteously, and licking me. I felt strangely confused, and, instinctively, tried to ward off the thing that licked. My head was curiously vacant, and, for the moment, I seemed incapable of action or thought. Then, things came back to me, and I called ‘Pepper,’ faintly. I was answered by a joyful bark, and renewed and frantic caresses.

In a little while, I felt stronger, and put out my hand for the matches. I groped about, for a few moments, blindly; then my hands lit upon them, and I struck a light, and looked confusedly around. All about me, I saw the old, familiar things. And there I sat, full of dazed wonders, until the flame of the match burnt my finger, and I dropped it; while a hasty expression of pain and anger, escaped my lips, surprising me with the sound of my own voice.

After a moment, I struck another match, and, stumbling across the room, lit the candles. As I did so, I observed that they had not burned away, but had been put out.

As the flames shot up, I turned, and stared about the study; yet there was nothing unusual to see; and, suddenly, a gust of irritation took me. What had happened? I held my head, with both hands, and tried to remember. Ah! the great, silent Plain, and the ring-shaped sun of red fire. Where were they? Where had I seen them? How long ago? I felt dazed and muddled. Once or twice, I walked up and down the room, unsteadily. My memory seemed dulled, and, already, the thing I had witnessed came back to me with an effort.

I have a remembrance of cursing, peevishly, in my bewilderment. Suddenly, I turned faint and giddy, and had to grasp at the table for support. During a few moments, I held on, weakly; and then managed to totter sideways into a chair. After a little time, I felt somewhat better, and succeeded in reaching the cupboard where, usually, I keep brandy and biscuits. I poured myself out a little of the stimulant, and drank it off. Then, taking a handful of biscuits, I returned to my chair, and began to devour them, ravenously. I was vaguely surprised at my hunger. I felt as though I had eaten nothing for an uncountably long while.

As I ate, my glance roved about the room, taking in its various details, and still searching, though almost unconsciously, for something tangible upon which to take hold, among the invisible mysteries that encompassed me. ‘Surely,’ I thought, ‘there must be something---’ And, in the same instant, my gaze dwelt upon the face of the clock in the opposite corner. Therewith, I stopped eating, and just stared. For, though its ticking indicated most certainly that it was still going, the hands were pointing to a little \textit{before} the hour of midnight; whereas it was, as well I knew, considerably \textit{after} that time when I had witnessed the first of the strange happenings I have just described.

For perhaps a moment I was astounded and puzzled. Had the hour been the same as when I had last seen the clock, I should have concluded that the hands had stuck in one place, while the internal mechanism went on as usual; but that would, in no way, account for the hands having traveled backward. Then, even as I turned the matter over in my wearied brain, the thought flashed upon me that it was now close upon the morning of the twenty-second, and that I had been unconscious to the visible world through the greater portion of the last twenty-four hours. The thought occupied my attention for a full minute; then I commenced to eat again. I was still very hungry.

And so the days pass on, and I am still filled with a wonder to know the meaning of all that I saw on that memorable night. Yet, well I know that my curiosity is little likely to be satisfied.

\clearpage

% File borderland-04.txt
% Version 2017/09/18
% In the original, this was Chapter V.

\clearpage
\label{ch:04}

\begin{ChapterStart}
\null\null
\ChapterTitle{4. The Thing in the Ravine}
\end{ChapterStart}

This house is, as I have said before, surrounded by a huge estate, and wild and uncultivated gardens.

Away at the back, distant some three hundred yards, is a dark, deep ravine. At the bottom runs a sluggish stream so overhung by trees as scarcely to be seen from above.

In passing, I must explain that this river has a subterranean origin, emerging suddenly at the East end of the ravine, and disappearing, as abruptly, beneath the cliffs that form its Western extremity.

It was some months after my vision (if vision it were) of the great Plain that my attention was particularly attracted to the ravine.

I happened, one day, to be walking along its Southern edge, when, suddenly, several pieces of rock and shale were dislodged from the face of the cliff immediately beneath me, and fell with a sullen crash through the trees. I heard them splash in the river at the bottom; and then silence. I should not have given this incident more than a passing thought, had not Pepper at once begun to bark savagely; nor would he be silent when I bade him, which is most unusual behavior on his part.

Feeling that there must be someone or something in the Pit, I went back to the house, quickly, for a stick. When I returned, Pepper had ceased his barks and was growling and smelling, uneasily, along the top.

Whistling to him to follow me, I started to descend cautiously. The depth to the bottom of the ravine must be about a hundred and fifty feet, and some time as well as considerable care was expended before we reached the bottom in safety.

Once down, Pepper and I started to explore along the banks of the river. It was very dark there due to the overhanging trees, and I moved warily, keeping my glance about me and my stick ready.

Pepper was quiet now and kept close to me all the time. Thus, we searched right up one side of the river, without hearing or seeing anything. Then, we crossed over---by the simple method of jumping---and commenced to beat our way back through the underbrush.

We had accomplished perhaps half the distance, when I heard again the sound of falling stones on the other side---the side from which we had just come. One large rock came thundering down through the treetops, struck the opposite bank, and bounded into the river, driving a great jet of water right over us. At this, Pepper gave out a deep growl; then stopped, and pricked up his ears. I listened, also.

A second later, a loud, half-human, half-piglike squeal sounded from among the trees, apparently about halfway up the South cliff. It was answered by a similar note from the bottom of the ravine. At this, Pepper gave a short, sharp bark, and, springing across the little river, disappeared into the bushes.

Immediately afterward, I heard his barks increase in depth and number, and in between there sounded a noise of confused jabbering. This ceased, and, in the succeeding silence, there rose a semi-human yell of agony. Almost immediately, Pepper gave a long-drawn howl of pain, and then the shrubs were violently agitated, and he came running out with his tail down, and glancing as he ran over his shoulder. As he reached me, I saw that he was bleeding from what appeared to be a great claw wound in the side that had almost laid bare his ribs.

Seeing Pepper thus mutilated, a furious feeling of anger seized me, and, whirling my staff, I sprang across, and into the bushes from which Pepper had emerged. As I forced my way through, I thought I heard a sound of breathing. Next instant, I had burst into a little clear space, just in time to see something, livid white in color, disappear among the bushes on the opposite side. With a shout, I ran toward it; but, though I struck and probed among the bushes with my stick, I neither saw nor heard anything further; and so returned to Pepper. There, after bathing his wound in the river, I bound my wetted handkerchief ’round his body; having done which, we retreated up the ravine and into the daylight again.

The thing I had seen run into the bushes had, so far as I had observed, a skin like a hog’s, only of a dead, unhealthy white color. And then---it had run upright, or nearly so, upon its hind feet, with a motion somewhat resembling that of a human being. This much I had noticed in my brief glimpse, and, truth to tell, I felt a good deal of uneasiness, besides curiosity as I turned the matter over in my mind.

It was in the morning that the above incident had occurred.

Then, it would be after dinner, as I sat reading, that, happening to look up suddenly, I saw something peering in over the window ledge the eyes and ears alone showing.

‘A pig, by Jove!’ I said, and rose to my feet. Thus, I saw the thing more completely; but it was no pig---God alone knows what it was. It reminded me, vaguely, of the hideous Thing that had haunted the great arena. It had a grotesquely human mouth and jaw; but with no chin of which to speak. The nose was prolonged into a snout; thus it was that with the little eyes and queer ears, gave it such an extraordinarily swinelike appearance. Of forehead there was little, and the whole face was of an unwholesome white color.

For perhaps a minute, I stood looking at the thing with an ever growing feeling of disgust, and some fear. The mouth kept jabbering, inanely, and once emitted a half-swinish grunt. I think it was the eyes that attracted me the most; they seemed to glow, at times, with a horribly human intelligence, and kept flickering away from my face, over the details of the room, as though my stare disturbed it.

It appeared to be supporting itself by two clawlike hands upon the windowsill. These claws, unlike the face, were of a clayey brown hue, and bore an indistinct resemblance to human hands, in that they had four fingers and a thumb; though these were webbed up to the first joint, much as are a duck’s. Nails it had also, but so long and powerful that they were more like the talons of an eagle than aught else.

As I have said, before, I felt some fear; though almost of an impersonal kind. I may explain my feeling better by saying that it was more a sensation of abhorrence; such as one might expect to feel, if brought in contact with something superhumanly foul; something unholy---belonging to some hitherto undreamt of state of existence.

I cannot say that I grasped these various details of the brute at the time. I think they seemed to come back to me, afterward, as though imprinted upon my brain. I imagined more than I saw as I looked at the thing, and the material details grew upon me later.

For perhaps a minute I stared at the creature; then as my nerves steadied a little I shook off the vague alarm that held me, and took a step toward the window. Even as I did so, the thing ducked and vanished. I rushed to the door and looked ’round hurriedly; but only the tangled bushes and shrubs met my gaze.

I ran back into the house, and, getting my gun, sallied out to search through the gardens. As I went, I asked myself whether the thing I had just seen was likely to be the same of which I had caught a glimpse in the morning. I inclined to think it was.

I would have taken Pepper with me; but judged it better to give his wound a chance to heal. Besides, if the creature I had just seen was, as I imagined, his antagonist of the morning, it was not likely that he would be of much use.

I began my search, systematically. I was determined, if it were possible, to find and put an end to that swine-thing. This was, at least, a material Horror!

At first, I searched, cautiously; with the thought of Pepper’s wound in my mind; but, as the hours passed, and not a sign of anything living, showed in the great, lonely gardens, I became less apprehensive. I felt almost as though I would welcome the sight of it. Anything seemed better than this silence, with the ever-present feeling that the creature might be lurking in every bush I passed. Later, I grew careless of danger, to the extent of plunging right through the bushes, probing with my gun barrel as I went.

For the rest of the afternoon, I prosecuted the search anxiously. I felt that I should be unable to sleep, with that bestial thing haunting the shrubberies, and yet, when evening fell, I had seen nothing. Then, as I turned homeward, I heard a short, unintelligible noise, among the bushes to my right. Instantly, I turned, and, aiming quickly, fired in the direction of the sound. Immediately afterward, I heard something scuttling away among the bushes. It moved rapidly, and in a minute had gone out of hearing. After a few steps I ceased my pursuit, realizing how futile it must be in the fast gathering gloom; and so, with a curious feeling of depression, I entered the house.

That night, I went ’round to all the windows and doors on the ground floor; and saw to it that they were securely fastened. This precaution was scarcely necessary as regards the windows, as all of those on the lower storey are strongly barred; but with the doors---of which there are five---it was wisely thought, as not one was locked.

Having secured these, I went to my study, yet, somehow, for once, the place jarred upon me; it seemed so huge and echoey. For some time I tried to read; but at last finding it impossible I carried my book down to the kitchen where a large fire was burning, and sat there.

I dare say, I had read for a couple of hours, when, suddenly, I heard a sound that made me lower my book, and listen, intently. It was a noise of something rubbing and fumbling against the back door. Once the door creaked, loudly; as though force were being applied to it. During those few, short moments, I experienced an indescribable feeling of terror, such as I should have believed impossible. My hands shook; a cold sweat broke out on me, and I shivered violently.

Gradually, I calmed. The stealthy movements outside had ceased.

Then for an hour I sat silent and watchful. All at once the feeling of fear took me again. I felt as I imagine an animal must, under the eye of a snake. Yet now I could hear nothing. Still, there was no doubting that some unexplained influence was at work.

Gradually, imperceptibly almost, something stole on my ear---a sound that resolved itself into a faint murmur. Quickly it developed and grew into a muffled but hideous chorus of bestial shrieks. It appeared to rise from the bowels of the earth.

I heard a thud, and realized in a dull, half comprehending way that I had dropped my book. After that, I just sat; and thus the daylight found me, when it crept wanly in through the barred, high windows of the great kitchen.

With the dawning light, the feeling of stupor and fear left me; and I came more into possession of my senses.

Thereupon I picked up my book, and crept to the door to listen. Not a sound broke the chilly silence. For some minutes I stood there; then, very gradually and cautiously, I drew back the bolt and opening the door peeped out.

My caution was unneeded. Nothing was to be seen, save the grey vista of dreary, tangled bushes and trees, extending to the distant plantation.

With a shiver, I closed the door, and made my way, quietly, up to bed.

\clearpage


% File borderland-05.txt
% Version 2017/09/18
% In the original, this was Chapter IX.
% Intervening adventure with swine-creatures omitted.

\clearpage
\label{ch:05}

\begin{ChapterStart}
\null\null
\ChapterTitle{5. The Cellars}
\end{ChapterStart}


At last, what with being tired and cold, and the uneasiness that possessed me, I resolved to take a walk through the house; first calling in at the study, for a glass of brandy to warm me. This, I did, and, while there, I examined the door, carefully; but found all as I had left it the night before.

The day was just breaking, as I left the tower; though it was still too dark in the house to be able to see without a light, and I took one of the study candles with me on my ’round. By the time I had finished the ground floor, the daylight was creeping in, wanly, through the barred windows. My search had shown me nothing fresh. Everything appeared to be in order, and I was on the point of extinguishing my candle, when the thought suggested itself to me to have another glance ’round the cellars.

For, perhaps, the half of a minute, I hesitated. I would have been very willing to forego the task---as, indeed, I am inclined to think any man well might---for of all the great, awe-inspiring rooms in this house, the cellars are the hugest and weirdest. Great, gloomy caverns of places, unlit by any ray of daylight. Yet, I would not shirk the work. I felt that to do so would smack of sheer cowardice. Besides, as I reassured myself, the cellars were really the most unlikely places in which to come across anything dangerous; considering that they can be entered, only through a heavy oaken door, the key of which, I carry always on my person.

It is in the smallest of these places that I keep my wine; a gloomy hole close to the foot of the cellar stairs; and beyond which, I have seldom proceeded. Indeed, save for the rummage ’round, already mentioned, I doubt whether I had ever, before, been right through the cellars.

As I unlocked the great door, at the top of the steps, I paused, nervously, a moment, at the strange, desolate smell that assailed my nostrils. Then, throwing the barrel of my weapon forward, I descended, slowly, into the darkness of the underground regions.

Reaching the bottom of the stairs, I stood for a minute, and listened. All was silent, save for a faint drip, drip of water, falling, drop-by-drop, somewhere to my left. As I stood, I noticed how quietly the candle burnt; never a flicker nor flare, so utterly windless was the place.

Quietly, I moved from cellar to cellar. I had but a very dim memory of their arrangement. The impressions left by my first search were blurred. I had recollections of a succession of great cellars, and of one, greater than the rest, the roof of which was upheld by pillars; beyond that my mind was hazy, and predominated by a sense of cold and darkness and shadows. Now, however, it was different; for, although nervous, I was sufficiently collected to be able to look about me, and note the structure and size of the different vaults I entered.

Of course, with the amount of light given by my candle, it was not possible to examine each place, minutely, but I was enabled to notice, as I went along, that the walls appeared to be built with wonderful precision and finish; while here and there, an occasional, massive pillar shot up to support the vaulted roof.

Thus, I came, at last, to the great cellar that I remembered. It is reached, through a huge, arched entrance, on which I observed strange, fantastic carvings, which threw queer shadows under the light of my candle. As I stood, and examined these, thoughtfully, it occurred to me how strange it was, that I should be so little acquainted with my own house. Yet, this may be easily understood, when one realizes the size of this ancient pile.

Holding the light high, I passed on into the cellar, and, keeping to the right, paced slowly up, until I reached the further end.\linebreak I walked quietly, and looked cautiously about, as I went. But, so far as the light showed, I saw nothing unusual.

At the top, I turned to the left, still keeping to the wall, and so continued, until I had traversed the whole of the vast chamber. As I moved along, I noticed that the floor was composed of solid rock, in places covered with a damp mould, in others bare, or almost so, save for a thin coating of light-grey dust.

I had halted at the doorway. Now, however, I turned, and made my way up the center of the place; passing among the pillars, and glancing to right and left, as I moved. About halfway up the cellar, I stubbed my foot against something that gave out a metallic sound. Stooping quickly, I held the candle, and saw that the object I had kicked, was a large, metal ring. Bending lower, I cleared the dust from around it, and, presently, discovered that it was attached to a ponderous trap door, black with age.

Feeling excited, and wondering to where it could lead, I laid my gun on the floor, and, sticking the candle in the trigger guard, took the ring in both hands, and pulled. The trap creaked loudly---the sound echoing, vaguely, through the huge place---and opened, heavily.

Propping the edge on my knee, I reached for the candle, and held it in the opening, moving it to right and left; but could see nothing. I was puzzled and surprised. There were no signs of steps, nor even the appearance of there ever having been any. Nothing; save an empty blackness. I might have been looking down into a bottomless, sideless well. Then, even as I stared, full of perplexity, I seemed to hear, far down, as though from untold depths, a faint whisper of sound. I bent my head, quickly, more into the opening, and listened, intently. It may have been fancy; but I could have sworn to hearing a soft titter, that grew into a hideous, chuckling, faint and distant. Startled, I leapt backward, letting the trap fall, with a hollow clang, that filled the place with echoes. Even then, I seemed to hear that mocking, suggestive laughter; but this, I knew, must be my imagination. The sound, I had heard, was far too slight to penetrate through the cumbrous trap.

For a full minute, I stood there, quivering---glancing, nervously, behind and before; but the great cellar was silent as a grave, and, gradually, I shook off the frightened sensation. With a calmer mind, I became again curious to know into what that trap opened; but could not, then, summon sufficient courage to make a further investigation. One thing I felt, however, was that the trap ought to be secured. This, I accomplished by placing upon it several large pieces of ‘dressed’ stone, which I had noticed in my tour along the East wall.

Then, after a final scrutiny of the rest of the place, I retraced my way through the cellars, to the stairs, and so reached the daylight, with an infinite feeling of relief, that the uncomfortable task was accomplished.

\clearpage

% File borderland-06.txt
% Version 2017/09/18
% In the original, this was Chapter XII.
% Heavily edited here.

\clearpage
\label{ch:06}

\begin{ChapterStart}
\null\null
\ChapterTitle{6. The Subterranean Pit}
\end{ChapterStart}

Another week came and went, during which I spent a great deal of my time about the ravine. I had come to the conclusion a few days earlier, that the arched hole, in the angle of the great rift, was the place through which the Swine-things had made their exit, from some unholy place in the bowels of the world. How near the probable truth this went, I was to learn later.

It may be easily understood, that I was tremendously curious, though in a frightened way, to know to what infernal place that hole led; though, so far, the idea had not struck me, seriously, of making an investigation. I was far too much imbued with a sense of horror of the Swine-creatures, to think of venturing, willingly, where there was any chance of coming into contact with them.

Gradually, however, as time passed, this feeling grew insensibly less; so that when, a few days later, the thought occurred to me that it might be possible to clamber down and have a look into the hole, I was not so exceedingly averse to it, as might have been imagined. Still, I do not think, even then, that I really intended to try any such foolhardy adventure. And yet, such is the pertinacity of human curiosity, that, at last, my chief desire was but to discover what lay beyond that gloomy entrance.

Slowly, as the days slid by, my fear of the Swine-things became an emotion of the past---more an unpleasant, incredible memory, than aught else.

Thus, a day came, when, throwing thoughts and fancies adrift, I procured a rope from the house, and, having made it fast to a stout tree, at the top of the rift, and some little distance back from the ravine edge, let the other end down into the cleft, until it dangled right across the mouth of the dark hole.

Then, cautiously, and with many misgivings as to whether it was not a mad act that I was attempting, I climbed slowly down, using the rope as a support, until I reached the hole. Here, still holding on to the rope, I stood, and peered in. All was perfectly dark, and not a sound came to me. Yet, a moment later, it seemed that I could hear something. I held my breath, and listened; but all was silent as the grave, and I breathed freely once more. At the same instant, I heard the sound again. It was like a noise of labored breathing---deep and sharp-drawn. For a short second, I stood, petrified; not able to move. But now the sounds had ceased again, and I could hear nothing.

As I stood there, anxiously, my foot dislodged a pebble, which fell inward, into the dark, with a hollow chink. At once, the noise was taken up and repeated a score of times; each succeeding echo being fainter, and seeming to travel away from me, as though into remote distance. Then, as the silence fell again, I heard that stealthy breathing. For each respiration I made, I could hear an answering breath. The sounds appeared to be coming nearer; and then, I heard several others; but fainter and more distant. Why I did not grip the rope, and spring up out of danger, I cannot say. It was as though I had been paralyzed. I broke out into a profuse sweat, and tried to moisten my lips with my tongue. My throat had gone suddenly dry, and I coughed, huskily. It came back to me, in a dozen, horrible, throaty tones, mockingly. I peered, helplessly, into the gloom; but still nothing showed. I had a strange, choky sensation, and again I coughed, dryly. Again the echo took it up, rising and falling, grotesquely, and dying slowly into a muffled silence.

Then, suddenly, a thought came to me, and I held my breath. The other breathing stopped. I breathed again, and, once more, it re-commenced. But now, I no longer feared. I knew that the strange sounds were not made by any lurking Swine-creature; but were simply the echo of my own respirations.

Yet, I had received such a fright, that I was glad to scramble up the rift, and haul up the rope. I was far too shaken and nervous to think of entering that dark hole then, and so returned to the house. I felt more myself next morning; but even then, I could not summon up sufficient courage to explore the place.

All this time, the water in the ravine had been creeping slowly up, and now stood but a little below the opening. At the rate at which it was rising, it would be level with the floor in less than another week; and I realized that, unless I carried out my investigations soon, I should probably never do so at all; as the water would rise and rise, until the opening, itself, was submerged.

It may have been that this thought stirred me to act; but, whatever it was, a couple of days later, saw me standing at the top of the cleft, fully equipped for the task.

This time, I was resolved to conquer my shirking, and go right through with the matter. With this intention, I had brought, in addition to the rope, a bundle of candles, meaning to use them as a torch; also my double-barreled shotgun. In my belt, I had a heavy horse-pistol, loaded with buckshot.

As before, I fastened the rope to the tree. Then, having tied my gun across my shoulders, with a piece of stout cord, I lowered myself over the edge of the ravine. At this movement, Pepper, who had been eyeing my actions, watchfully, rose to his feet, and ran to me, with a half bark, half wail, it seemed to me, of warning. But I was resolved on my enterprise, and bade him lie down. I would much have liked to take him with me; but this was next to impossible, in the existing circumstances. As my face dropped level with the ravine edge, he licked me, right across the mouth; and then, seizing my sleeve between his teeth, began to pull back, strongly. It was very evident that he did not want me to go. Yet, having made up my mind, I had no intention of giving up the attempt; and, with a sharp word to Pepper, to release me, I continued my descent, leaving the poor old fellow at the top, barking and crying like a forsaken pup.

Carefully, I lowered myself from projection to projection. I knew that a slip might mean a wetting.

Reaching the entrance, I let go the rope, and untied the gun from my shoulders. Then, with a last look at the sky---which I noticed was clouding over, rapidly---I went forward a couple of paces, so as to be shielded from the wind, and lit one of the candles. Holding it above my head, and grasping my gun, firmly, I began to move on, slowly, throwing my glances in all directions.

For the first minute, I could hear the melancholy sound of Pepper’s howling, coming down to me. Gradually, as I penetrated further into the darkness, it grew fainter; until, in a little while, I could hear nothing. The path tended downward somewhat, and to the left. Thence it kept on, still running to the left, until I found that it was leading me right in the direction of the house.

Very cautiously, I moved onward, stopping, every few steps, to listen. I had gone, perhaps, a hundred yards, when, suddenly, it seemed to me that I caught a faint sound, somewhere along the passage behind. With my heart thudding heavily, I listened. The noise grew plainer, and appeared to be approaching, rapidly. I could hear it distinctly, now. It was the soft padding of running feet. In the first moments of fright, I stood, irresolute; not knowing whether to go forward or backward. Then, with a sudden realization of the best thing to do, I backed up to the rocky wall on my right, and, holding the candle above my head, waited---gun in hand---cursing my foolhardy curiosity, for bringing me into such a strait.

I had not long to wait, but a few seconds, before two eyes reflected back from the gloom, the rays of my candle. I raised my gun, using my right hand only, and aimed quickly. Even as I did so, something leapt out of the darkness, with a blustering bark of joy that woke the echoes, like thunder. It was Pepper. How he had contrived to scramble down the cleft, I could not conceive. As I brushed my hand, nervously, over his coat, I noticed that he was dripping; and concluded that he must have tried to follow me, and fallen into the water; from which he would not find it very difficult to climb.

Having waited a minute, or so, to steady myself, I proceeded along the way, Pepper following, quietly. I was curiously glad to have the old fellow with me. He was company, and, somehow, with him at my heels, I was less afraid. Also, I knew how quickly his keen ears would detect the presence of any unwelcome creature, should there be such, amid the darkness that wrapped us.

For some minutes we went slowly along; the path still leading straight toward the house. Soon, I concluded, we should be standing right beneath it, did the path but carry far enough. I led the way, cautiously, for another fifty yards, or so. Then, I stopped, and held the light high; and reason enough I had to be thankful that I did so; for there, not three paces forward, the path vanished, and, in place, showed a hollow blackness, that sent sudden fear through me.

Very cautiously, I crept forward, and peered down; but could see nothing. Then, I crossed to the left of the passage, to see whether there might be any continuation of the path. Here, right against the wall, I found that a narrow track, some three feet wide, led onward. Carefully, I stepped on to it; but had not gone far, before I regretted venturing thereon. For, after a few paces, the already narrow way, resolved itself into a mere ledge, with, on the one side the solid, unyielding rock, towering up, in a great wall, to the unseen roof, and, on the other, that yawning chasm.

To my great relief, a little further on, the track suddenly broadened out again to its original breadth. Gradually, as I went onward, I noticed that the path trended steadily to the right, and so, after some minutes, I discovered that I was not going forward; but simply circling the huge abyss. I had, evidently, come to the end of the great passage.

Five minutes later, I stood on the spot from which I had started; having been completely ‘round, what I guessed now to be a vast Pit, the mouth of which must be at least a hundred yards across.

For some little time, I stood there, lost in perplexing thought. ‘What does it all mean?’ was the cry that had begun to reiterate through my brain.

A sudden idea struck me, and I searched ‘round for a piece of stone. Presently, I found a bit of rock, about the size of a small loaf. Sticking the candle upright in a crevice of the floor, I went back from the edge, somewhat, and, taking a short run, launched the stone forward into the chasm---my idea being to throw it far enough to keep it clear of the sides. Then, I stooped forward, and listened; but, though I kept perfectly quiet, for at least a full minute, no sound came back to me from out of the dark.

I knew, then, that the depth of the hole must be immense; for the stone, had it struck anything, was large enough to have set the echoes of that weird place, whispering for an indefinite period. Even as it was, the cavern had given back the sounds of my footfalls, multitudinously. The place was awesome, and I would willingly have retraced my steps, and left the mysteries of its solitudes unsolved; only, to do so, meant admitting defeat.

Then, a thought came, to try to get a view of the abyss. It occurred to me that, if I placed my candles ‘round the edge of the hole, I should be able to get, at least, some dim sight of the place.

I found, on counting, that I had brought fifteen candles, in the bundle---my first intention having been, as I have already said, to make a torch of the lot. These, I proceeded to place ‘round the Pit mouth, with an interval of about twenty yards between each.

Having completed the circle, I stood in the passage, and endeavored to get an idea of how the place looked. But I discovered, immediately, that they were totally insufficient for my purpose. They did little more than make the gloom visible. One thing they did, however, and that was, they confirmed my opinion of the size of the opening; and, although they showed me nothing that I wanted to see; yet the contrast they afforded to the heavy darkness, pleased me, curiously. It was as though fifteen tiny stars shone through the subterranean night.

Then, even as I stood, Pepper gave a sudden howl, that was taken up by the echoes, and repeated with ghastly variations, dying away, slowly. With a quick movement, I held aloft the one candle that I had kept, and glanced down at the dog; at the same moment, I seemed to hear a noise, like a diabolical chuckle, rise up from the hitherto, silent depths of the Pit. I started; then, I recollected that it was, probably, the echo of Pepper’s howl.

Pepper had moved away from me, up the passage, a few steps; he was nosing along the rocky floor; and I thought I heard him lapping. I went toward him, holding the candle low. As I moved, I heard my boot go sop, sop; and the light was reflected from something that glistened, and crept past my feet, swiftly toward the Pit. I bent lower, and looked; then gave vent to an expression of surprise. From somewhere, higher up the path, a stream of water was running quickly in the direction of the great opening, and growing in size every second.

Again, Pepper gave vent to that deep-drawn howl, and, running at me, seized my coat, and attempted to drag me up the path toward the entrance.

And now I understood the cause of the catastrophe. It was raining heavily, literally in torrents. The surface of the lake was level with the bottom of the opening---nay! more than level, it was above it. Evidently, the rain had swollen the lake, and caused this premature rise; for, at the rate the ravine had been filling, it would not have reached the entrance for a couple more days.

Luckily, the rope by which I had descended, was streaming into the opening, upon the inrushing waters. Seizing the end, I knotted it securely ‘round Pepper’s body, then, summoning up the last remnant of my strength, I commenced to swarm up the side of the cliff. I reached the ravine edge, in the last stage of exhaustion. Yet, I had to make one more effort, and haul Pepper into safety.

Slowly and wearily, I hauled on the rope. Once or twice, it seemed that I should have to give up; for Pepper is a weighty dog, and I was utterly done. Yet, to let go, would have meant certain death to the old fellow, and the thought spurred me to greater exertions. I have but a very hazy remembrance of the end. I recall pulling, through moments that lagged strangely. I have also some recollection of seeing Pepper’s muzzle, appearing over the ravine edge, after what seemed an indefinite period of time. Then, all grew suddenly dark.

\clearpage


% File borderland-07.txt
% Version 2017/09/18
% In the original, this was Chapter XIII.

\clearpage
\label{ch:07}

\begin{ChapterStart}
\null\null
\ChapterTitle{7. The Trap in the Great Cellar}
\end{ChapterStart}

It was not until a couple of days later, that I managed to get across to the ravine. There, I found that, in my few weeks’ absence, there had been wrought a wondrous change. Instead of the three-parts filled ravine, I looked out upon a great lake, whose placid surface, reflected the light, coldly. The water had risen to within half a dozen feet of the ravine edge. Only in one part was the lake disturbed, and that was above the place where, far down under the silent waters, yawned the entrance to the vast, underground Pit. Here, there was a continuous bubbling; and, occasionally, a curious sort of sobbing gurgle would find its way up from the depth. Beyond these, there was nothing to tell of the things that were hidden beneath. As I stood there, it came to me how wonderfully things had worked out. It was completely shut off and concealed from human curiosity forever.

Is it not possible that it has, all along, held a deeper significance, a hint---could one but have guessed---of the Pit that lies far down in the earth, beneath this old house? Under this house! Even now, the idea is strange and terrible to me. For I have proved, beyond doubt, that the Pit yawns right below the house, which is evidently supported, somewhere above the center of it, upon a tremendous, arched roof, of solid rock.

It happened in this wise, that, having occasion to go down to the cellars, the thought occurred to me to pay a visit to the great vault, where the trap is situated; and see whether everything was as I had left it.

Reaching the place, I walked slowly up the center, until I came to the trap. There it was, with the stones piled upon it, just as I had seen it last. I had a lantern with me, and the idea came to me, that now would be a good time to investigate whatever lay under the great, oak slab. Placing the lantern on the floor, I tumbled the stones off the trap, and, grasping the ring, pulled the door open. As I did so, the cellar became filled with the sound of a murmurous thunder, that rose from far below. At the same time, a damp wind blew up into my face, bringing with it a load of fine spray. Therewith, I dropped the trap, hurriedly, with a half frightened feeling of wonder.

For a moment, I stood puzzled. Then, a sudden thought possessed me, and I raised the ponderous door, with a feeling of excitement. Leaving it standing upon its end, I seized the lantern, and, kneeling down, thrust it into the opening. As I did so, the moist wind and spray drove in my eyes, making me unable to see, for a few moments. Even when my eyes were clear, I could distinguish nothing below me, save darkness, and whirling spray.

Seeing that it was useless to expect to make out anything, with the light so high, I felt in my pockets for a piece of twine, with which to lower it further into the opening. Even as I fumbled, the lantern slipped from my fingers, and hurtled down into the darkness. For a brief instant, I watched its fall, and saw the light shine on a tumult of white foam, some eighty or a hundred feet below me. Then it was gone. My sudden surmise was correct, and now, I knew the cause of the wet and noise. The great cellar was connected with the Pit, by means of the trap, which opened right above it; and the moisture, was the spray, rising from the water, falling into the depths.

In an instant, I had an explanation of certain things, that had hitherto puzzled me. These thoughts flashed through my brain, as I stood in the dark, searching my pockets for matches. I had the box in my hand now, and, striking a light, I stepped to the trap door, and closed it. Then, I piled the stones back upon it; after which, I made my way out from the cellars.

And so, I suppose the water goes on, thundering down into that bottomless hell-pit. Sometimes, I have an inexplicable desire to go down to the great cellar, open the trap, and gaze into the impenetrable, spray-damp darkness. At times, the desire becomes almost overpowering, in its intensity. It is not mere curiosity, that prompts me; but more as though some unexplained influence were at work. Still, I never go; and intend to fight down the strange longing, and crush it; even as I would the unholy thought of self-destruction.

This idea of some intangible force being exerted, may seem reasonless. Yet, my instinct warns me, that it is not so. In these things, reason seems to me less to be trusted than instinct.

One thought there is, in closing, that impresses itself upon me, with ever growing insistence. It is, that I live in a very strange house; a very awful house. And I have begun to wonder whether I am doing wisely in staying here. Yet, if I left, where could I go, and still obtain the solitude, and the sense of her presence, that alone make my old life bearable?\footnote{An apparently unmeaning interpolation. I can find no previous reference in the MS. to this matter. It becomes clearer, however, in the light of succeeding incidents.---\allsmcp{WHH}}

\clearpage

% File borderland-08.txt
% Version 2017/09/18
% In the original, this was Chapter XIV.

\clearpage
\label{ch:08}

\begin{ChapterStart}
\null\null
\ChapterTitle{8. The Sea of Sleep}
\end{ChapterStart}

For a considerable period after the last incident which I have narrated in my diary, I had serious thoughts of leaving this house, and might have done so; but for the great and wonderful thing, of which I am about to write.

How well I was advised, in my heart, when I stayed on here---spite of those visions and sights of unknown and unexplainable things; for, had I not stayed, then I had not seen again the face of her I loved. Yes, though few know it, I have loved and, ah! me---lost.

I would write down the story of those sweet, old days; but it would be like the tearing of old wounds; yet, after that which has happened, what need have I to care? For she has come to me out of the unknown. Strangely, she warned me; warned me passionately against this house; begged me to leave it; but admitted, when I questioned her, that she could not have come to me, had I been elsewhere. Yet, in spite of this, still she warned me, earnestly; telling me that it was a place, long ago given over to evil, and under the power of grim laws, of which none here have knowledge. And I---I just asked her, again, whether she would come to me elsewhere, and she could only stand, silent.

Thus I came to the place of the Sea of Sleep---so she termed it, in her dear speech with me. I had stayed up, in my study, reading; and must have dozed over the book. Suddenly, I awoke and sat upright, with a start. For a moment, I looked ’round, with a puzzled sense of something unusual. There was a misty look about the room, giving a curious softness to each table and chair and furnishing.

Gradually, the mistiness increased; growing, as it were, out of nothing. Then, slowly, a soft, white light began to glow in the room. The flames of the candles shone through it, palely. I looked from side to side, and found that I could still see each piece of furniture; but in a strangely unreal way, more as though the ghost of each table and chair had taken the place of the solid article.

Gradually, as I looked, I saw them fade and fade; until, slowly, they resolved into nothingness. Now, I looked again at the candles. They shone wanly, and, even as I watched, grew more unreal, and so vanished. The room was filled, now, with a soft, yet luminous, white twilight, like a gentle mist of light. Beyond this, I could see nothing. Even the walls had vanished.

Presently, I became conscious that a faint, continuous sound, pulsed through the silence that wrapped me. I listened intently. It grew more distinct, until it appeared to me that I harked to the breathings of some great sea. I cannot tell how long a space passed thus; but, after a while, it seemed that I could see through the mistiness; and, slowly, I became aware that I was standing upon the shore of an immense and silent sea. This shore was smooth and long, vanishing to right and left of me, in extreme distances. In front, swam a still immensity of sleeping ocean. At times, it seemed to me that I caught a faint glimmer of light, under its surface; but of this, I could not be sure. Behind me, rose up, to an extraordinary height, gaunt, black cliffs.

Overhead, the sky was of a uniform cold grey color---the whole place being lit by a stupendous globe of pale fire, that swam a little above the far horizon, and shed a foamlike light above the quiet waters.

Beyond the gentle murmur of the sea, an intense stillness prevailed. For a long while, I stayed there, looking out across its strangeness. Then, as I stared, it seemed that a bubble of white foam floated up out of the depths, and then, even now I know not how it was, I was looking upon, nay, looking \textit{into} the face of Her---aye! into her face---into her soul; and she looked back at me, with such a commingling of joy and sadness, that I ran toward her, blindly; crying strangely to her, in a very agony of remembrance, of terror, and of hope, to come to me. Yet, spite of my crying, she stayed out there upon the sea, and only shook her head, sorrowfully; but, in her eyes was the old earth-light of tenderness, that I had come to know, before all things, ere we were parted.

At her perverseness, I grew desperate, and essayed to wade out to her; yet, though I would, I could not. Something, some invisible barrier, held me back, and I was fain to stay where I was, and cry out to her in the fullness of my soul, ‘O, my Darling, my Darling---’ but could say no more, for very intensity. And, at that, she came over, swiftly, and touched me, and it was as though heaven had opened. Yet, when I reached out my hands to her, she put me from her with tenderly stern hands, and I was abashed.

\clearpage

% The following poem was appended to the plain-text source I used.
% It seemed out of context there.
% I have taken the liberty of moving it here, as if it appeared in the diary at this point.

\vspace*{0.5\nbs} % so that title does not intrude into top margin
{\centering\charscale[1.2]{Grief}\par}
\vspace{0.5\nbs}
\begin{adjustwidth}{5.5em}{0em} % eyeball alignment
\noindent Fierce hunger reigns within my breast,\\
I had not dreamt that this whole world,\\
Crushed in the hand of God, could yield\\
Such bitter essence of unrest,\\
Such pain as Sorrow now hath hurled\\
Out of its dreadful heart, unsealed!\par
\vspace{0.5\nbs} % doesn't quite fit on one page, with full skip
\noindent Each sobbing breath is but a cry,\\
My heart-strokes knells of agony,\\
And my whole brain has but one thought\\
That nevermore through life shall I\\
(Save in the ache of memory)\\
Touch hands with thee, who now art naught!\par
\vspace{0.5\nbs}
\noindent Through the whole void of night I search,\\
So dumbly crying out to thee;\\
But thou are \textit{not}; and night’s vast throne\\
Becomes an all stupendous church\\
With star-bells knelling unto me\\
Who in all space am most alone!\par
\vspace{0.5\nbs}
\noindent An hungered, to the shore I creep,\\
Perchance some comfort waits on me\\
From the old Sea’s eternal heart;\\
But lo! from all the solemn deep,\\
Far voices out of mystery\\
Seem questioning why we are apart!\par
\vspace{0.5\nbs}
\noindent Where’er I go I am alone\\
Who once, through thee, had all the world.\\
My breast is one whole raging pain\\
For that which \textit{was}, and now is flown\\
Into the Blank where life is hurled\\
Where all is not, nor is again!\par
\end{adjustwidth}

\clearpage

% File borderland-09.txt
% Version 2017/09/18
% In the original, this was Chapter XV.

\clearpage
\label{ch:09}

\begin{ChapterStart}
\null\null
\ChapterTitle{9. The Noise in the Night}
\end{ChapterStart}

And now, I come to the strangest of all the strange happenings that have befallen me in this house of mysteries. It occurred quite lately---within the month; and I have little doubt but that what I saw was in reality the end of all things. However, to my story.

I do not know how it is; but, up to the present, I have never been able to write these things down, directly they happened. It is as though I have to wait a time, recovering my just balance, and digesting---as it were---the things I have heard or seen. No doubt, this is as it should be; for, by waiting, I see the incidents more truly, and write of them in a calmer and more judicial frame of mind.

It is now the end of November. My story relates to what happened in the first week of the month.

It was night, about eleven o’clock. Pepper and I kept one another company in the study---that great, old room of mine, where I read and work. I was reading, curiously enough, the Bible. I have begun, in these later days, to take a growing interest in that great and ancient book. Suddenly, a distinct tremor shook the house, and there came a faint and distant, whirring buzz, that grew rapidly into a far, muffled screaming. It reminded me, in a queer, gigantic way, of the noise that a clock makes, when the catch is released, and it is allowed to run down. The sound appeared to come from some remote height---somewhere up in the night. There was no repetition of the shock. I looked across at Pepper. He was sleeping peacefully.

Gradually, the whirring noise decreased, and there came a long silence.

All at once, a glow lit up the end window, which protrudes far out from the side of the house, so that, from it, one may look both East and West. I felt puzzled, and, after a moment’s hesitation, walked across the room, and pulled aside the blind. As I did so, I saw the Sun rise, from behind the horizon. It rose with a steady, perceptible movement. I could see it travel upward. In a minute, it seemed, it had reached the tops of the trees, through which I had watched it. Up, up---It was broad daylight now. Behind me, I was conscious of a sharp, mosquito-like buzzing. I glanced ’round, and knew that it came from the clock. Even as I looked, it marked off an hour. The minute hand was moving ’round the dial, faster than an ordinary second-hand. The hour hand moved quickly from space to space. I had a numb sense of astonishment. A moment later, so it seemed, the two candles went out, almost together. I turned swiftly back to the window; for I had seen the shadow of the window-frames, traveling along the floor toward me, as though a great lamp had been carried up past the window.

I saw now, that the sun had risen high into the heavens, and was still visibly moving. It passed above the house, with an extraordinary sailing kind of motion. As the window came into shadow, I saw another extraordinary thing. The fine-weather clouds were not passing, easily, across the sky---they were scampering, as though a hundred-mile-an-hour wind blew. As they passed, they changed their shapes a thousand times a minute, as though writhing with a strange life; and so were gone. And, presently, others came, and whisked away likewise.

To the West, I saw the sun, drop with an incredible, smooth, swift motion. Eastward, the shadows of every seen thing crept toward the coming greyness. And the movement of the shadows was visible to me---a stealthy, writhing creep of the shadows of the wind-stirred trees. It was a strange sight.

Quickly, the room began to darken. The sun slid down to the horizon, and seemed, as it were, to disappear from my sight, almost with a jerk. Through the greyness of the swift evening, I saw the silver crescent of the moon, falling out of the Southern sky, toward the West. The evening seemed to merge into an almost instant night. Above me, the many constellations passed in a strange, ‘noiseless’ circling, Westward. The moon fell through that last thousand fathoms of the night-gulf, and there was only the starlight....

About this time, the buzzing in the corner ceased; telling me that the clock had run down. A few minutes passed, and I saw the Eastward sky lighten. A grey, sullen morning spread through all the darkness, and hid the march of the stars. Overhead, there moved, with a heavy, everlasting rolling, a vast, seamless sky of grey clouds---a cloud-sky that would have seemed motionless, through all the length of an ordinary earth-day. The sun was hidden from me; but, from moment to moment, the world would brighten and darken, brighten and darken, beneath waves of subtle light and shadow....

The light shifted ever Westward, and the night fell upon the earth. A vast rain seemed to come with it, and a wind of a most extraordinary loudness---as though the howling of a nightlong gale, were packed into the space of no more than a minute.

This noise passed, almost immediately, and the clouds broke; so that, once more, I could see the sky. The stars were flying Westward, with astounding speed. It came to me now, for the first time, that, though the noise of the wind had passed, yet a constant ‘blurred’ sound was in my ears. Now that I noticed it, I was aware that it had been with me all the time. It was the world-noise.

And then, even as I grasped at so much comprehension, there came the Eastward light. No more than a few heartbeats, and the sun rose, swiftly. Through the trees, I saw it, and then it was above the trees. Up---up, it soared and all the world was light. It passed, with a swift, steady swing to its highest altitude, and fell thence, Westward. I saw the day roll visibly over my head. A few light clouds flittered Northward, and vanished. The sun went down with one swift, clear plunge, and there was about me, for a few seconds, the darker growing grey of the gloaming.

Southward and Westward, the moon was sinking rapidly. The night had come, already. A minute it seemed, and the moon fell those remaining fathoms of dark sky. Another minute, or so, and the Eastward sky glowed with the coming dawn. The sun leapt upon me with a frightening abruptness, and soared ever more swiftly toward the zenith. Then, suddenly, a fresh thing came to my sight. A black thundercloud rushed up out of the South, and seemed to leap all the arc of the sky, in a single instant. As it came, I saw that its advancing edge flapped, like a monstrous black cloth in the heaven, twirling and undulating rapidly, with a horrid suggestiveness. In an instant, all the air was full of rain, and a hundred lightning flashes seemed to flood downward, as it were in one great shower. In the same second of time, the world-noise was drowned in the roar of the wind, and then my ears ached, under the stunning impact of the thunder.

And, in the midst of this storm, the night came; and then, within the space of another minute, the storm had passed, and there was only the constant ‘blur’ of the world-noise on my hearing. Overhead, the stars were sliding quickly Westward; and something, mayhaps the particular speed to which they had attained, brought home to me, for the first time, a keen realization of the knowledge that it was the world that revolved. I seemed to see, suddenly, the world---a vast, dark mass---revolving visibly against the stars.

The dawn and sun seemed to come together, so greatly had the speed of the world-revolution increased. The sun drove up, in one long, steady curve; passed its highest point, swept down into the Western sky, and disappeared. I was scarcely conscious of evening, so brief was it. Then I was watching the flying constellations, and the Westward hastening moon. In but a space of seconds, so it seemed, it was sliding swiftly downward through the night-blue, and then was gone. And, almost directly, came the morning.

And now there seemed to come a strange acceleration. The sun made one clean, clear sweep through the sky, and disappeared behind the Westward horizon, and the night came and went with a like haste.

As the succeeding day, opened and closed upon the world, I was aware of a sweat of snow, suddenly upon the earth. The night came, and, almost immediately, the day. In the brief leap of the sun, I saw that the snow had vanished; and then, once more, it was night.

Thus matters were; and, even after the many incredible things that I have seen, I experienced all the time a most profound awe. To see the sun rise and set, within a space of time to be measured by seconds; to watch (after a little) the moon leap---a pale, and ever growing orb---up into the night sky, and glide, with a strange swiftness, through the vast arc of blue; and, presently, to see the sun follow, springing out of the Eastern sky, as though in chase; and then again the night, with the swift and ghostly passing of starry constellations, was all too much to view believingly. Yet, so it was---the day slipping from dawn to dusk, and the night sliding swiftly into day, ever rapidly and more rapidly.

The last three passages of the sun had shown me a snow-covered earth, which, at night, had seemed, for a few seconds, incredibly weird under the fast-shifting light of the soaring and falling moon. Now, however, for a little space, the sky was hidden, by a sea of swaying, leaden-white clouds, which lightened and blackened, alternately, with the passage of day and night.

The clouds rippled and vanished, and there was once more before me, the vision of the swiftly leaping sun, and nights that came and went like shadows.

Faster and faster, spun the world. And now each day and night was completed within the space of but a few seconds; and still the speed increased.

It was a little later, that I noticed that the sun had begun to have the suspicion of a trail of fire behind it. This was due, evidently, to the speed at which it, apparently, traversed the heavens. And, as the days sped, each one quicker than the last, the sun began to assume the appearance of a vast, flaming comet\footnote{The Recluse uses this as an illustration, evidently in the sense of the popular conception of a comet.---\allsmcp{WHH}} flaring across the sky at short, periodic intervals. At night, the moon presented, with much greater truth, a comet-like aspect; a pale, and singularly clear, fast traveling shape of fire, trailing streaks of cold flame. The stars showed now, merely as fine hairs of fire against the dark.

Once, I turned from the window, and glanced at Pepper. In the flash of a day, I saw that he slept, quietly, and I moved once more to my watching.

The sun was now bursting up from the Eastern horizon, like a stupendous rocket, seeming to occupy no more than a second or two in hurling from East to West. I could no longer perceive the passage of clouds across the sky, which seemed to have darkened somewhat. The brief nights, appeared to have lost the proper darkness of night; so that the hair-like fire of the flying stars, showed but dimly. As the speed increased, the sun began to sway very slowly in the sky, from South to North, and then, slowly again, from North to South.

So, amid a strange confusion of mind, the hours passed.

All this while had Pepper slept. Presently, feeling lonely and distraught, I called to him, softly; but he took no notice. Again, I called, raising my voice slightly; still he moved not. I walked over to where he lay, and touched him with my foot, to rouse him. At the action, gentle though it was, he fell to pieces. That is what happened; he literally and actually crumbled into a mouldering heap of bones and dust.

For the space of, perhaps a minute, I stared down at the shapeless heap, that had once been Pepper. I stood, feeling stunned. What can have happened? I asked myself; not at once grasping the grim significance of that little hill of ash. Then, as I stirred the heap with my foot, it occurred to me that this could only happen in a great space of time. Years---and years.

Outside, the weaving, fluttering light held the world. Inside, I stood, trying to understand what it meant---what that little pile of dust and dry bones, on the carpet, meant. But I could not think, coherently.

I glanced away, ’round the room, and now, for the first time, noticed how dusty and old the place looked. Dust and dirt everywhere; piled in little heaps in the corners, and spread about upon the furniture. The very carpet, itself, was invisible beneath a coating of the same, all pervading, material. As I walked, little clouds of the stuff rose up from under my footsteps, and assailed my nostrils, with a dry, bitter odor that made me wheeze, huskily.

Suddenly, as my glance fell again upon Pepper’s remains, I stood still, and gave voice to my confusion---questioning, aloud, whether the years were, indeed, passing; whether this, which I had taken to be a form of vision, was, in truth, a reality. I paused. A new thought had struck me. Quickly, but with steps which, for the first time, I noticed, tottered, I went across the room to the great pier-glass, and looked in. It was too covered with grime, to give back any reflection, and, with trembling hands, I began to rub off the dirt. Presently, I could see myself. The thought that had come to me, was confirmed. Instead of the great, hale man, who scarcely looked fifty, I was looking at a bent, decrepit man, whose shoulders stooped, and whose face was wrinkled with the years of a century. The hair---which a few short hours ago had been nearly coal black---was now silvery white. Only the eyes were bright. Gradually, I traced, in that ancient man, a faint resemblance to my self of other days.

I turned away, and tottered to the window. I knew, now, that I was old, and the knowledge seemed to confirm my trembling walk. For a little space, I stared moodily out into the blurred vista of changeful landscape. Even in that short time, a year passed, and, with a petulant gesture, I left the window. As I did so, I noticed that my hand shook with the palsy of old age; and a short sob choked its way through my lips.

For a little while, I paced, tremulously, between the window and the table; my gaze wandering hither and thither, uneasily. How dilapidated the room was. Everywhere lay the thick dust. The fender was a shape of rust. The chains that held the brass clock-weights, had rusted through long ago, and now the weights lay on the floor beneath; themselves two cones of verdigris.

As I glanced about, it seemed to me that I could see the very furniture of the room rotting and decaying before my eyes. Nor was this fancy, on my part; for, all at once, the bookshelf, along the sidewall, collapsed, with a cracking and rending of rotten wood, precipitating its contents upon the floor, and filling the room with a smother of dusty atoms.

How tired I felt. As I walked, it seemed that I could hear my dry joints, creak and crack at every step. All had happened so quickly and suddenly. This must be, indeed, the beginning of the end of all things! It occurred to me, to go to look for her; but I felt too weary. And then, she had been so queer about these happenings, of late. Of late! I repeated the words, and laughed, feebly---mirthlessly, as the realization was borne in upon me that I spoke of a time, half a century gone. Half a century! It might have been twice as long!

I moved slowly to the window, and looked out once more across the world. I can best describe the passage of day and night, at this period, as a sort of gigantic, ponderous flicker. Moment by moment, the acceleration of time continued; so that, at nights now, I saw the moon, only as a swaying trail of palish fire, that varied from a mere line of light to a nebulous path, and then dwindled again, disappearing periodically.

The flicker of the days and nights quickened. The days had grown perceptibly darker, and a queer quality of dusk lay, as it were, in the atmosphere. The nights were so much lighter, that the stars were scarcely to be seen, saving here and there an occasional hair-like line of fire, that seemed to sway a little, with the moon.

Quicker, and ever quicker, ran the flicker of day and night; and, suddenly it seemed, I was aware that the flicker had died out, and, instead, there reigned a comparatively steady light, which was shed upon all the world, from an eternal river of flame that swung up and down, North and South, in stupendous, mighty swings.

The sky was now grown very much darker, and there was in the blue of it a heavy gloom, as though a vast blackness peered through it upon the earth. Yet, there was in it, also, a strange and awful clearness, and emptiness. Periodically, I had glimpses of a ghostly track of fire that swayed thin and darkly toward the sun-stream; vanished and reappeared. It was the scarcely visible moon-stream.

Looking out at the landscape, I was conscious again, of a blurring ‘flitter,’ that came either from the light of the ponderous, swinging sun-stream, or was the result of the incredibly rapid changes of the earth’s surface. And every few moments, so it seemed, the snow would lie suddenly upon the world, and vanish as abruptly, as though an invisible giant ‘flitted’ a white sheet off and on the earth.

Time fled, and the weariness that was mine, grew insupportable. I turned from the window, and walked once across the room, the heavy dust deadening the sound of my footsteps. Each step that I took, seemed a greater effort than the one before. An intolerable ache, knew me in every joint and limb, as I trod my way, with a weary uncertainty.

By the opposite wall, I came to a weak pause, and wondered, dimly, what was my intent. I looked to my left, and saw my old chair. The thought of it brought a faint sense of comfort to my bewildered wretchedness. Yet, because I was so weary and old and tired, I would scarcely brace my mind to do anything but stand, and wish myself past those few yards. I rocked, as I stood. The floor, even, seemed a place for rest; but the dust lay so thick and sleepy and black. I turned, with a great effort of will, and made toward my chair. I reached it, with a groan of thankfulness. I sat down.

Everything about me appeared to be growing dim. It was all so strange and unthought of. Last night, I was a comparatively strong, though elderly man; and now, only a few hours later---! I looked at the little dust-heap that had once been Pepper. Hours! and I laughed, a feeble, bitter laugh; a shrill, cackling laugh, that shocked my dimming senses.

For a while, I must have dozed. Then I opened my eyes, with a start. Somewhere across the room, there had been a muffled noise of something falling. I looked, and saw, vaguely, a cloud of dust hovering above a pile of \textit{débris}. Nearer the door, something else tumbled, with a crash. It was one of the cupboards; but I was tired, and took little notice. I closed my eyes, and sat there in a state of drowsy, semi-unconsciousness. Once or twice---as though coming through thick mists---I heard noises, faintly. Then I must have slept.

\clearpage

% File borderland-10.txt
% Version 2017/09/18
% In the original, this was Chapter XVI.

\clearpage
\label{ch:10}

\begin{ChapterStart}
\null\null
\ChapterTitle{10. The Awakening}
\end{ChapterStart}

I awoke, with a start. For a moment, I wondered where I was. Then memory came to me....

The room was still lit with that strange light---half-sun, half-moon. I felt refreshed, and the tired, weary ache had left me. I went slowly to the window, and looked out. Overhead, the river of flame drove up and down, North and South, in a dancing semi-circle of fire. As a mighty sleigh in the loom of time it seemed---in a sudden fancy of mine---to be beating home the picks of the years. For, so vastly had the passage of time been accelerated, that there was no longer any sense of the sun passing from East to West. The only apparent movement was the North and South beat of the sun-stream, that had become so swift now, as to be better described as a \textit{quiver}.

As I peered out, there came to me a sudden, inconsequent memory of that last journey among the Outer worlds. I remembered the sudden vision that had come to me, as I neared the Solar System, of the fast whirling planets about the sun---as though the governing quality of time had been held in abeyance, and the Machine of a Universe allowed to run down an eternity, in a few moments or hours. The memory passed, along with a, but partially comprehended, suggestion that I had been permitted a glimpse into further time spaces. I stared out again, seemingly, at the quake of the sun-stream. The speed seemed to increase, even as I looked. Several lifetimes came and went, as I watched.

Suddenly, it struck me, with a sort of grotesque seriousness, that I was still alive. I thought of Pepper, and wondered how it was that I had not followed his fate. He had reached the time of his dying, and had passed, probably through sheer length of years. And here was I, alive, hundreds of thousands of centuries after my rightful period of years.

For, a time, I mused, absently. ‘Yesterday---’ I stopped, suddenly. Yesterday! There was no yesterday. The yesterday of which I spoke had been swallowed up in the abyss of years, ages gone. I grew dazed with much thinking.

Presently, I turned from the window, and glanced ’round the room. It seemed different---strangely, utterly different. Then, I knew what it was that made it appear so strange. It was bare: there was not a piece of furniture in the room; not even a solitary fitting of any sort. Gradually, my amazement went, as I remembered, that this was but the inevitable end of that process of decay, which I had witnessed commencing, before my sleep. Millions of years!

Over the floor was spread a deep layer of dust, that reached half way up to the window-seat. It had grown immeasurably, whilst I slept; and represented the dust of untold ages. Undoubtedly, atoms of the old, decayed furniture helped to swell its bulk; and, somewhere among it all, mouldered the long-ago-dead Pepper.

All at once, it occurred to me, that I had no recollection of wading knee-deep through all that dust, after I awoke. True, an incredible age of years had passed, since I approached the window; but that was evidently as nothing, compared with the countless spaces of time that, I conceived, had vanished whilst I was sleeping. I remembered now, that I had fallen asleep, sitting in my old chair. Had it gone ...? I glanced toward where it had stood. Of course, there was no chair to be seen. I could not satisfy myself, whether it had disappeared, after my waking, or before. If it had mouldered under me, surely, I should have been waked by the collapse. Then I remembered that the thick dust, which covered the floor, would have been sufficient to soften my fall; so that it was quite possible, I had slept upon the dust for a million years or more.

As these thoughts wandered through my brain, I glanced again, casually, to where the chair had stood. Then, for the first time,\linebreak I noticed that there were no marks, in the dust, of my footprints, between it and the window. But then, ages of years had passed, since I had awaked---tens of thousands of years!

My look rested thoughtfully, again upon the place where once had stood my chair. Suddenly, I passed from abstraction to intentness; for there, in its standing place, I made out a long undulation, rounded off with the heavy dust. Yet it was not so much hidden, but that I could tell what had caused it. I knew---and shivered at the knowledge---that it was a human body, ages-dead, lying there, beneath the place where I had slept. It was lying on its right side, its back turned toward me. I could make out and trace each curve and outline, softened, and moulded, as it were, in the black dust. In a vague sort of way, I tried to account for its presence there. Slowly, I began to grow bewildered, as the thought came to me that it lay just about where I must have fallen when the chair collapsed.

Gradually, an idea began to form itself within my brain; a thought that shook my spirit. It seemed hideous and insupportable; yet it grew upon me, steadily, until it became a conviction. The body under that coating, that shroud of dust, was neither more nor less than my own dead shell. I did not attempt to prove it. I knew it now, and wondered I had not known it all along. I was a bodiless thing.

Awhile, I stood, trying to adjust my thoughts to this new problem. In time---how many thousands of years, I know not---I attained to some degree of quietude---sufficient to enable me to pay attention to what was transpiring around me.

Now, I saw that the elongated mound had sunk, collapsed, level with the rest of the spreading dust. And fresh atoms, impalpable, had settled above that mixture of grave-powder, which the aeons had ground. A long while, I stood, turned from the window. Gradually, I grew more collected, while the world slipped across the centuries into the future.

Presently, I began a survey of the room. Now, I saw that time was beginning its destructive work, even on this strange old building. That it had stood through all the years was, it seemed to me, proof that it was something different from any other house. I do not think, somehow, that I had thought of its decaying. It was not until I had meditated upon the matter, for some considerable time, that I fully realized that the extraordinary space of time through which it had stood, was sufficient to have utterly pulverized the very stones of which it was built, had they been taken from any earthly quarry. Yes, it was undoubtedly mouldering now. All the plaster had gone from the walls; even as the woodwork of the room had gone, many ages before.

While I stood, in contemplation, a piece of glass, from one of the small, diamond-shaped panes, dropped, with a dull tap, amid the dust upon the sill behind me, and crumbled into a little heap of powder. As I turned from contemplating it, I saw light between a couple of the stones that formed the outer wall. Evidently, the mortar was falling away....

After awhile, I turned once more to the window, and peered out. I discovered, now, that the speed of time had become enormous. The lateral quiver of the sun-stream, had grown so swift as to cause the dancing semi-circle of flame to merge into, and disappear in, a sheet of fire that covered half the Southern sky from East to West.

From the sky, I glanced down to the gardens. They were just a blur of a palish, dirty green. I had a feeling that they stood higher, than in the old days; a feeling that they were nearer my window, as though they had risen, bodily. Yet, they were still a long way below me; for the rock, over the mouth of the pit, on which this house stands, arches up to a great height.

It was later, that I noticed a change in the constant color of the gardens. The pale, dirty green was growing ever paler and paler, toward white. At last, after a great space, they became greyish-white, and stayed thus for a very long time. Finally, however, the greyness began to fade, even as had the green, into a dead white. And this remained, constant and unchanged. And by this I knew that, at last, snow lay upon all the Northern world.

And so, by millions of years, time winged onward through eternity, to the end---the end, of which, in the old-earth days, I had thought remotely, and in hazily speculative fashion. And now, it was approaching in a manner of which none had ever dreamed.

I recollect that, about this time, I began to have a lively, though morbid, curiosity, as to what would happen when the end came---but I seemed strangely without imaginings.

All this while, the steady process of decay was continuing. The few remaining pieces of glass, had long ago vanished; and, every now and then, a soft thud, and a little cloud of rising dust, would tell of some fragment of fallen mortar or stone.

I looked up again, to the fiery sheet that quaked in the heavens above me and far down into the Southern sky. As I looked, the impression was borne in upon me, that it had lost some of its first brilliancy---that it was duller, deeper hued.

I glanced down, once more, to the blurred white of the worldscape. Sometimes, my look returned to the burning sheet of dulling flame, that was, and yet hid, the sun. At times, I glanced behind me, into the growing dusk of the great, silent room, with its aeon-carpet of sleeping dust....

So, I watched through the fleeting ages, lost in soul-wearing thoughts and wonderings, and possessed with a new weariness.

\clearpage

% File borderland-11.txt
% Version 2017/09/18
% In the original, this was Chapter XVII.

\clearpage
\label{ch:11}

\begin{ChapterStart}
\null\null
\ChapterTitle{11. The Slowing Rotation}
\end{ChapterStart}

It might have been a million years later, that I perceived, beyond possibility of doubt, that the fiery sheet that lit the world, was indeed darkening.

Another vast space went by, and the whole enormous flame had sunk to a deep, copper color. Gradually, it darkened, from copper to copper-red, and from this, at times, to a deep, heavy, purplish tint, with, in it, a strange loom of blood.

Although the light was decreasing, I could perceive no diminishment in the apparent speed of the sun. It still spread itself in that dazzling veil of speed.

The world, so much of it as I could see, had assumed a dreadful shade of gloom, as though, in very deed, the last day of the worlds approached.

The sun was dying; of that there could be little doubt; and still the earth whirled onward, through space and all the aeons. At this time, I remember, an extraordinary sense of bewilderment took me. I found myself, later, wandering, mentally, amid an odd chaos of fragmentary modern theories and the old Biblical story of the world’s ending.

Then, for the first time, there flashed across me, the memory that the sun, with its system of planets, was, and had been, traveling through space at an incredible speed. Abruptly, the question rose---\textit{Where?} For a very great time, I pondered this matter; but, finally, with a certain sense of the futility of my puzzlings, I let my thoughts wander to other things. I grew to wondering, how much longer the house would stand. Also, I queried, to myself, whether I should be doomed to stay, bodiless, upon the earth, through the dark-time that I knew was coming. From these thoughts, I fell again to speculations upon the possible direction of the sun’s journey through space.... And so another great while passed.

Gradually, as time fled, I began to feel the chill of a great winter. Then, I remembered that, with the sun dying, the cold must be, necessarily, extraordinarily intense. Slowly, slowly, as the aeons slipped into eternity, the earth sank into a heavier and redder gloom. The dull flame in the firmament took on a deeper tint, very somber and turbid.

Then, at last, it was borne upon me that there was a change. The fiery, gloomy curtain of flame that hung quaking overhead, and down away into the Southern sky, began to thin and contract; and, in it, as one sees the fast vibrations of a jarred harp-string, I saw once more the sun-stream quivering, giddily, North and South.

Slowly, the likeness to a sheet of fire, disappeared, and I saw, plainly, the slowing beat of the sun-stream. Yet, even then, the speed of its swing was inconceivably swift. And all the time, the brightness of the fiery arc grew ever duller. Underneath, the world loomed dimly---an indistinct, ghostly region.

Overhead, the river of flame swayed slower, and even slower; until, at last, it swung to the North and South in great, ponderous beats, that lasted through seconds. A long space went by, and now each sway of the great belt lasted nigh a minute; so that, after a great while, I ceased to distinguish it as a visible movement; and the streaming fire ran in a steady river of dull flame, across the deadly-looking sky.

An indefinite period passed, and it seemed that the arc of fire became less sharply defined. It appeared to me to grow more attenuated, and I thought blackish streaks showed, occasionally. Presently, as I watched, the smooth onward-flow ceased; and I was able to perceive that there came a momentary, but regular, darkening of the world. This grew until, once more, night descended, in short, but periodic, intervals upon the wearying earth.

Longer and longer became the nights, and the days equaled them; so that, at last, the day and the night grew to the duration of seconds in length, and the sun showed, once more, like an almost invisible, coppery-red colored ball, within the glowing mistiness of its flight. Corresponding to the dark lines, showing at times in its trail, there were now distinctly to be seen on the half-visible sun itself, great, dark belts.

Year after year flashed into the past, and the days and nights spread into minutes. The sun had ceased to have the appearance of a tail; and now rose and set---a tremendous globe of a glowing copper-bronze hue; in parts ringed with blood-red bands; in others, with the dusky ones, that I have already mentioned. These circles---both red and black---were of varying thicknesses. For a time, I was at a loss to account for their presence. Then it occurred to me, that it was scarcely likely that the sun would cool evenly all over; and that these markings were due, probably, to differences in temperature of the various areas; the red representing those parts where the heat was still fervent, and the black those portions which were already comparatively cool.

It struck me, as a peculiar thing, that the sun should cool in evenly defined rings; until I remembered that, possibly, they were but isolated patches, to which the enormous rotatory speed of the sun had imparted a belt-like appearance. The sun, itself, was very much greater than the sun I had known in the old-world days; and, from this, I argued that it was considerably nearer.

At nights, the moon\footnote{No further mention is made of the moon. From what is said here, it is evident that our satellite had greatly increased its distance from the earth. Possibly, at a later age it may even have broken loose from our attraction. I cannot but regret that no light is shed on this point.---\allsmcp{WHH}} still showed; but small and remote; and the light she reflected was so dull and weak that she seemed little more than the small, dim ghost of the olden moon, that I had known.

The days and nights lengthened out, until they equaled a space somewhat less than one of the old-earth hours; the sun rising and setting like a great, ruddy bronze disk, crossed with ink-black bars. About this time, I found myself, able once more, to see the gardens, with clearness. For the world had now grown very still, and changeless. Yet, I am not correct in saying, ‘gardens’; for there were no gardens---nothing that I knew or recognized. In place thereof, I looked out upon a vast plain, stretching away into distance. A little to my left, there was a low range of hills. Everywhere, there was a uniform, white covering of snow, in places rising into hummocks and ridges.

It was only now, that I recognized how really great had been the snowfall. In places it was vastly deep, as was witnessed by a great, upleaping, wave-shaped hill, away to my right; though it is not impossible, that this was due, in part, to some rise in the surface of the ground. Strangely enough, the range of low hills to my left---already mentioned---was not entirely covered with the universal snow; instead, I could see their bare, dark sides showing in several places. And everywhere and always there reigned an incredible death-silence and desolation. The immutable, awful quiet of a dying world.

All this time, the days and nights were lengthening, perceptibly. Already, each day occupied, maybe, some two hours from dawn to dusk. At night, I had been surprised to find that there were very few stars overhead, and these small, though of an extraordinary brightness; which I attributed to the peculiar, but clear, blackness of the nighttime.

Away to the North, I could discern a nebulous sort of mistiness; not unlike, in appearance, a small portion of the Milky Way. It might have been an extremely remote star-cluster; or---the thought came to me suddenly---perhaps it was the sidereal universe that I had known, and now left far behind, forever---a small, dimly glowing mist of stars, far in the depths of space.

Still, the days and nights lengthened, slowly. Each time, the sun rose duller than it had set. And the dark belts increased in breadth.

About this time, there happened a fresh thing. The sun, earth, and sky were suddenly darkened, and, apparently, blotted out for a brief space. I had a sense, a certain awareness (I could learn little by sight), that the earth was enduring a very great fall of snow. Then, in an instant, the veil that had obscured everything, vanished, and I looked out, once more. A marvelous sight met my gaze. The hollow in which this house, with its gardens, stands, was brimmed with snow.\footnote{Conceivably, frozen air.---\allsmcp{WHH}} It lipped over the sill of my window. Everywhere, it lay, a great level stretch of white, which caught and reflected, gloomily, the somber coppery glows of the dying sun. The world had become a shadowless plain, from horizon to horizon.

I glanced up at the sun. It shone with an extraordinary, dull clearness. I saw it, now, as one who, until then, had seen it, only through a partially obscuring medium. All about it, the sky had become black, with a clear, deep blackness, frightful in its nearness, and its unmeasured deep, and its utter unfriendliness. For a great time, I looked into it, newly, and shaken and fearful. It was so near. Had I been a child, I might have expressed some of my sensation and distress, by saying that the sky had lost its roof.

Later, I turned, and peered about me, into the room. Everywhere, it was covered with a thin shroud of the all-pervading white. I could see it but dimly, by reason of the somber light that now lit the world. It appeared to cling to the ruined walls; and the thick, soft dust of the years, that covered the floor knee-deep, was nowhere visible. The snow must have blown in through the open framework of the windows. Yet, in no place had it drifted; but lay everywhere about the great, old room, smooth and level. Moreover, there had been no wind these many thousand years. But there was the snow.\footnote{Previous footnote would explain the snow (?) within the room.---\allsmcp{WHH}}

And all the earth was silent. And there was a cold, such as no living man can ever have known.

The earth was now illuminated, by day, with a most doleful light, beyond my power to describe. It seemed as though I looked at the great plain, through the medium of a bronze-tinted sea.

It was evident that the earth’s rotatory movement was departing, steadily.

The end came, all at once. The night had been the longest yet; and when the dying sun showed, at last, above the world’s edge, I had grown so wearied of the dark, that I greeted it as a friend. It rose steadily, until about twenty degrees above the horizon. Then, it stopped suddenly, and, after a strange retrograde movement, hung motionless---a great shield in the sky.\footnote{I am confounded that neither here, nor later on, does the Recluse make any further mention of the continued north and south movement (apparent, of course,) of the sun from solstice to solstice.---\allsmcp{WHH}} Only the circular rim of the sun showed bright---only this, and one thin streak of light near the equator.

Gradually, even this thread of light died out; and now, all that was left of our great and glorious sun, was a vast dead disk, rimmed with a thin circle of bronze-red light.

\clearpage

% File borderland-12.txt
% Version 2017/09/18
% In the original, this was Chapter XVIII.

\clearpage
\label{ch:12}

\begin{ChapterStart}
\null\null
\ChapterTitle{12. The Green Star}
\end{ChapterStart}

The world was held in a savage gloom---cold and intolerable. Outside, all was quiet---quiet! From the dark room behind me, came the occasional, soft thud of falling matter---fragments of rotting stone.\footnote{At this time the sound-carrying atmosphere must have been either incredibly attenuated, or---more probably---nonexistent. In the light of this, it cannot be supposed that these, or any other, noises would have been apparent to living ears---to hearing, as we, in the material body, understand that sense.---\allsmcp{WHH}} So time passed, and night grasped the world, wrapping it in wrappings of impenetrable blackness.

There was no night-sky, as we know it. Even the few straggling stars had vanished, conclusively. I might have been in a shuttered room, without a light; for all that I could see. Only, in the impalpableness of gloom, opposite, burnt that vast, encircling hair of dull fire. Beyond this, there was no ray in all the vastitude of night that surrounded me; save that, far in the North, that soft, mistlike glow still shone.

Silently, years moved on. What period of time passed, I shall never know. It seemed to me, waiting there, that eternities came and went, stealthily; and still I watched. I could see only the glow of the sun’s edge, at times; for now, it had commenced to come and go---lighting up a while, and again becoming extinguished.

All at once, during one of these periods of life, a sudden flame cut across the night---a quick glare that lit up the dead earth, shortly; giving me a glimpse of its flat lonesomeness. The light appeared to come from the sun---shooting out from somewhere near its center, diagonally. A moment, I gazed, startled. Then the leaping flame sank, and the gloom fell again. But now it was not so dark; and the sun was belted by a thin line of vivid, white light. I stared, intently. Had a volcano broken out on the sun? Yet, I negatived the thought, as soon as formed. I felt that the light had been far too intensely white, and large, for such a cause.

Another idea there was, that suggested itself to me. It was, that one of the inner planets had fallen into the sun---becoming incandescent, under that impact. This theory appealed to me, as being more plausible, and accounting more satisfactorily for the extraordinary size and brilliance of the blaze, that had lit up the dead world, so unexpectedly.

Full of interest and emotion, I stared, across the darkness, at that line of white fire, cutting the night. One thing it told to me, unmistakably: the sun was yet rotating at an enormous speed.\footnote{I can only suppose that the time of the earth’s yearly journey had ceased to bear its present \textit{relative} proportion to the period of the sun’s rotation.---WHH} Thus, I knew that the years were still fleeting at an incalculable rate; though so far as the earth was concerned, life, and light, and time, were things belonging to a period lost in the long gone ages.

After that one burst of flame, the light had shown, only as an encircling band of bright fire. Now, however, as I watched, it began to sink into a ruddy tint, and, later, to a dark, copper-red color; much as the sun had done. Presently, it sank to a deeper hue; and, in a still further space of time, it began to fluctuate; having periods of glowing, and anon, dying. Thus, after a great while, it disappeared.

Long before this, the smoldering edge of the sun had deadened into blackness. And so, in that supremely future time, the world, dark and intensely silent, rode on its gloomy orbit around the ponderous mass of the dead sun.

My thoughts, at this period, can be scarcely described. At first, they were chaotic and wanting in coherence. But, later, as the ages came and went, my soul seemed to imbibe the very essence of the oppressive solitude and dreariness, that held the earth.

With this feeling, there came a wonderful clearness of thought, and I realized, despairingly, that the world might wander for ever, through that enormous night. For a while, the unwholesome idea filled me, with a sensation of overbearing desolation; so that I could have cried like a child. In time, however, this feeling grew, almost insensibly, less, and an unreasoning hope possessed me. Patiently, I waited.

From time to time, the noise of dropping particles, behind in the room, came dully to my ears. Once, I heard a loud crash, and turned, instinctively, to look; forgetting, for the moment, the impenetrable night in which every detail was submerged. In a while, my gaze sought the heavens; turning, unconsciously, toward the North. Yes, the nebulous glow still showed. Indeed, I could have almost imagined that it looked somewhat plainer. For a long time, I kept my gaze fixed upon it; feeling, in my lonely soul, that its soft haze was, in some way, a tie with the past. Strange, the trifles from which one can suck comfort! And yet, had I but known---But I shall come to that in its proper time.

For a very long space, I watched, without experiencing any of the desire for sleep, that would so soon have visited me in the old-earth days. How I should have welcomed it; if only to have passed the time, away from my perplexities and thoughts.

Several times, the comfortless sound of some great piece of masonry falling, disturbed my meditations; and, once, it seemed I could hear whispering in the room, behind me. Yet it was utterly useless to try to see anything. Such blackness, as existed, scarcely can be conceived. It was palpable, and hideously brutal to the sense; as though something dead, pressed up against me---something soft, and icily cold.

Under all this, there grew up within my mind, a great and overwhelming distress of uneasiness, that left me, but to drop me into an uncomfortable brooding. I felt that I must fight against it; and, presently, hoping to distract my thoughts, I turned to the window, and looked up toward the North, in search of the nebulous whiteness, which, still, I believed to be the far and misty glowing of the universe we had left. Even as I raised my eyes, I was thrilled with a feeling of wonder; for, now, the hazy light had resolved into a single, great star, of vivid green.

As I stared, astonished, the thought flashed into my mind; that the earth must be traveling toward the star; not away, as I had imagined. Next, that it could not be the universe the earth had left; but, possibly, an outlying star, belonging to some vast star-cluster, hidden in the enormous depths of space. With a sense of commingled awe and curiosity, I watched it, wondering what new thing was to be revealed to me.

For a while, vague thoughts and speculations occupied me, during which my gaze dwelt insatiably upon that one spot of light, in the otherwise pitlike darkness. Hope grew up within me, banishing the oppression of despair, that had seemed to stifle me. Wherever the earth was traveling, it was, at least, going once more toward the realms of light. Light! One must spend an eternity wrapped in soundless night, to understand the full horror of being without it.

Slowly, but surely, the star grew upon my vision, until, in time, it shone as brightly as had the planet Jupiter, in the old-earth days. With increased size, its color became more impressive; reminding me of a huge emerald, scintillating rays of fire across the world.

Years fled away in silence, and the green star grew into a great splash of flame in the sky. A little later, I saw a thing that filled me with amazement. It was the ghostly outline of a vast crescent, in the night; a gigantic new moon, seeming to be growing out of the surrounding gloom. Utterly bemused, I stared at it. It appeared to be quite close---comparatively; and I puzzled to understand how the earth had come so near to it, without my having seen it before.

The light, thrown by the star, grew stronger; and, presently, I was aware that it was possible to see the earthscape again; though indistinctly. Awhile, I stared, trying to make out whether I could distinguish any detail of the world’s surface, but I found the light insufficient. In a little, I gave up the attempt, and glanced once more toward the star. Even in the short space, that my attention had been diverted, it had increased considerably, and seemed now,\linebreak to my bewildered sight, about a quarter of the size of the full moon. The light it threw, was extraordinarily powerful; yet its color was so abominably unfamiliar, that such of the world as I could see, showed unreal; more as though I looked out upon a landscape of shadow, than aught else.

All this time, the great crescent was increasing in brightness, and began, now, to shine with a perceptible shade of green. Steadily, the star increased in size and brilliancy, until it showed, fully as large as half a full moon; and, as it grew greater and brighter, so did the vast crescent throw out more and more light, though of an ever deepening hue of green. Under the combined blaze of their radiances, the wilderness that stretched before me, became steadily more visible. Soon, I seemed able to stare across the whole world, which now appeared, beneath the strange light, terrible in its cold and awful, flat dreariness.

It was a little later, that my attention was drawn to the fact, that the great star of green flame, was slowly sinking out of the North, toward the East. At first, I could scarcely believe that I saw aright; but soon there could be no doubt that it was so. Gradually, it sank, and, as it fell, the vast crescent of glowing green, began to dwindle and dwindle, until it became a mere arc of light, against the livid colored sky. Later it vanished, disappearing in the self-same spot from which I had seen it slowly emerge.

By this time, the star had come to within some thirty degrees of the hidden horizon. In size it could now have rivaled the moon at its full; though, even yet, I could not distinguish its disk. This fact led me to conceive that it was, still, an extraordinary distance away; and, this being so, I knew that its size must be huge, beyond the conception of man to understand or imagine.

Suddenly, as I watched, the lower edge of the star vanished---cut by a straight, dark line. A minute---or a century---passed, and it dipped lower, until the half of it had disappeared from sight. Far away out on the great plain, I saw a monstrous shadow blotting it out, and advancing swiftly. Only a third of the star was visible now. Then, like a flash, the solution of this extraordinary phenomenon revealed itself to me. The star was sinking behind the enormous mass of the dead sun. Or rather, the sun---obedient to its attraction---was rising toward it, with the earth following in its trail.\footnote{A careful reading of the MS. suggests that, either the sun is traveling on an orbit of great eccentricity, or else that it was approaching the green star on a lessening orbit. And at this moment, I conceive it to be finally torn directly from its oblique course, by the gravitational pull of the immense star.---WHH} As these thoughts expanded in my mind, the star vanished; being completely hidden by the tremendous bulk of the sun. Over the earth there fell, once more, the brooding night.

With the darkness, came an intolerable feeling of loneliness and dread. For the first time, I thought of the Pit, and its inmates. After that, there rose in my memory the still more terrible Thing, that had haunted the shores of the Sea of Sleep, and lurked in the shadows of this old building. Where were they? I wondered---and shivered with miserable thoughts. For a time, fear held me, and I prayed, wildly and incoherently, for some ray of light with which to dispel the cold blackness that enveloped the world.

How long I waited, it is impossible to say---certainly for a very great period. Then, all at once, I saw a loom of light shine out ahead. Gradually, it became more distinct. Suddenly, a ray of vivid green, flashed across the darkness. At the same moment, I saw a thin line of livid flame, far in the night. An instant, it seemed, and it had grown into a great clot of fire; beneath which, the world lay bathed in a blaze of emerald green light. Steadily it grew, until, presently, the whole of the green star had come into sight again. But now, it could be scarcely called a star; for it had increased to vast proportions, being incomparably greater than the sun had been in the olden time.

Then, as I stared, I could see the edge of the lifeless sun, glowing like a great crescent-moon. Slowly, its lighted surface, broadened out to me, until half of its diameter was visible; and the star began to drop away on my right. Time passed, and the earth moved on, slowly traversing the tremendous face of the dead sun.\footnote{It will be noticed here that the earth was “\textit{slowly} traversing the tremendous face of the dead sun.” No explanation is given of this, and we must conclude, either that the speed of time had slowed, or else that the earth was actually progressing on its orbit at a rate, slow, when measured by existing standards. A careful study of the MS. however, leads me to conclude that the speed of time had been steadily decreasing for a very considerable period.---WHH}

Gradually, as the earth traveled forward, the star fell still more to the right; until, at last, it shone on the back of the house, sending a flood of broken rays, in through the skeleton-like walls. Glancing upward, I saw that much of the ceiling had vanished, enabling me to see that the upper storeys were even more decayed. The roof had, evidently, gone entirely; and I could see the green effulgence of the Starlight shining in, slantingly.

\clearpage

% File borderland-13.txt
% Version 2017/09/18
% In the original, this was Chapter XIX.

\clearpage
\label{ch:13}

\begin{ChapterStart}
\null\null
\ChapterTitle{13. The End of the Solar System}
\end{ChapterStart}

From the abutment, where once had been the windows, through which I had watched that first, fatal dawn, I could see that the sun was hugely greater, than it had been, when first the Star lit the world. So great was it, that its lower edge seemed almost to touch the far horizon. Even as I watched, I imagined that it drew closer. The radiance of green that lit the frozen earth, grew steadily brighter.

Thus, for a long space, things were. Then, on a sudden, I saw that the sun was changing shape, and growing smaller, just as the moon would have done in past time. In a while, only a third of the illuminated part was turned toward the earth. The Star bore away on the left.

Gradually, as the world moved on, the Star shone upon the front of the house, once more; while the sun showed, only as a great bow of green fire. An instant, it seemed, and the sun had vanished. The Star was still fully visible. Then the earth moved into the black shadow of the sun, and all was night---Night, black, starless, and intolerable.

Filled with tumultuous thoughts, I watched across the night---waiting. Years, it may have been, and then, in the dark house behind me, the clotted stillness of the world was broken. I leapt from the window, out on to the frozen world. I have a confused notion of having run awhile; and, after that, I just waited---waited. Time moved onward. I was conscious of little, save a sensation of cold and hopelessness and fear.

An age, it seemed, and there came a glow, that told of the coming light. It grew, tardily. Then---with a loom of unearthly glory---the first ray from the Green Star, struck over the edge of the dark sun, and lit the world. It fell upon a great, ruined structure, some two hundred yards away. It was the house.

The world moved out into the light of the Star, and I saw that, now, it seemed to stretch across a quarter of the heavens. The glory of its livid light was so tremendous, that it appeared to fill the sky with quivering flames. Then, I saw the sun. It was so close that half of its diameter lay below the horizon; and, as the world circled across its face, it seemed to tower right up into the sky, a stupendous dome of emerald colored fire.

Years appeared to pass, slowly. The earth had almost reached the center of the sun’s disk. The light from the Green \textit{Sun}---as now it must be called---shone through the interstices, that gapped the mouldered walls of the old house, giving them the appearance of being wrapped in green flames.

Suddenly, up from the center of the roofless house, shot a vast column of blood-red flame. I saw the little, twisted towers and turrets flash into fire; yet still preserving their twisted crookedness. The beams of the Green Sun, beat upon the house, and intermingled with its glows; so that it appeared a furnace of red and green fire.

Fascinated, I watched, until an overwhelming sense of coming danger, drew my attention. I glanced up, and, at once, it was borne upon me, that the sun was closer; so close, in fact, that it seemed to overhang the world. Then---I know not how---I was caught up into strange heights---floating like a bubble in the awful effulgence.

Far below me, I saw the earth, with the burning house leaping into an ever growing mountain of flame, ’round about it, the ground appeared to be glowing; and, in places, heavy wreaths of yellow smoke ascended from the earth. It seemed as though the world were becoming ignited from that one plague-spot of fire. Then the ground seemed to cave in, suddenly, and the house, disappeared into the depths of the earth, sending a strange, blood colored cloud into the heights. I remembered the hell Pit under the house.

In a while, I looked ’round. The huge bulk of the sun, rose high above me. The distance between it and the earth, grew rapidly less. Suddenly, the earth appeared to shoot forward. In a moment, it had traversed the space between it and the sun. I heard no sound; but, out from the sun’s face, gushed an ever-growing tongue of dazzling flame. It seemed to leap, almost to the distant Green Sun---shearing through the emerald light, a very cataract of blinding fire. It reached its limit, and sank; and, on the sun, glowed a vast splash of burning white---the grave of the earth.

The sun was very close to me, now. Presently, I found that I was rising higher; until, at last, I rode above it, in the emptiness. The Green Sun was now so huge that its breadth seemed to fill up all the sky, ahead. I looked down, and noted that the sun was passing directly beneath me.

A year may have gone by---or a century---and I was left, suspended, alone. The sun showed far in front---a black, circular mass, against the molten splendor of the great, Green Orb. Near one edge, I observed that a lurid glow had appeared, marking the place where the earth had fallen. By this, I knew that the long-dead sun was still revolving, though with great slowness.

Afar to my right, I seemed to catch, at times, a faint glow of whitish light. For a great time, I was uncertain whether to put this down to fancy or not. Thus, for a while, I stared, with fresh wonderings; until, at last, I knew that it was no imaginary thing; but a reality. It grew brighter; and, presently, there slid out of the green, a pale globe of softest white. It came nearer, and I saw that it was apparently surrounded by a robe of gently glowing clouds. Time passed....

I glanced toward the diminishing sun. It showed, only as a dark blot on the face of the Green Sun. As I watched, I saw it grow smaller, steadily, as though rushing toward the superior orb, at an immense speed. Intently, I stared. What would happen? I was conscious of extraordinary emotions, as I realized that it would strike the Green Sun. It grew no bigger than a pea, and I looked, with my whole soul, to witness the final end of our System---that system which had borne the world through so many aeons, with its multitudinous sorrows and joys; and now---

Suddenly, something crossed my vision, cutting from sight all vestige of the spectacle I watched with such soul-interest. What happened to the dead sun, I did not see; but I have no reason---in the light of that which I saw afterward---to disbelieve that it fell into the strange fire of the Green Sun, and so perished.

And then, suddenly, an extraordinary question rose in my mind, whether this stupendous globe of green fire might not be the vast Central Sun---the great sun, ’round which our universe and countless others revolve. I felt confused. I thought of the probable end of the dead sun, and another suggestion came, dumbly---Do the dead stars make the Green Sun their grave? The idea appealed to me with no sense of grotesqueness; but rather as something both possible and probable.

\clearpage

% File borderland-14.txt
% Version 2017/09/18
% In the original, this was Chapter XX.

\clearpage
\label{ch:14}

\begin{ChapterStart}
\null\null
\ChapterTitle{14. The Celestial Globes}
\end{ChapterStart}

For a while, many thoughts crowded my mind, so that I was unable to do aught, save stare, blindly, before me. I seemed whelmed in a sea of doubt and wonder and sorrowful remembrance.

It was later, that I came out of my bewilderment. I looked about, dazedly. Thus, I saw so extraordinary a sight that, for a while, I could scarcely believe I was not still wrapped in the visionary tumult of my own thoughts. Out of the reigning green, had grown a boundless river of softly shimmering globes---each one enfolded in a wondrous fleece of pure cloud. They reached, both above and below me, to an unknown distance; and, not only hid the shining of the Green Sun; but supplied, in place thereof, a tender glow of light, that suffused itself around me, like unto nothing I have ever seen, before or since.

In a little, I noticed that there was about these spheres, a sort of transparency, almost as though they were formed of clouded crystal, within which burned a radiance---gentle and subdued. They moved on, past me, continually, floating onward at no great speed; but rather as though they had eternity before them. A great while, I watched, and could perceive no end to them. At times, I seemed to distinguish faces, amid the cloudiness; but strangely indistinct, as though partly real, and partly formed of the mistiness through which they showed.

For a long time, I waited, passively, with a sense of growing content. I had no longer that feeling of unutterable loneliness; but felt, rather, that I was less alone, than I had been for kalpas of years. This feeling of contentment, increased, so that I would have been satisfied to float in company with those celestial globules, forever.

Ages slipped by, and I saw the shadowy faces, with increased frequency, also with greater plainness. Whether this was due to my soul having become more attuned to its surroundings, I cannot tell---probably it was so. But, however this may be, I am assured now, only of the fact that I became steadily more conscious of a new mystery about me, telling me that I had, indeed, penetrated within the borderland of some unthought-of region---some subtle, intangible place, or form, of existence.

The enormous stream of luminous spheres continued to pass me, at an unvarying rate---countless millions; and still they came, showing no signs of ending, nor even diminishing.

Then, as I was borne, silently, upon the ether, I felt a sudden, irresistible, forward movement, toward one of the passing globes. An instant, and I was beside it. Then, I slid through, into the interior, without experiencing the least resistance, of any description. For a short while, I could see nothing; and waited, curiously.

All at once, I became aware that a sound broke the inconceivable stillness. It was like the murmur of a great sea at calm---a sea breathing in its sleep. Gradually, the mist that obscured my sight, began to thin away; and so, in time, my vision dwelt once again upon the silent surface of the Sea of Sleep.

For a little, I gazed, and could scarcely believe I saw aright. I glanced ’round. There was the great globe of pale fire, swimming, as I had seen it before, a short distance above the dim horizon. To my left, far across the sea, I discovered, presently, a faint line, as of thin haze, which I guessed to be the shore, where my Love and I had met, during those wonderful periods of soul-wandering, that had been granted to me in the old earth days.

A troubled, memory came to me---of the Formless Thing that had haunted the shores of the Sea of Sleep. The guardian of that silent, echoless place. These, and other, details, I remembered, and knew, without doubt that I was looking out upon that same sea. I was filled with an overwhelming feeling of surprise, and joy, and shaken expectancy, conceiving it possible that I was about to see my Love, again. Intently, I gazed around; but could catch no sight of her. At that, for a little, I felt hopeless. Fervently, I prayed, and ever peered, anxiously.... How still was the sea!

Down, far beneath me, I could see the many trails of changeful fire, that had drawn my attention, formerly. Vaguely, I wondered what caused them; also, I remembered that I had intended to ask my dear One about them, as well as many other matters---and I had been forced to leave her, before the half that I had wished to say, was said.

My thoughts came back with a leap. I was conscious that something had touched me. I turned quickly. God, Thou wert indeed gracious---it was She! She looked up into my eyes, with an eager longing, and I looked down to her, with all my soul. I should like to have held her; but the glorious purity of her face, kept me afar. Then, out of the winding mist, she put her dear arms. Her whisper came to me, soft as the rustle of a passing cloud. ‘Dearest!’ she said. That was all; but I had heard, and, in a moment I held her to me---as I prayed---forever.

In a little, she spoke of many things, and I listened. Willingly, would I have done so through all the ages that are to come. At times, I whispered back, and my whispers brought to her spirit face, once more, an indescribably delicate tint---the bloom of love. Later, I spoke more freely, and to each word she listened, and made answer, delightfully; so that, already, I was in Paradise.

She and I; and nothing, save the silent, spacious void to see us; and only the quiet waters of the Sea of Sleep to hear us.

Long before, the floating multitude of cloud-enfolded spheres had vanished into nothingness. Thus, we looked upon the face of the slumberous deeps, and were alone. Alone, God, I would be thus alone in the hereafter, and yet be never lonely! I had her, and, greater than this, she had me. Aye, aeon-aged me; and on this thought, and some others, I hope to exist through the few remaining years that may yet lie between us.

\clearpage

% File borderland-15.txt
% Version 2017/09/18
% In the original, this was Chapter XXI.

\clearpage
\label{ch:15}

\begin{ChapterStart}
\null\null
\ChapterTitle{15. The Dark Sun}
\end{ChapterStart}

I was waked from my happiness, by a diminution of the pale and gentle light that lit the Sea of Sleep. I turned toward the huge, white orb, with a premonition of trouble. One side of it was curving inward, as though a convex, black shadow were sweeping across it. It was thus, that the darkness had come, before our last parting. I turned toward my Love, inquiringly. I noticed how wan and unreal she had grown, even in that brief space. Her voice seemed to come to me from a distance. The touch of her hands was no more than the gentle pressure of a summer wind, and grew less perceptible.

Already, quite half of the immense globe was shrouded. A feeling of desperation seized me. Was she about to leave me? Would she have to go, as she had gone before? I questioned her, anxiously, frightenedly; and she, nestling closer, explained, in that strange, faraway voice, that it was imperative she should leave me, before the Sun of Darkness---as she termed it---blotted out the light. At this confirmation of my fears, I was overcome with despair; and could only look, voicelessly, across the quiet plains of the silent sea.

At last, only a crescent of pale fire, lit the, now dim, Sea of Sleep. All this while, she had held me; but, with so soft a caress, that I had been scarcely conscious of it. We waited there, together, she and I; speechless, for very sorrow. In the dimming light, her face showed, shadowy---blending into the dusky mistiness that encircled us.

Then, when a thin, curved line of soft light was all that lit the sea, she released me---pushing me from her, tenderly. Her voice sounded, ‘I may not stay longer, Dear One.’ It ended in a sob.

She seemed to float away from me, and became invisible. Her voice came to me, out of the shadows, faintly; apparently from a great distance:---

‘A little while---’ It died away, remotely. In a breath, the Sea of Sleep darkened into night. Far to my left, I seemed to see, for a brief instant, a soft glow. It vanished, and, in the same moment, I became aware that I was no longer above the still sea; but once more suspended in infinite space, with the Green Sun---now eclipsed by a vast, dark sphere---before me.

Utterly bewildered, I stared, almost unseeingly, at the ring of green flames, leaping above the dark edge. Even in the chaos of my thoughts, I wondered, dully, at their extraordinary shapes. A multitude of questions assailed me. I thought more of her, I had so lately seen, than of the sight before me. My grief, and thoughts of the future, filled me. Was I doomed to be separated from her, always? Even in the old earth-days, she had been mine, only for a little while; then she had left me, as I thought, forever. Since then, I had seen her but these times, upon the Sea of Sleep.

A feeling of fierce resentment filled me, and miserable questionings. Why could I not have gone with my Love? What reason to keep us apart? Why had I to wait alone, while she slumbered through the years, on the still bosom of the Sea of Sleep? The Sea of Sleep! My thoughts turned, inconsequently, out of their channel of bitterness, to fresh, desperate questionings. Where was it? Where was it? I seemed to have but just parted from my Love, upon its quiet surface, and it had gone, utterly. It could not be far away! And the White Orb which I had seen hidden in the shadow of the Sun of Darkness! My sight dwelt upon the Green Sun---eclipsed. What had eclipsed it? Was there a vast, dead star circling it? Was the \textit{Central} Sun---as I had come to regard it---a double star? The thought had come, almost unbidden; yet why should it not be so?

My thoughts went back to the White Orb. Strange, that it should have been---I stopped. An idea had come, suddenly. The White Orb and the Green Sun! Were they one and the same? My imagination wandered backward, and I remembered the luminous globe to which I had been so unaccountably attracted. It was curious that I should have forgotten it, even momentarily. Where were the others? I reverted again to the globe I had entered. I thought, for a time, and matters became clearer. I conceived that, by entering that impalpable globule, I had passed, at once, into some further, and, until then, invisible dimension; There, the Green Sun was still visible; but as a stupendous sphere of pale, white light---almost as though its ghost showed, and not its material part.

A long time, I mused on the subject. I remembered how, on entering the sphere, I had, immediately, lost all sight of the others. For a still further period, I continued to revolve the different details in my mind.

In a while, my thoughts turned to other things. I came more into the present, and began to look about me, seeingly. For the first time, I perceived that innumerable rays, of a subtle, violet hue, pierced the strange semi-darkness, in all directions. They radiated from the fiery rim of the Green Sun. They seemed to grow upon my vision, so that, in a little, I saw that they were countless. The night was filled with them---spreading outward from the Green Sun, fan-wise. I concluded that I was enabled to see them, by reason of the Sun’s glory being cut off by the eclipse. They reached right out into space, and vanished.

Gradually, as I looked, I became aware that fine points of intensely brilliant light, traversed the rays. Many of them seemed to travel from the Green Sun, into distance. Others came out of the void, toward the Sun; but one and all, each kept strictly to the ray in which it traveled. Their speed was inconceivably great; and it was only when they neared the Green Sun, or as they left it, that I could see them as separate specks of light. Further from the sun, they became thin lines of vivid fire within the violet.

The discovery of these rays, and the moving sparks, interested me, extraordinarily. To where did they lead, in such countless profusion? I thought of the worlds in space.... And those sparks! Messengers! Possibly, the idea was fantastic; but I was not conscious of its being so. Messengers! Messengers from the Central Sun!

An idea evolved itself, slowly. Was the Green Sun the abode of some vast Intelligence? The thought was bewildering. Visions of the Unnameable rose, vaguely. Had I, indeed, come upon the dwelling-place of the Eternal? For a time, I repelled the thought, dumbly. It was too stupendous. Yet....

Huge, vague thoughts had birth within me. I felt, suddenly, terribly naked. And an awful Nearness, shook me.

And Heaven ...! Was that an illusion?

My thoughts came and went, erratically. The Sea of Sleep---and she! Heaven.... I came back, with a bound, to the present. Somewhere, out of the void behind me, there rushed an immense, dark body---huge and silent. It was a dead star, hurling onward to the burying place of the stars. It drove between me and the Central Suns---blotting them out from my vision, and plunging me into an impenetrable night.

An age, and I saw again the violet rays. A great while later---aeons it must have been---a circular glow grew in the sky, ahead, and I saw the edge of the receding star, show darkly against it. Thus, I knew that it was nearing the Central Suns. Presently, I saw the bright ring of the Green Sun, show plainly against the night The star had passed into the shadow of the Dead Sun. After that, I just waited. The strange years went slowly, and ever, I watched, intently.

The thing I had expected, came at last---suddenly, awfully. A vast flare of dazzling light. A streaming burst of white flame across the dark void. For an indefinite while, it soared outward---a gigantic mushroom of fire. It ceased to grow. Then, as time went by, it began to sink backward, slowly. I saw, now, that it came from a huge, glowing spot near the center of the Dark Sun. Mighty flames, still soared outward from this. Yet, spite of its size, the grave of the star was no more than the shining of Jupiter upon the face of an ocean, when compared with the inconceivable mass of the Dead Sun.

I may remark here, once more, that no words will ever convey to the imagination, the enormous bulk of the two Central Suns.

\clearpage

% File borderland-16.txt
% Version 2017/09/18
% In the original, this was Chapter XXII.

\clearpage
\label{ch:16}

\begin{ChapterStart}
\null\null
\ChapterTitle{16. The Dark Nebula}
\end{ChapterStart}

Years melted into the past, centuries, aeons. The light of the incandescent star, sank to a furious red.

It was later, that I saw the dark nebula---at first, an impalpable cloud, away to my right. It grew, steadily, to a clot of blackness in the night. How long I watched, it is impossible to say; for time, as we count it, was a thing of the past. It came closer, a shapeless monstrosity of darkness---tremendous. It seemed to slip across the night, sleepily---a very hell-fog. Slowly, it slid nearer, and passed into the void, between me and the Central Suns. It was as though a curtain had been drawn before my vision. A strange tremor of fear took me, and a fresh sense of wonder.

The green twilight that had reigned for so many millions of years, had now given place to impenetrable gloom. Motionless, I peered about me. A century fled, and it seemed to me that I detected occasional dull glows of red, passing me at intervals.

Earnestly, I gazed, and, presently, seemed to see circular masses, that showed muddily red, within the clouded blackness. They appeared to be growing out of the nebulous murk. Awhile, and they became plainer to my accustomed vision. I could see them, now, with a fair amount of distinctness---ruddy-tinged spheres, similar, in size, to the luminous globes that I had seen, so long previously.

They floated past me, continually. Gradually, a peculiar uneasiness seized me. I became aware of a growing feeling of repugnance and dread. It was directed against those passing orbs, and seemed born of intuitive knowledge, rather than of any real cause or reason.

Some of the passing globes were brighter than others; and, it was from one of these, that a face looked, suddenly. A face, human in its outline; but so tortured with woe, that I stared, aghast. I had not thought there was such sorrow, as I saw there. I was conscious of an added sense of pain, on perceiving that the eyes, which glared so wildly, were sightless. A while longer, I saw it; then it had passed on, into the surrounding gloom. After this, I saw others---all wearing that look of hopeless sorrow; and blind.

A long time went by, and I became aware that I was nearer to the orbs, than I had been. At this, I grew uneasy; though I was less in fear of those strange globules, than I had been, before seeing their sorrowful inhabitants; for sympathy had tempered my fear.

Later, there was no doubt but that I was being carried closer to the red spheres, and, presently, I floated among them. In awhile, I perceived one bearing down upon me. I was helpless to move from its path. In a minute, it seemed, it was upon me, and I was submerged in a deep red mist. This cleared, and I stared, confusedly, across the immense breadth of the Plain of Silence. It appeared just as I had first seen it. I was moving forward, steadily, across its surface. Away ahead, shone the vast, blood-red ring that lit the place.\footnote{Without doubt, the flame-edged mass of the Dead Central Sun, seen from another dimension.---\allsmcp{WHH}.} All around, was spread the extraordinary desolation of stillness, that had so impressed me during my previous wanderings across its starkness.

Presently, I saw, rising up into the ruddy gloom, the distant peaks of the mighty amphitheatre of mountains, where, untold ages before, I had been shown my first glimpse of the terrors that underlie many things; and where, vast and silent, watched by a thousand mute gods, stands the replica of this house of mysteries---this house that I had seen swallowed up in that hell-fire, ere the earth had kissed the sun, and vanished for ever.

Though I could see the crests of the mountain-amphitheatre, yet it was a great while before their lower portions became visible. Possibly, this was due to the strange, ruddy haze, that seemed to cling to the surface of the Plain. However, be this as it may, I saw them at last.

In a still further space of time, I had come so close to the mountains, that they appeared to overhang me. Presently, I saw the great rift, open before me, and I drifted into it; without volition on my part.

Later, I came out upon the breadth of the enormous arena. There, at an apparent distance of some five miles, stood the House, huge, monstrous and silent---lying in the very center of that stupendous amphitheatre. So far as I could see, it had not altered in any way; but looked as though it were only yesterday that I had seen it. Around, the grim, dark mountains frowned down upon me from their lofty silences.

Far to my right, away up among inaccessible peaks, loomed the enormous bulk of the great Beast-god. Higher, I saw the hideous form of the dread goddess, rising up through the red gloom, thousands of fathoms above me. To the left, I made out the monstrous Eyeless-Thing, grey and inscrutable. Further off, reclining on its lofty ledge, the livid Ghoul-Shape showed---a splash of sinister color, among the dark mountains.

Slowly, I moved out across the great arena---floating. As I went, I made out the dim forms of many of the other lurking Horrors that peopled those supreme heights.

Gradually, I neared the House, and my thoughts flashed back across the abyss of years. I remembered the dread Specter of the Place. A short while passed, and I saw that I was being wafted directly toward the enormous mass of that silent building.

About this time, I became aware, in an indifferent sort of way, of a growing sense of numbness, that robbed me of the fear, which I should otherwise have felt, on approaching that awesome Pile. As it was, I viewed it, calmly---much as a man views calamity through the haze of his tobacco smoke.

In a little while, I had come so close to the House, as to be able to distinguish many of the details about it. The longer I looked, the more was I confirmed in my long-ago impressions of its entire similitude to this strange house. Save in its enormous size, I could find nothing unlike.

With a curious inconsequence, my thoughts abruptly left the matter; to dwell, wonderingly, upon the peculiar material, out of which the House was constructed. It was---as I have mentioned, earlier---of a deep, green color. Yet, now that I had come so close to it, I perceived that it fluctuated at times, though slightly---glowing and fading, much as do the fumes of phosphorus, when rubbed upon the hand, in the dark.

Presently, my attention was distracted from this, by coming to the great entrance. Here, for the first time, I was afraid; for, all in a moment, the huge doors swung back, and I drifted in between them, helplessly. Inside, all was blackness, impalpable. In an instant, I had crossed the threshold, and the great doors closed, silently, shutting me in that lightless place.

Then a door opened somewhere ahead; a white haze of light filtered through, and I floated slowly into a room, that seemed strangely familiar. All at once, there came a bewildering, screaming noise, that deafened me. I saw a blurred vista of visions, flaming before my sight. My senses were dazed, through the space of an eternal moment. Then, my power of seeing, came back to me. The dizzy, hazy feeling passed, and I saw, clearly.

\clearpage

% File borderland-17.txt
% Version 2017/09/18
% In the original, this was Chapter XXIII.

\clearpage
\label{ch:17}

\begin{ChapterStart}
\null\null
\ChapterTitle{17. Pepper}
\end{ChapterStart}

I was seated in my chair, back again in this old study. My glance wandered ’round the room. For a minute, it had a strange, quivery appearance---unreal and unsubstantial. This disappeared, and I saw that nothing was altered in any way. I looked toward the end window---the blind was up.

With a succession of efforts, I trod my way to the window, and looked out. The sun was just rising, lighting up the tangled wilderness of gardens. For, perhaps, a minute, I stood, and stared. I passed my hand, confusedly, across my forehead.

Presently, amid the chaos of my senses, a sudden thought came to me; I turned, quickly, and called to Pepper. There was no answer, and I stumbled across the room, in a quick access of fear. As I went, I tried to frame his name; but my lips were numb. I reached the table, and stooped down to him, with a catching at my heart. He was lying in the shadow of the table, and I had not been able to see him, distinctly, from the window. Now, as I stooped, I took my breath, shortly. There was no Pepper; instead, I was reaching toward an elongated, little heap of grey, ashlike dust....

I must have remained, in that half-stooped position, for some minutes. I was dazed---stunned. Pepper had really passed into the land of shadows.

\clearpage

% File borderland-18.txt
% Version 2017/09/18
% In the original, this was Chapter XXIV.

\clearpage
\label{ch:18}

\begin{ChapterStart}
\null\null
\ChapterTitle{18. The Footsteps in the Garden}
\end{ChapterStart}

Pepper is dead! Even now, at times, I seem scarcely able to realize that this is so. It is many weeks, since I came back from that strange and terrible journey through space and time. Sometimes, in my sleep, I dream about it, and go through, in imagination, the whole of that fearsome happening. When I wake, my thoughts dwell upon it. That Sun---those Suns, were they indeed the great Central Suns, ’round which the whole universe, of the unknown heavens, revolves? Who shall say? And the bright globules, floating forever in the light of the Green Sun! And the Sea of Sleep on which they float! How unbelievable it all is. If it were not for Pepper, I should, even after the many extraordinary things that I have witnessed, be inclined to imagine that it was but a gigantic dream. Then, there is that dreadful, dark nebula (with its multitudes of red spheres) moving always within the shadow of the Dark Sun, sweeping along on its stupendous orbit, wrapped eternally in gloom. And the faces that peered out at me! God, do they, and does such a thing really exist? ... There is still that little heap of grey ash, on my study floor. I will not have it touched.

Now that I am writing, let me record that I am certain, there is something horrible about to happen. Last night, a thing occurred, which has filled me with an even greater terror, than did the Pit. I will write it down now, and, if anything more happens, endeavor to make a note of it, at once. I have a feeling, that there is more in this last affair, than in all those others. I am shaky and nervous, even now, as I write. Somehow, I think death is not very far away.\linebreak Not that I fear death---as death is understood. Yet, there is that in the air, which bids me fear---an intangible, cold horror. I felt it last night. It was thus:---

Last night, I was sitting here in my study, writing. The door, leading into the garden, was half open. At times, the metallic rattle of a dog’s chain, sounded faintly. It belongs to the dog I have bought, since Pepper’s death. I will not have him in the house---not after Pepper. Still, I have felt it better to have a dog about the place. They are wonderful creatures.

I was much engrossed in my work, and the time passed, quickly. Suddenly, I heard a soft noise on the path, outside in the garden---pad, pad, pad, it went, with a stealthy, curious sound. I sat upright, with a quick movement, and looked out through the opened door. Again the noise came---pad, pad, pad. It appeared to be approaching. With a slight feeling of nervousness, I stared into the gardens; but the night hid everything.

Then the dog gave a long howl, and I started. For a minute, perhaps, I peered, intently; but could hear nothing. After a little, I picked up the pen, which I had laid down, and recommenced my work. The nervous feeling had gone; for I imagined that the sound I had heard, was nothing more than the dog walking ’round his kennel, at the length of his chain.

A quarter of an hour may have passed; then, all at once, the dog howled again, and with such a plaintively sorrowful note, that I jumped to my feet, dropping my pen, and inking the page on which I was at work.

‘Curse that dog!’ I muttered, noting what I had done. Then, even as I said the words, there sounded again that queer---pad, pad, pad. It was horribly close---almost by the door, I thought. I knew, now, that it could not be the dog; his chain would not allow him to come so near.

The dog’s growl came again, and I noted, subconsciously, the taint of fear in it.

Frightened, and puzzled, I seized a stick from the corner, and went toward the door, silently; taking one of the candles with me. I had come to within a few paces of it, when, suddenly, a peculiar sense of fear thrilled through me---a fear, palpitant and real; whence, I knew not, nor why. So great was the feeling of terror, that I wasted no time; but retreated straight-way---walking backward, and keeping my gaze, fearfully, on the door. I would have given much, to rush at it, fling it to, and shoot the bolts; for I have had it repaired and strengthened, so that, now, it is far stronger than ever it has been. I continued my, almost unconscious, progress backward, until the wall brought me up. At that, I started, nervously, and glanced ’round, apprehensively. As I did so, my eyes dwelt, momentarily, on the rack of firearms, and I took a step toward them; but stopped, with a curious feeling that they would be needless. Outside, in the gardens, the dog moaned, strangely.

Pad, pad, pad---Something passed down the garden path, and a faint, mouldy odor seemed to come in through the open door, and mingle with the burnt smell.

The dog had been silent for a few moments. Now, I heard him yowl, sharply, as though in pain. Then, he was quiet, save for an occasional, subdued whimper of fear.

A minute went by; then the gate on the West side of the gardens, slammed, distantly. After that, nothing; not even the dog’s whine.

I must have stood there some minutes. Then a fragment of courage stole into my heart, and I made a frightened rush at the door, dashed it to, and bolted it. After that, for a full half-hour, I sat, helpless---staring before me, rigidly.

Slowly, my life came back into me, and I made my way, shakily, up-stairs to bed.

\clearpage

% File borderland-19.txt
% Version 2017/09/18
% In the original, this was Chapter XXV.

\clearpage
\label{ch:19}

\begin{ChapterStart}
\null\null
\ChapterTitle{19. The Thing from the Arena}
\end{ChapterStart}

This morning, early, I went through the gardens; but found everything as usual. Near the door, I examined the path, for footprints; yet, here again, there was nothing to tell me whether, or not, I dreamed last night.

It was only when I came to speak to the dog, that I discovered tangible proof, that something did happen. When I went to his kennel, he kept inside, crouching up in one corner, and I had to coax him, to get him out. When, finally, he consented to come, it was in a cowed and subdued manner. As I patted him, my attention was attracted to a greenish patch, on his left flank. On examining it, I found, that the fur and skin had been apparently, burnt off; for the flesh showed, raw and scorched. The shape of the mark was curious, reminding me of the imprint of a large talon or hand.

I stood up, thoughtful. My gaze wandered toward the study window. The rays of the rising sun, shimmered on the smoky patch in the lower corner, causing it to fluctuate from green to red, oddly. Ah! that was undoubtedly another proof; and, suddenly, the horrible Thing I saw last night, rose in my mind. I looked at the dog, again. I knew the cause, now, of that hateful looking wound on his side---I knew, also, that, what I had seen last night, had been a real happening. And a great discomfort filled me. Pepper! And now this poor animal ...! I glanced at the dog again, and noticed that he was licking at his wound.

‘Poor brute!’ I muttered, and bent to pat his head. At that, he got upon his feet, nosing and licking my hand, wistfully.

Presently, I left him, having other matters to which to attend.

After dinner, I went to see him, again. He seemed quiet, and disinclined to leave his kennel. He has refused all food today.

The day has passed, uneventfully enough. After tea, I went, again, to have a look at the dog. He seemed moody, and somewhat restless; yet persisted in remaining in his kennel. Before locking up, for the night, I moved his kennel out, away from the wall, so that I shall be able to watch it from the small window, tonight. The thought came to me, to bring him into the house for the night; but consideration has decided me, to let him remain out. I cannot say that the house is, in any degree, less to be feared than the gardens. Pepper was in the house, and yet....

It is now two o’clock. Since eight, I have watched the kennel, from the small, side window in my study. Yet, nothing has occurred, and I am too tired to watch longer. I will go to bed....

During the night, I was restless. This is unusual for me; but, toward morning, I obtained a few hours’ sleep.

I rose early, and, after breakfast, visited the dog. He was quiet; but morose, and refused to leave his kennel. I wish there was some horse doctor near here; I would have the poor brute looked to. All day, he has taken no food; but has shown an evident desire for water---lapping it up, greedily. I was relieved to observe this.

The evening has come, and I am in my study. I intend to follow my plan of last night, and watch the kennel. The door, leading into the garden, is bolted, securely. I am consciously glad there are bars to the windows....

Night:---Midnight has gone. The dog has been silent, up to the present. Through the side window, on my left, I can make out, dimly, the outlines of the kennel. For the first time, the dog moves, and I hear the rattle of his chain. I look out, quickly. As I stare, the dog moves again, restlessly, and I see a small patch of luminous light, shine from the interior of the kennel. It vanishes; then the dog stirs again, and, once more, the gleam comes. I am puzzled. The dog is quiet, and I can see the luminous thing, plainly. It shows distinctly. There is something familiar about the shape of it. For a moment,\linebreak I wonder; then it comes to me, that it is not unlike the four fingers and thumb of a hand. Like a hand! And I remember the contour of that fearsome wound on the dog’s side. It must be the wound I see. It is luminous at night---Why? The minutes pass. My mind is filled with this fresh thing....

Suddenly, I hear a sound, out in the gardens. How it thrills through me. It is approaching. Pad, pad, pad. A prickly sensation traverses my spine, and seems to creep across my scalp. The dog moves in his kennel, and whimpers, frightenedly. He must have turned ’round; for, now, I can no longer see the outline of his shining wound.

Outside, the gardens are silent, once more, and I listen, fearfully. A minute passes, and another; then I hear the padding sound, again. It is quite close, and appears to be coming down the graveled path. The noise is curiously measured and deliberate. It ceases outside the door; and I rise to my feet, and stand motionless. From the door, comes a slight sound---the latch is being slowly raised. A singing noise is in my ears, and I have a sense of pressure about the head---

The latch drops, with a sharp click, into the catch. The noise startles me afresh; jarring, horribly, on my tense nerves. After that, I stand, for a long while, amid an ever-growing quietness. All at once, my knees begin to tremble, and I have to sit, quickly.

An uncertain period of time passes, and, gradually, I begin to shake off the feeling of terror, that has possessed me. Yet, still I sit. I seem to have lost the power of movement. I am strangely tired, and inclined to doze. My eyes open and close, and, presently, I find myself falling asleep, and waking, in fits and starts.

It is some time later, that I am sleepily aware that one of the candles is guttering. When I wake again, it has gone out, and the room is very dim, under the light of the one remaining flame. The semi-darkness troubles me little. I have lost that awful sense of dread, and my only desire seems to be to sleep---sleep.

Suddenly, although there is no noise, I am awake---wide awake. I am acutely conscious of the nearness of some mystery, of some overwhelming Presence. The very air seems pregnant with terror.\linebreak I sit huddled, and just listen, intently. Still, there is no sound. Nature, herself, seems dead. Then, the oppressive stillness is broken by a little eldritch scream of wind, that sweeps ’round the house, and dies away, remotely.

I let my gaze wander across the half-lighted room. By the great clock in the far corner, is a dark, tall shadow. For a short instant, I stare, frightenedly. Then, I see that it is nothing, and am, momentarily, relieved.

In the time that follows, the thought flashes through my brain, why not leave this house---this house of mystery and terror? Then, as though in answer, there sweeps up, across my sight, a vision of the wondrous Sea of Sleep,---the Sea of Sleep where she and I have been allowed to meet, after the years of separation and sorrow; and I know that I shall stay on here, whatever happens.

Through the side window, I note the somber blackness of the night. My glance wanders away, and ’round the room; resting on one shadowy object and another. Suddenly, I turn, and look at the window on my right; as I do so, I breathe quickly, and bend forward, with a frightened gaze at something outside the window, but close to the bars. I am looking at a vast, misty swine-face, over which fluctuates a flamboyant flame, of a greenish hue. It is the Thing from the arena.\footnote{Previously encountered in ch. 2.---Anon. Ed.} The quivering mouth seems to drip with a continual, phosphorescent slaver. The eyes are staring straight into the room, with an inscrutable expression. Thus, I sit rigidly---frozen.

The Thing has begun to move. It is turning, slowly, in my direction. Its face is coming ’round toward me. It sees me. Two huge, inhumanly human, eyes are looking through the dimness at me. I am cold with fear; yet, even now, I am keenly conscious, and note, in an irrelevant way, that the distant stars are blotted out by the mass of the giant face.

A fresh horror has come to me. I am rising from my chair, without the least intention. I am on my feet, and something is impelling me toward the door that leads out into the gardens.\linebreak I wish to stop; but cannot. Some immutable power is opposed to my will, and I go slowly forward, unwilling and resistant. My glance flies ’round the room, helplessly, and stops at the window. The great swine-face has disappeared, and I hear, again, that stealthy pad, pad, pad. It stops outside the door---the door toward which I am being compelled....

There succeeds a short, intense silence; then there comes a sound. It is the rattle of the latch, being slowly lifted. At that, I am filled with desperation. I will not go forward another step.\linebreak I make a vast effort to return; but it is, as though I press back, upon an invisible wall. I groan out loud, in the agony of my fear, and the sound of my voice is frightening. Again comes that rattle, and I shiver, clammily. I try---aye, fight and struggle, to hold back, \textit{back}; but it is no use....

I am at the door, and, in a mechanical way, I watch my hand go forward, to undo the topmost bolt. It does so, entirely without my volition. Even as I reach up toward the bolt, the door is violently shaken, and I get a sickly whiff of mouldy air, which seems to drive in through the interstices of the doorway. I draw the bolt back, slowly, fighting, dumbly, the while. It comes out of its socket, with a click, and I begin to shake, aguishly. There are two more; one at the bottom of the door; the other, a massive affair, is placed about the middle.

For, perhaps a minute, I stand, with my arms hanging slackly, by my sides. The influence to meddle with the fastenings of the door, seems to have gone. All at once, there comes the sudden rattle of iron, at my feet. I glance down, and realize, with an unspeakable terror, that my foot is pushing back the lower bolt. An awful sense of helplessness assails me.... The bolt comes out of its hold, with a slight, ringing sound and I stagger on my feet, grasping at the great, central bolt, for support. A minute passes, an eternity; then another------My God, help me! I am being forced to work upon the last fastening. \textit{I will not!} Better to die, than open to the Terror, that is on the other side of the door. Is there no escape ...? God help me, I have jerked the bolt half out of its socket! My lips emit a hoarse scream of terror, the bolt is three parts drawn, now, and still my unconscious hands work toward my doom. Only a fraction of steel, between my soul and That. Twice, I scream out in the supreme agony of my fear; then, with a mad effort, I tear my hands away. My eyes seem blinded. A great blackness is falling upon me. Nature has come to my rescue. I feel my knees giving. There is a loud, quick thudding upon the door, and I am falling, falling....

I must have lain there, at least a couple of hours. As I recover, I am aware that the other candle has burnt out, and the room is in an almost total darkness. I cannot rise to my feet, for I am cold, and filled with a terrible cramp. Yet my brain is clear, and there is no longer the strain of that unholy influence.

Cautiously, I get upon my knees, and feel for the central bolt. I find it, and push it securely back into its socket; then the one at the bottom of the door. By this time, I am able to rise to my feet, and so manage to secure the fastening at the top. After that, I go down upon my knees, again, and creep away among the furniture, in the direction of the stairs. By doing this, I am safe from observation from the window.

I reach the opposite door, and, as I leave the study, cast one nervous glance over my shoulder, toward the window. Out in the night, I seem to catch a glimpse of something impalpable; but it may be only a fancy. Then, I am in the passage, and on the stairs.

Reaching my bedroom, I clamber into bed, all clothed as I am, and pull the bedclothes over me. There, after awhile, I begin to regain a little confidence. It is impossible to sleep; but I am grateful for the added warmth of the bedclothes. Presently, I try to think over the happenings of the past night; but, though I cannot sleep, I find that it is useless, to attempt consecutive thought. My brain seems curiously blank.

Toward morning, I begin to toss, uneasily. I cannot rest, and, after awhile, I get out of bed, and pace the floor. The wintry dawn is beginning to creep through the windows, and shows the bare discomfort of the old room. Strange, that, through all these years, it has never occurred to me how dismal the place really is.

After a time, I go to the window, and, opening it, look out. The sun is now above the horizon, and the air, though cold, is sweet and crisp. Gradually, my brain clears, and a sense of security, for the time being, comes to me. Somewhat happier, I go down stairs, and out into the garden, to have a look at the dog.

As I approach the kennel, I am greeted by the same mouldy stench that assailed me at the door last night. Shaking off a momentary sense of fear, I call to the dog; but he takes no heed, and, after calling once more, I throw a small stone into the kennel. At this, he moves, uneasily, and I shout his name, again; but do not go closer.

In a little the poor beast rises, and shambles out lurching queerly. In the daylight he stands swaying from side to side, and blinking stupidly. I look and note that the horrid wound is larger, much larger, and seems to have a whitish, fungoid appearance. It is impossible to tell what may be the matter with him; and it is well to be cautious.

The day slips away, uneventfully; and night comes on. I have determined to repeat my experiment of last night. I cannot say that it is wisdom; yet my mind is made up. Still, however, I have taken precautions; for I have driven stout nails in at the back of each of the three bolts, that secure the door, opening from the study into the gardens. This will, at least, prevent a recurrence of the danger I ran last night.

From ten to about two-thirty, I watch; but nothing occurs; and, finally, I stumble off to bed, where I am soon asleep.

\clearpage

% File borderland-20.txt
% Version 2017/09/18
% In the original, this was Chapter XXVI.

\clearpage
\label{ch:20}

\begin{ChapterStart}
\null\null
\ChapterTitle{20. The Luminous Speck}
\end{ChapterStart}

I awake suddenly. It is still dark. I turn over, once or twice, in my endeavors to sleep again; but I cannot sleep. My head is aching, slightly; and, by turns I am hot and cold. In a little, I give up the attempt, and stretch out my hand, for the matches. I will light my candle, and read, awhile; perhaps, I shall be able to sleep, after a time. For a few moments, I grope; then my hand touches the box; but, as I open it, I am startled, to see a phosphorescent speck of fire, shining amid the darkness. I put out my other hand, and touch it. It is on my wrist. With a feeling of vague alarm, I strike a light, hurriedly, and look; but can see nothing, save a tiny scratch.

‘Fancy!’ I mutter, with a half sigh of relief. Then the match burns my finger, and I drop it, quickly. As I fumble for another, the thing shines out again. I know, now, that it is no fancy. This time, I light the candle, and examine the place, more closely. There is a slight, greenish discoloration ‘round the scratch. I am puzzled and worried. Then a thought comes to me. I remember the morning after the Thing appeared. I remember that the dog licked my hand. It was this one, with the scratch on it; though I have not been even conscious of the abasement, until now. A horrible fear has come to me. It creeps into my brain---the dog’s wound, shines at night. With a dazed feeling, I sit down on the side of the bed, and try to think; but cannot. My brain seems numbed with the sheer horror of this new fear.

Once, I rouse up, and try to persuade myself that I am mistaken; but it is no use. In my heart, I have no doubt.

Hour after hour, I sit in the darkness and silence, and shiver, hopelessly....

The day has come and gone, and it is night again.

This morning, early, I shot the dog, and buried it, away among the bushes. It is better so. The foul growth had almost hidden its left side. And I---the place on my wrist has enlarged, perceptibly. Several times, I have caught myself muttering prayers---little things learnt as a child. God, Almighty God, help me! I shall go mad.

Six days, and I have eaten nothing. It is night. I am sitting in my chair. Ah, God! I wonder have any ever felt the horror of life that I have come to know? I am swathed in terror. I feel ever the burning of this dread growth. It has covered all my right arm and side, and is beginning to creep up my neck. Tomorrow, it will eat into my face. I shall become a terrible mass of living corruption. There is no escape. Yet, a thought has come to me, born of a sight of the gun-rack, on the other side of the room. I have looked again---with the strangest of feelings. The thought grows upon me. God, Thou knowest, Thou must know, that death is better, aye, better a thousand times than This. This! Jesus, forgive me, but I cannot live, cannot, cannot! I dare not! I am beyond all help---there is nothing else left. It will, at least, spare me that final horror....

I think I must have been dozing. I am very weak, and oh! so miserable, so miserable and tired---tired. The rustle of the paper, tries my brain. My hearing seems preternaturally sharp. I will sit awhile and think....

Hush! I hear something, down---down in the cellars. It is a creaking sound. My God, it is the opening of the great, oak trap. What can be doing that? The scratching of my pen deafens me ... I must listen.... There are steps on the stairs; strange padding steps, that come up and nearer.... Jesus, be merciful to me, an old man. There is something fumbling at the door-handle. O God, help me now! Jesus---The door is opening---slowly. Somethi---

\null
{\centering\textit{...End Manuscript}\par}

\clearpage % the following material begins with \cleartorecto

% File borderland-21-conclusion.txt
% Version 2017/09/18
% The voice of the Introduction (Berreggnog) returns.
% I have added "As Told by Mr. Berreggnog" for clarity.
% This was Chapter XXVII in the original.

\cleartorecto % this must begin recto
\label{ch:conclusion}

\begin{ChapterStart}
\null\null
\ChapterTitle{Conclusion}
\null
\ChapterDeco{As Told by Mr. Berreggnog}
\end{ChapterStart}

I put down the Manuscript, and glanced across at Tonnison: he was sitting, staring out into the dark. I waited a minute; then I spoke.

“Well?” I said.

He turned, slowly, and looked at me. His thoughts seemed to have gone out of him into a great distance.

“Was he mad?” I asked, and indicated the MS., with a half nod.

Tonnison stared at me, unseeingly, a moment; then, his wits came back to him, and, suddenly, he comprehended my question.

“No!” he said.

I opened my lips, to offer a contradictory opinion; for my sense of the saneness of things, would not allow me to take the story literally; then I shut them again, without saying anything. Somehow, the certainty in Tonnison’s voice affected my doubts. I felt, all at once, less assured; though I was by no means convinced as yet.

After a few moments’ silence, Tonnison rose, stiffly, and began to undress. He seemed disinclined to talk; so I said nothing; but followed his example.

Somehow, as I rolled into my blankets, there crept into my mind a memory of the old gardens, as we had seen them. I remembered the odd fear that the place had conjured up in our hearts; and it grew upon me, with conviction, that Tonnison was right.

It was very late when we rose---nearly midday; for the greater part of the night had been spent in reading the MS.

Tonnison was grumpy, and I felt out of sorts. It was a somewhat dismal day, and there was a touch of chilliness in the air. There was no mention of going out fishing on either of our parts. We got dinner, and, after that, just sat and smoked in silence.

Presently, Tonnison asked for the Manuscript: I handed it to him, and he spent most of the afternoon in reading it through by himself.

It was while he was thus employed, that a thought came to me:---

“What do you say to having another look at---?” I nodded my head down stream.

Tonnison looked up. “Nothing!” he said, abruptly; and, somehow, I was less annoyed, than relieved, at his answer.

After that, I left him alone.

A little before teatime, he looked up at me, curiously.

“Sorry, old chap, if I was a bit short with you just now;” (just now, indeed! he had not spoken for the last three hours) “but I would not go there again,” and he indicated with his head, “for anything that you could offer me. Ugh!” and he put down that history of a man’s terror and hope and despair.

The next morning, we rose early, and went for our accustomed swim: we had partly shaken off the depression of the previous day; and so, took our rods when we had finished breakfast, and spent the day at our favorite sport.

After that day, we enjoyed our holiday to the utmost; though both of us looked forward to the time when our driver should come; for we were tremendously anxious to inquire of him, and through him among the people of the tiny hamlet, whether any of them could give us information about that strange garden, lying away by itself in the heart of an almost unknown tract of country.

At last, the day came, on which we expected the driver to come for us. He arrived early, while we were still abed; and, the first thing we knew, he was at the opening of the tent, inquiring whether we had had good sport. We replied in the affirmative; and then, almost in the same breath, we asked the question that was uppermost in our minds:---Did he know anything about an old garden, and a great ravine, and a lake, situated some miles away, down the river; also, had he ever heard of a great house thereabouts?

No, he did not, and had not; yet, stay, he had heard a rumor, once upon a time, of a great, old house standing alone out in the wilderness; but, if he remembered rightly it was a place given over to the fairies; or, if that had not been so, he was certain that there had been something “quare” about it; and, anyway, he had heard nothing of it for a very long while. No, he could not remember anything particular about it; indeed, he did not know he remembered anything “at all, at all” until we questioned him.

“Look here,” said Tonnison, finding that this was about all that he could tell us, “just take a walk ’round the village, while we dress, and find out something, if you can.”

With a nondescript salute, the man departed on his errand; while we made haste to get into our clothes; after which, we began to prepare breakfast.

We were just sitting down to it, when he returned.

“It’s all in bed the lazy divvils is, sor,” he said, with a repetition of the salute, and an appreciative eye to the good things spread out on our provision chest, which we utilized as a table.

“Oh, well, sit down,” replied my friend, “and have something to eat with us.” Which the man did without delay.

After breakfast, Tonnison sent him off again on the same errand, while we sat and smoked. He was away some three-quarters of an hour, and, when he returned, it was evident that he had found out something. He had got into conversation with an ancient man of the village, who, probably, knew more---though it was little enough---of the strange house, than any other person living.

The substance of this knowledge was, that, in the “ancient man’s” youth---and goodness knows how long back that was---there had stood a great house in the center of the gardens, where now was left only that fragment of ruin. This house had been empty for a great while; years before his---the ancient man’s---birth. It was a place shunned by the people of the village, as it had been shunned by their fathers before them. There were many things said about it, and all were of evil. No one ever went near it, either by day or night. In the village it was a synonym of all that is unholy and dreadful.

And then, one day, a man, a stranger, had ridden through the village, and turned off down the river, in the direction of the House, as it was always termed by the villagers. Some hours afterward, he had ridden back, taking the track by which he had come, toward Ardrahan. Then, for three months or so, nothing was heard. At the end of that time, he reappeared, and gone straight down the bank of the river, in the direction of the House.

Since that time, no one, save the porter chartered to bring over monthly supplies of necessaries from Ardrahan, had ever seen the stranger: and him, none had ever induced to talk; evidently, the porter had been well paid for his trouble.

The years had moved onward, uneventfully enough, in that little hamlet; the porter making his monthly journeys, regularly.

One day, the porter had appeared as usual on his customary errand. He had passed through the village without exchanging more than a surly nod with the inhabitants and gone on toward the House. Usually, it was evening before he made the return journey. On this occasion, however, he had reappeared in the village, a few hours later, in an extraordinary state of excitement, and with the astounding information, that the House had disappeared bodily, and that a stupendous pit now yawned in the place where it had stood.

This news, it appears, so excited the curiosity of the villagers, that they overcame their fears, and marched \textit{en masse} to the place. There, they found everything, just as described by the carrier.

This was all that we could learn. Of the author of the MS., who he was, and whence he came, we shall never know. His identity is, as he seems to have desired, buried forever.

That same day, we left the lonely village of Kraighten. We have never been there since.

Sometimes, in my dreams, I see that enormous pit, surrounded, as it is, on all sides by wild trees and bushes. And the noise of the water rises upward, and blends---in my sleep---with other and lower noises; while, over all, hangs the eternal shroud of spray.

\clearpage % the main TeX file follows with \cleartoend



% No back matter. Indeed, back matter is very rare in faction, or even in nonfiction.
% If this book had endnotes or Appendix, then it would be in main matter,
% with continued Arabic page numbering.


\cleartoend % Ensures final page is blank verso, possibly preceded by blank recto.


\end{document}

