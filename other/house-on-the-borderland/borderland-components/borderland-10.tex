% File borderland-10.txt
% Version 2017/09/18
% In the original, this was Chapter XVI.

\clearpage
\label{ch:10}

\begin{ChapterStart}
\null\null
\ChapterTitle{10. The Awakening}
\end{ChapterStart}

I awoke, with a start. For a moment, I wondered where I was. Then memory came to me....

The room was still lit with that strange light---half-sun, half-moon. I felt refreshed, and the tired, weary ache had left me. I went slowly to the window, and looked out. Overhead, the river of flame drove up and down, North and South, in a dancing semi-circle of fire. As a mighty sleigh in the loom of time it seemed---in a sudden fancy of mine---to be beating home the picks of the years. For, so vastly had the passage of time been accelerated, that there was no longer any sense of the sun passing from East to West. The only apparent movement was the North and South beat of the sun-stream, that had become so swift now, as to be better described as a \textit{quiver}.

As I peered out, there came to me a sudden, inconsequent memory of that last journey among the Outer worlds. I remembered the sudden vision that had come to me, as I neared the Solar System, of the fast whirling planets about the sun---as though the governing quality of time had been held in abeyance, and the Machine of a Universe allowed to run down an eternity, in a few moments or hours. The memory passed, along with a, but partially comprehended, suggestion that I had been permitted a glimpse into further time spaces. I stared out again, seemingly, at the quake of the sun-stream. The speed seemed to increase, even as I looked. Several lifetimes came and went, as I watched.

Suddenly, it struck me, with a sort of grotesque seriousness, that I was still alive. I thought of Pepper, and wondered how it was that I had not followed his fate. He had reached the time of his dying, and had passed, probably through sheer length of years. And here was I, alive, hundreds of thousands of centuries after my rightful period of years.

For, a time, I mused, absently. ‘Yesterday---’ I stopped, suddenly. Yesterday! There was no yesterday. The yesterday of which I spoke had been swallowed up in the abyss of years, ages gone. I grew dazed with much thinking.

Presently, I turned from the window, and glanced ’round the room. It seemed different---strangely, utterly different. Then, I knew what it was that made it appear so strange. It was bare: there was not a piece of furniture in the room; not even a solitary fitting of any sort. Gradually, my amazement went, as I remembered, that this was but the inevitable end of that process of decay, which I had witnessed commencing, before my sleep. Millions of years!

Over the floor was spread a deep layer of dust, that reached half way up to the window-seat. It had grown immeasurably, whilst I slept; and represented the dust of untold ages. Undoubtedly, atoms of the old, decayed furniture helped to swell its bulk; and, somewhere among it all, mouldered the long-ago-dead Pepper.

All at once, it occurred to me, that I had no recollection of wading knee-deep through all that dust, after I awoke. True, an incredible age of years had passed, since I approached the window; but that was evidently as nothing, compared with the countless spaces of time that, I conceived, had vanished whilst I was sleeping. I remembered now, that I had fallen asleep, sitting in my old chair. Had it gone ...? I glanced toward where it had stood. Of course, there was no chair to be seen. I could not satisfy myself, whether it had disappeared, after my waking, or before. If it had mouldered under me, surely, I should have been waked by the collapse. Then I remembered that the thick dust, which covered the floor, would have been sufficient to soften my fall; so that it was quite possible, I had slept upon the dust for a million years or more.

As these thoughts wandered through my brain, I glanced again, casually, to where the chair had stood. Then, for the first time,\linebreak I noticed that there were no marks, in the dust, of my footprints, between it and the window. But then, ages of years had passed, since I had awaked---tens of thousands of years!

My look rested thoughtfully, again upon the place where once had stood my chair. Suddenly, I passed from abstraction to intentness; for there, in its standing place, I made out a long undulation, rounded off with the heavy dust. Yet it was not so much hidden, but that I could tell what had caused it. I knew---and shivered at the knowledge---that it was a human body, ages-dead, lying there, beneath the place where I had slept. It was lying on its right side, its back turned toward me. I could make out and trace each curve and outline, softened, and moulded, as it were, in the black dust. In a vague sort of way, I tried to account for its presence there. Slowly, I began to grow bewildered, as the thought came to me that it lay just about where I must have fallen when the chair collapsed.

Gradually, an idea began to form itself within my brain; a thought that shook my spirit. It seemed hideous and insupportable; yet it grew upon me, steadily, until it became a conviction. The body under that coating, that shroud of dust, was neither more nor less than my own dead shell. I did not attempt to prove it. I knew it now, and wondered I had not known it all along. I was a bodiless thing.

Awhile, I stood, trying to adjust my thoughts to this new problem. In time---how many thousands of years, I know not---I attained to some degree of quietude---sufficient to enable me to pay attention to what was transpiring around me.

Now, I saw that the elongated mound had sunk, collapsed, level with the rest of the spreading dust. And fresh atoms, impalpable, had settled above that mixture of grave-powder, which the aeons had ground. A long while, I stood, turned from the window. Gradually, I grew more collected, while the world slipped across the centuries into the future.

Presently, I began a survey of the room. Now, I saw that time was beginning its destructive work, even on this strange old building. That it had stood through all the years was, it seemed to me, proof that it was something different from any other house. I do not think, somehow, that I had thought of its decaying. It was not until I had meditated upon the matter, for some considerable time, that I fully realized that the extraordinary space of time through which it had stood, was sufficient to have utterly pulverized the very stones of which it was built, had they been taken from any earthly quarry. Yes, it was undoubtedly mouldering now. All the plaster had gone from the walls; even as the woodwork of the room had gone, many ages before.

While I stood, in contemplation, a piece of glass, from one of the small, diamond-shaped panes, dropped, with a dull tap, amid the dust upon the sill behind me, and crumbled into a little heap of powder. As I turned from contemplating it, I saw light between a couple of the stones that formed the outer wall. Evidently, the mortar was falling away....

After awhile, I turned once more to the window, and peered out. I discovered, now, that the speed of time had become enormous. The lateral quiver of the sun-stream, had grown so swift as to cause the dancing semi-circle of flame to merge into, and disappear in, a sheet of fire that covered half the Southern sky from East to West.

From the sky, I glanced down to the gardens. They were just a blur of a palish, dirty green. I had a feeling that they stood higher, than in the old days; a feeling that they were nearer my window, as though they had risen, bodily. Yet, they were still a long way below me; for the rock, over the mouth of the pit, on which this house stands, arches up to a great height.

It was later, that I noticed a change in the constant color of the gardens. The pale, dirty green was growing ever paler and paler, toward white. At last, after a great space, they became greyish-white, and stayed thus for a very long time. Finally, however, the greyness began to fade, even as had the green, into a dead white. And this remained, constant and unchanged. And by this I knew that, at last, snow lay upon all the Northern world.

And so, by millions of years, time winged onward through eternity, to the end---the end, of which, in the old-earth days, I had thought remotely, and in hazily speculative fashion. And now, it was approaching in a manner of which none had ever dreamed.

I recollect that, about this time, I began to have a lively, though morbid, curiosity, as to what would happen when the end came---but I seemed strangely without imaginings.

All this while, the steady process of decay was continuing. The few remaining pieces of glass, had long ago vanished; and, every now and then, a soft thud, and a little cloud of rising dust, would tell of some fragment of fallen mortar or stone.

I looked up again, to the fiery sheet that quaked in the heavens above me and far down into the Southern sky. As I looked, the impression was borne in upon me, that it had lost some of its first brilliancy---that it was duller, deeper hued.

I glanced down, once more, to the blurred white of the worldscape. Sometimes, my look returned to the burning sheet of dulling flame, that was, and yet hid, the sun. At times, I glanced behind me, into the growing dusk of the great, silent room, with its aeon-carpet of sleeping dust....

So, I watched through the fleeting ages, lost in soul-wearing thoughts and wonderings, and possessed with a new weariness.

\clearpage
