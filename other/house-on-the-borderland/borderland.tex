% !TeX TS-program = LuaLaTeX
% !TeX encoding = UTF-8
\documentclass{novel} % v. 1.50.2 or later
% Note that just about all of the setup uses defaults!
\SetTitle{The House on the Borderland}
% Abriged Version
\SetAuthor{William Hope Hodgson}
\SetPDFX[CGATSTR001]{X-1a:2001} % US Web Coated SWOP v2. Profile not embedded.

\begin{document}

% The House on the Borderland, abridged version
% Version 2018/04/03
% This is a demonstration of LuaLaTeX `novel' document class.


% ----------------------------------------------------------------------------


\frontmatter % Sets page to lowercase roman i.


% Half-Title, page i
\thispagestyle{empty}
% \nbs is normal baseline skip.
\vspace*{6\nbs} % Note asterisk. Without asterisk, does not force space, when at top of page.
{\centering\charscale[1.4]{The House on the Borderland}\par}
\cleartorecto % So that next page ii is blank.


% Title, page iii
\thispagestyle{empty}
\vspace*{4\nbs}
{\centering\charscale[4]{The House on}\par}
\vspace{2\nbs} % Could use asterisk; doesn't matter mid-page, for this command.
{\centering\charscale[4]{the Borderland}\par}
\vspace{2\nbs}
% In that era, books sometimes had elaborate subtitles, such as the following.
% In our era, this is rarely done.
\begin{center}
\textit{From the Manuscript discovered in 1877\\
by Messrs. Tonnison and Berreggnog\\
in the Ruins that lie to the South of\\
the Village of Kraighten, in the West of Ireland.\\
Set out here, with Notes.}\par
\end{center}
\vspace{3\nbs}
{\centering\charscale[2.5]{William Hope Hodgson}\par}
\vspace{3\nbs}
{\centering\textsc{ABRIDGED VERSION 2018/04/03}\par}
\vfill % forces remainder to bottom of page
{\centering A LuaLaTeX Typesetting Demonstration\par}
\clearpage


% Copyright, page iv
\thispagestyle{empty}
\null
\vfill
\begin{legalese}
The main text of \textit{The House on the Borderland}, by William Hope Hodgson, is in the Public Domain of the United States of America because it was published in the U.S.A. prior to January 1, 1923. It is in the Public Domain of its creator’s home jurisdiction of the United Kingdom, because more than seventy years have passed since death of its creator in \lnum{1918}.\par
\null
Accompanying Cover Artwork, Foreword, and (to the extent covered by copyright law) book design and TeX markup, are Copyright ©\lnum{2017} Anonymous Editor, located in California, United States of America. These copyrights are hereby licensed under the LaTeX Project Public License, v. \lnum{1.3c}.\par
\null
This is a work of fiction. Places, events, and characters were created in the imagination of William Hope Hodgson.\par
\null
\allsmcp{ABRIDGED VERSION 2017/09/18.} This is a demonstration file for the LuaLaTeX \textit{novel} document class. It is not an authoritative edition. Substantial portions of the original work have been omitted, and portions of the included work have been edited.\par
\null
\allsmcp{DISCLAIMER:} Although the resulting PDF files are labeled as print-ready, there is \allsmcp{NO WARRANTY, EXPRESS OR IMPLIED} that they are actually suitable for printing or other publication.\par
\end{legalese}
\clearpage


% Dedication, page v.
% Although the poem, Shoon of the Dead, appears to be part of the original Dedication,
% I have moved it to a separate Epigraph.
% "Shoon" is an archaic word for "shoes."
\thispagestyle{empty}
\vspace*{6\nbs}
{\centering\charscale[1.2]{To My Father}\par}
\vspace{0.5\nbs}
{\centering\textit{(Whose feet tread the lost aeons)}\par}
\clearpage % Not \cleartorecto here, because Epigraph appears on verso.


% Epigraph, page vi
% This poem was apparently part of the original Dedication.
% I have makde it a separate Epigraph, due to its length,
% and also as a demonstration of how an Epigraph may sometimes be used verso instead of recto,
% particularly when facing a table of contents.
\thispagestyle{empty}
% Although the header does not yet appear, I reset the head texts now, lest they be forgotten.
\SetVersoHeadText{\textsc{William\, Hope\, Hodgson}}
\SetRectoHeadText{\textsc{The\, House\, on\, the\, Borderland}}
\vspace*{6\nbs}
\begin{adjustwidth}{7em}{0em}
\noindent Open the door,\\
And listen!\\
Only the wind’s muffled roar,\\
And the glisten\\
Of tears ’round the moon.\\
And, in fancy, the tread\\
Of vanishing shoon---\\
Out in the night with the Dead.\par
\null
\noindent Hush! And hark\\
To the sorrowful cry\\
Of the wind in the dark.\\
Hush and hark, without murmur or sigh,\\
To shoon that tread the lost aeons:\\
To the sound that bids you to die.\\
Hush and hark! Hush and Hark!\\
Shoon of the Dead\par
\end{adjustwidth}
\clearpage


% Table of Contents, page vii
\thispagestyle{empty}
\vspace*{2\nbs}
{\centering\charscale[1.4]{Contents}\par}
\null
\begin{toc}[0]{3em}
\tocitem*{\textsc{Foreword}}{\pageref{ch:foreword}}
\vspace{0.5\nbs}
\tocitem*{\textsc{Author's Note}}{\pageref{ch:authors-note}}
\vspace{0.5\nbs}
\tocitem*{\textsc{Introduction.} The Finding of the Manuscript}{\pageref{ch:intro}}
\vspace{0.5\nbs}
\tocitem*[1]{The Plain of Silence}{\pageref{ch:01}}
\tocitem*[2]{The House in the Arena}{\pageref{ch:02}}
\tocitem*[3]{The Earth}{\pageref{ch:03}}
\tocitem*[4]{The Thing in the Ravine}{\pageref{ch:04}}
\tocitem*[5]{The Cellars}{\pageref{ch:05}}
\tocitem*[6]{The Subterranean Pit}{\pageref{ch:06}}
\tocitem*[7]{The Trap in the Great Cellar}{\pageref{ch:07}}
\tocitem*[8]{The Sea of Sleep}{\pageref{ch:08}}
\tocitem*[9]{The Noise in the Night}{\pageref{ch:09}}
\tocitem*[10]{The Awakening}{\pageref{ch:10}}
\tocitem*[11]{The Slowing Rotation}{\pageref{ch:11}}
\tocitem*[12]{The Green Star}{\pageref{ch:12}}
\tocitem*[13]{The End of the Solar System}{\pageref{ch:13}}
\tocitem*[14]{The Celestial Globes}{\pageref{ch:14}}
\tocitem*[15]{The Dark Sun}{\pageref{ch:15}}
\tocitem*[16]{The Dark Nebula}{\pageref{ch:16}}
\tocitem*[17]{Pepper}{\pageref{ch:17}}
\tocitem*[18]{The Footsteps in the Garden}{\pageref{ch:18}}
\tocitem*[19]{The Thing From the Arena}{\pageref{ch:19}}
\tocitem*[20]{The Luminous Speck}{\pageref{ch:20}}
\vspace{0.5\nbs}
\tocitem*{\textsc{Conclusion}}{\pageref{ch:conclusion}}
\end{toc}
\cleartorecto % So that next page viii is blank


% Foreword, pages ix-x
\thispagestyle{empty} % See also the use of this command at end of 2-page Foreword.
% A Preface or Foreword does not have author/title in the header.
% If it goes on for more than a page, or sometimes more than 2 pages,
% then the header will show Preface or Foreword verso/recto.
% Although this was not necessary in this document, due to having
% a short Foreword, it is general practice for longer material.
% In that case, use \SetVersoHeadText and \SetRectoHeadText.
% Then, remember to change them again!
\label{ch:foreword}
\begin{ChapterStart}
\null\null
\ChapterTitle{Foreword}
\end{ChapterStart}
This abridged edition of \textit{The House on the Borderland} demonstrates the typesetting capabilities of LuaLaTeX, using the \textit{novel} document class. It is not an exact transcription of the original text, and should not be used as a reference for critical study.\par
Hodgson’s complete novel has several early chapters dealing with swine-creatures, who lurk in the Pit near his house. In the interest of keeping this edition brief, I have removed most of the adventures involving those creatures, leaving only a brief, early encounter. I have kept the one essential swine-thing that appears in the fantastic dimension and materializes later in the story. Although this removes some of the action adventure, I believe that the remaining story is sufficiently coherent.\par
The name “Pit” was originally used to mean both the large ravine, and the deep subterranean pit that lies underneath the house and has an entrance toward the ravine. For greater clarity, I have changed “Pit” to “ravine” when it does not refer to the subterranean pit, or to the demolished remains.\par
The Recluse was accompanied by his elderly sister, who appears from time to time in the original story, usually as a background element. I have removed her, and all references to her and her cat, as they were not essential. I believe the result places tighter focus in the Recluse and his frame of mind.\par
The original has portions that were supposed to be damaged or illegible, due to weathering of the manuscript prior to its discovery. Fragments were printed, along with notes to that effect. I did not think that the story was in any way advanced by their presence, other than as a pretense, so I removed them.\par
The poem “Grief” appeared at the end of the plain text version. I do not know its position or purpose in the original print. Thus, I have taken the liberty of placing it at the end of the chapter where the Recluse first mentions his lost love, as if it were part of the diary entry there. I believe it is consistent with the tone and flow at that point.\par
In Hodgson’s era, toward the tail end of Romanticism, many words were often used when few would have sufficed; an emotion rarely passed without being re-stated. From time to time throughout the book, I have deleted extraneous or redundant material, for the sake of directness and typeset appearance. The deletions may have been a few words, paragraphs, or entire chapters. In the remaining material, my own editing consists largely of necessary transitions, clarifications, and grammatical corrections due to splicing.\par
\null
\stake\hfill---Anonymous Editor\par
\stake\hfill California, September 2017\par
\thispagestyle{empty} % Retroactively blanks the header for this second page of Foreword.
\cleartorecto


% Author's Note (WHH), page xii.
% This page was "Author's Introduction to the Manuscript" in the plain-text version that I used.
% I have not seen the original book.
% It is somewhat problematic, because it is signed by WHH as author, yet it is fictional.
% Since it pretends to be a factual note by Hodgson, concerning material that came into his hands from someone else, I have decided to place it front matter, and re-title it as "Author's Note." This was a judgement call.
% At the time I write this, extra inter-word space must be added when small caps are used in headers. This can be achieved using \, to add a half-space. In the future, this behavior may change. It is not a property of novel package, but of an included package.
\label{ch:authors-note}
\begin{ChapterStart}[7]
\null\null
\ChapterTitle{Author’s Note}
\end{ChapterStart}
Many are the hours in which I have pondered upon the story that is set forth in the following pages. I trust that my instincts are not awry when they prompt me to leave the account, in simplicity, as it was handed to me.\par
And the MS. itself---You must picture me, when first it was given into my care, turning it over, curiously, and making a swift, jerky examination. A small book it is; but thick, and all, save the last few pages, filled with a quaint but legible handwriting, and writ very close. I have the queer, faint, pit-water smell of it in my nostrils now as I write, and my fingers have subconscious memories of the soft, “cloggy” feel of the long-damp pages.\par
I read, and, in reading, lifted the Curtains of the Impossible that blind the mind, and looked out into the unknown. Amid stiff, abrupt sentences I wandered; and, presently, I had no fault to charge against their abrupt tellings; for, better far than my own ambitious phrasing, is this mutilated story capable of bringing home all that the old Recluse, of the vanished house, had striven to tell.\par
Of the simple, stiffly given account of weird and extraordinary matters, I will say little. It lies before you. The inner story must be uncovered, personally, by each reader, according to ability and desire. And even should any fail to see, as now I see, the shadowed picture and conception of that to which one may well give the accepted titles of Heaven and Hell; yet can I promise certain thrills, merely taking the story as a story.\par
\null
\stake\hfill---William Hope Hodgson\par
\stake\hfill December 17, 1907\par


% -------------------------------------------------------------------------------------


\mainmatter % Includes \cleartorecto from novel v. 1.40.1.


% Advantage of components: Can compile in pieces during editing.
% Use \input, not \include.
% Comment-out lines that are not used during partial compilation.
% File borderland-00-introduction.txt.
% Version 2017/09/18
% According to the book's lengthy subtitle, the manuscript was discovered by "Messrs. Tonnison and Berreggnog."
% Therefore, Hodgson was not the discoverer, but he acquired the manuscript from someone else.
% In the following Introduction (originally Chapter I) we deduce that the voice must be that of Bereggnog, since Tonnison is separately mentioned.
% I have added "As Told by Mr. Bereggnog" to clarify this, and added "Begin Manuscript" at the end of the Introduction, so that the change of first-person narrator is clear.
% Since this Introduction is entirely fictional and and integral part of the story, its header shows author/title rather than Introduction.

\cleartorecto % this must begin recto

\label{ch:intro}

\begin{ChapterStart}
\null
\ChapterTitle{Introduction}
\null
\ChapterSubtitle{The Finding of the Manuscript}
\null
\ChapterDeco{As Told by Mr. Berreggnog}
\end{ChapterStart}


Right away in the west of Ireland lies a tiny hamlet called Kraighten. It is situated, alone, at the base of a low hill. Far around there spreads a waste of bleak and totally inhospitable country; where, here and there at great intervals, one may come upon the ruins of some long desolate cottage---unthatched and stark. The whole land is bare and unpeopled, the very earth scarcely covering the rock that lies beneath it, and with which the country abounds, in places rising out of the soil in wave-shaped ridges.

Yet, in spite of its desolation, my friend Tonnison and I had elected to spend our vacation there. He had stumbled on the place by mere chance the year previously, during the course of a long walking tour, and discovered the possibilities for the angler in a small and unnamed river that runs past the outskirts of the little village.

I have said that the river is without name; I may add that no map that I have hitherto consulted has shown either village or stream. They seem to have entirely escaped observation: indeed, they might never exist for all that the average guide tells one. Possibly this can be partly accounted for by the fact that the nearest railway station (Ardrahan) is some forty miles distant.

It was early one warm evening when my friend and I arrived in Kraighten. We had reached Ardrahan the previous night, sleeping there in rooms hired at the village post office, and leaving in good time on the following morning, clinging insecurely to one of the typical jaunting cars.

It had taken us all day to accomplish our journey over some of the roughest tracks imaginable, with the result that we were thoroughly tired and somewhat bad tempered. However, the tent had to be erected and our goods stowed away before we could think of food or rest. And so we set to work, with the aid of our driver, and soon had the tent up upon a small patch of ground just outside the little village, and quite near to the river.

Then, having stored all our belongings, we dismissed the driver, as he had to make his way back as speedily as possible, and told him to come across to us at the end of a fortnight. We had brought sufficient provisions to last us for that space of time, and water we could get from the stream. Fuel we did not need, as we had included a small oil-stove among our outfit, and the weather was fine and warm.

It was Tonnison’s idea to camp out instead of getting lodgings in one of the cottages. As he put it, there was no joke in sleeping in a room with a numerous family of healthy Irish in one corner and the pigsty in the other, while overhead a ragged colony of roosting fowls distributed their blessings impartially, and the whole place so full of peat smoke that it made a fellow sneeze his head off just to put it inside the doorway.

Tonnison had got the stove lit now and was busy cutting slices of bacon into the frying pan; so I took the kettle and walked down to the river for water. On the way, I had to pass close to a little group of the village people, who eyed me curiously, but not in any unfriendly manner, though none of them ventured a word.

As I returned with my kettle filled, I went up to them and, after a friendly nod, to which they replied in like manner, I asked them casually about the fishing; but, instead of answering, they just shook their heads silently, and stared at me. I repeated the question, addressing more particularly a great, gaunt fellow at my elbow; yet again I received no answer. Then the man turned to a comrade and said something rapidly in a language that I did not understand; and, at once, the whole crowd of them fell to jabbering in what, after a few moments, I guessed to be pure Irish. At the same time they cast many glances in my direction. For a minute, perhaps, they spoke among themselves thus; then the man I had addressed faced ’round at me and said something. By the expression of his face I guessed that he, in turn, was questioning me; but now I had to shake my head, and indicate that I did not comprehend what it was they wanted to know; and so we stood looking at one another, until I heard Tonnison calling to me to hurry up with the kettle. Then, with a smile and a nod, I left them, and all in the little crowd smiled and nodded in return, though their faces still betrayed their puzzlement.

It was evident, I reflected as I went toward the tent, that the inhabitants of these few huts in the wilderness did not know a word of English; and when I told Tonnison, he remarked that he was aware of the fact, and, more, that it was not at all uncommon in that part of the country, where the people often lived and died in their isolated hamlets without ever coming in contact with the outside world.

“I wish we had got the driver to interpret for us before he left,” I remarked, as we sat down to our meal. “It seems so strange for the people of this place not even to know what we’ve come for.”

Tonnison grunted an assent, and thereafter was silent for a while.

Later, having satisfied our appetites somewhat, we began to talk, laying our plans for the morrow; then, after a smoke, we closed the flap of the tent, and prepared to turn in.

“I suppose there’s no chance of those fellows outside taking anything?” I asked, as we rolled ourselves in our blankets.

Tonnison said that he did not think so, at least while we were about; and, as he went on to explain, we could lock up everything, except the tent, in the big chest that we had brought to hold our provisions. I agreed to this, and soon we were both asleep.

Next morning, early, we rose and went for a swim in the river; after which we dressed and had breakfast. Then we roused out our fishing tackle and overhauled it, by which time, our breakfasts having settled somewhat, we made all secure within the tent and strode off in the direction my friend had explored on his previous visit.

During the day we fished happily, working steadily upstream, and by evening we had one of the prettiest creels of fish that I had seen for a long while. Returning to the village, we made a good feed off our day’s spoil, after which, having selected a few of the finer fish for our breakfast, we presented the remainder to the group of villagers who had assembled at a respectful distance to watch our doings. They seemed wonderfully grateful, and heaped mountains of what I presumed to be Irish blessings upon our heads.

Thus we spent several days, having splendid sport, and first-rate appetites to do justice upon our prey. We were pleased to find how friendly the villagers were inclined to be, and that there was no evidence of their having ventured to meddle with our belongings during our absences.

It was on a Tuesday that we arrived in Kraighten, and it would be on the Sunday following that we made a great discovery. Hitherto we had always gone up-stream; on that day, however, we laid aside our rods, and, taking some provisions, set off for a long ramble in the opposite direction. The day was warm, and we trudged along leisurely enough, stopping about mid-day to eat our lunch upon a great flat rock near the riverbank. Afterward we sat and smoked awhile, resuming our walk only when we were tired of inaction.

For perhaps another hour we wandered onward, chatting quietly and comfortably on this and that matter, and on several occasions stopping while my companion---who is something of an artist---made rough sketches of striking bits of the wild scenery.

And then, without any warning whatsoever, the river we had followed so confidently, came to an abrupt end---vanishing into the earth.

“Good Lord!” I said, “who ever would have thought of this?”

And I stared in amazement; then I turned to Tonnison. He was looking, with a blank expression upon his face, at the place where the river disappeared.

In a moment he spoke.

“Let us go on a bit; it may reappear again---anyhow, it is worth investigating.”

I agreed, and we went forward once more, though rather aimlessly; for we were not at all certain in which direction to prosecute our search. For perhaps a mile we moved onward; then Tonnison, who had been gazing about curiously, stopped and shaded his eyes.

“See!” he said, after a moment, “isn’t that mist or something, over there to the right---away in a line with that great piece of rock?” And he indicated with his hand.

I stared, and, after a minute, seemed to see something, but could not be certain, and said so.

“Anyway,” my friend replied, “we’ll just go across and have a glance.” And he started off in the direction he had suggested, I following. Presently, we came among bushes, and, after a time, out upon the top of a high, boulder-strewn bank, from which we looked down into a wilderness of bushes and trees.

“Seems as though we had come upon an oasis in this desert of stone,” muttered Tonnison, as he gazed interestedly. Then he was silent, his eyes fixed; and I looked also; for up from somewhere about the center of the wooded lowland there rose high into the quiet air a great column of hazelike spray, upon which the sun shone, causing innumerable rainbows.

“How beautiful!” I exclaimed.

“Yes,” answered Tonnison, thoughtfully. “There must be a waterfall, or something, over there. Perhaps it’s our river come to light again. Let’s go and see.”

Down the sloping bank we made our way, and entered among the trees and shrubberies. The bushes were matted, and the trees overhung us, so that the place was disagreeably gloomy; though not dark enough to hide from me the fact that many of the trees were fruit trees, and that, here and there, one could trace indistinctly, signs of a long departed cultivation. Thus it came to me that we were making our way through the riot of a great and ancient garden. I said as much to Tonnison, and he agreed that there certainly seemed reasonable grounds for my belief.

What a wild place it was, so dismal and somber! Somehow, as we went forward, a sense of the silent loneliness and desertion of the old garden grew upon me, and I felt shivery. One could imagine things lurking among the tangled bushes; while, in the very air of the place, there seemed something uncanny. I think Tonnison was conscious of this also, though he said nothing.

Suddenly, we came to a halt. Through the trees there had grown upon our ears a distant sound. Tonnison bent forward, listening. I could hear it more plainly now; it was continuous and harsh---a sort of droning roar, seeming to come from far away. I experienced a queer, indescribable, little feeling of nervousness. What sort of place was it into which we had got? I looked at my companion, to see what he thought of the matter; and noted that there was only puzzlement in his face; and then, as I watched his features, an expression of comprehension crept over them, and he nodded his head.

“That’s a waterfall,” he exclaimed, with conviction. “I know the sound now.” And he began to push vigorously through the bushes, in the direction of the noise.

As we went forward, the sound became plainer continually, showing that we were heading straight toward it. Steadily, the roaring grew louder and nearer, until it appeared, as I remarked to Tonnison, almost to come from under our feet---and still we were surrounded by the trees and shrubs.

“Take care!” Tonnison called to me. “Look where you’re going.” And then, suddenly, we came out from among the trees, on to a great open space, where, not six paces in front of us, yawned the mouth of a tremendous chasm, from the depths of which the noise appeared to rise, along with the continuous, mistlike spray that we had witnessed from the top of the distant bank.

For quite a minute we stood in silence, staring in bewilderment at the sight; then my friend went forward cautiously to the edge of the abyss. I followed, and, together, we looked down through a boil of spray at a monster cataract of frothing water that burst, spouting, from the side of the chasm, nearly a hundred feet below.

“Good Lord!” said Tonnison.

I was silent, and rather awed. The sight was so unexpectedly grand and eerie; though this latter quality came more upon me later.

Presently, I looked up and across to the further side of the chasm. There, I saw something towering up among the spray: it looked like a fragment of a great ruin, and I touched Tonnison on the shoulder. He glanced ’round, with a start, and I pointed toward the thing. His gaze followed my finger, and his eyes lighted up with a sudden flash of excitement, as the object came within his field of view.

“Come along,” he shouted above the uproar. “We’ll have a look at it. There’s something queer about this place; I feel it in my bones.” And he started off, ’round the edge of the craterlike abyss. As we neared this new thing, I saw that I had not been mistaken in my first impression. It was undoubtedly a portion of some ruined building; yet now I made out that it was not built upon the edge of the chasm itself, as I had at first supposed; but perched almost at the extreme end of a huge spur of rock that jutted out some fifty or sixty feet over the abyss. In fact, the jagged mass of ruin was literally suspended in midair.

Arriving opposite it, we walked out on to the projecting arm of rock, and I must confess to having felt an intolerable sense of terror as I looked down from that dizzy perch into the unknown depths below us---into the deeps from which there rose ever the thunder of the falling water and the shroud of rising spray.

Reaching the ruin, we clambered ’round it cautiously, and, on the further side, came upon a mass of fallen stones and rubble. The ruin itself seemed to me, as I proceeded now to examine it minutely, to be a portion of the outer wall of some prodigious structure, it was so thick and substantially built; yet what it was doing in such a position I could by no means conjecture. Where was the rest of the house, or castle, or whatever there had been?

I went back to the outer side of the wall, and thence to the edge of the chasm, leaving Tonnison rooting systematically among the heap of stones and rubbish on the outer side. Then I commenced to examine the surface of the ground, near the edge of the abyss, to see whether there were not left other remnants of the building to which the fragment of ruin evidently belonged. But though I scrutinized the earth with the greatest care, I could see no signs of anything to show that there had ever been a building erected on the spot, and I grew more puzzled than ever.

Then, I heard a cry from Tonnison; he was shouting my name, excitedly, and without delay I hurried along the rocky promontory to the ruin. I wondered whether he had hurt himself, and then the thought came, that perhaps he had found something.

I reached the crumbled wall and climbed ’round. There I found Tonnison standing within a small excavation that he had made among the \textit{débris}: he was brushing the dirt from something that looked like a book, much crumpled and dilapidated; and opening his mouth, every second or two, to bellow my name. As soon as he saw that I had come, he handed his prize to me, telling me to put it into my satchel so as to protect it from the damp, while he continued his explorations. This I did, first, however, running the pages through my fingers, and noting that they were closely filled with neat, old-fashioned writing which was quite legible, save in one portion, where many of the pages were almost destroyed, being muddied and crumpled, as though the book had been doubled back at that part. This, I found out from Tonnison, was actually as he had discovered it, and the damage was due, probably, to the fall of masonry upon the opened part. Curiously enough, the book was fairly dry, which I attributed to its having been so securely buried among the ruins.

Having put the volume away safely, I turned-to and gave Tonnison a hand with his self-imposed task of excavating; yet, though we put in over an hour’s hard work, turning over the whole of the upheaped stones and rubbish, we came upon nothing more than some fragments of broken wood, that might have been parts of a desk or table; and so we gave up searching, and went back along the rock, once more to the safety of the land.

The next thing we did was to make a complete tour of the tremendous chasm, which we were able to observe was in the form of an almost perfect circle, save for where the ruin-crowned spur of rock jutted out, spoiling its symmetry.

The abyss was, as Tonnison put it, like nothing so much as a gigantic well or pit going sheer down into the bowels of the earth.

For some time longer, we continued to stare about us, and then, noticing that there was a clear space away to the north of the chasm, we bent our steps in that direction.

Here, distant from the mouth of the mighty pit by some hundreds of yards, we came upon a great lake of silent water---silent, that is, save in one place where there was a continuous bubbling and gurgling.

Now, being away from the noise of the spouting cataract, we were able to hear one another speak, without having to shout at the tops of our voices, and I asked Tonnison what he thought of the place---I told him that I didn’t like it, and that the sooner we were out of it the better I should be pleased.

He nodded in reply, and glanced at the woods behind furtively. I asked him if he had seen or heard anything. He made no answer; but stood silent, as though listening, and I kept quiet also.

Suddenly, he spoke.

“Hark!” he said, sharply. I looked at him, and then away among the trees and bushes, holding my breath involuntarily. A minute came and went in strained silence; yet I could hear nothing, and I turned to Tonnison to say as much; and then, even as I opened my lips to speak, there came a strange wailing noise out of the wood on our left.... It appeared to float through the trees, and there was a rustle of stirring leaves, and then silence.

All at once, Tonnison spoke, and put his hand on my shoulder. “Let us get out of here,” he said, and began to move slowly toward where the surrounding trees and bushes seemed thinnest. As I followed him, it came to me suddenly that the sun was low, and that there was a raw sense of chilliness in the air.

Tonnison said nothing further, but kept on steadily. We were among the trees now, and I glanced around, nervously; but saw nothing, save the quiet branches and trunks and the tangled bushes. Onward we went, and no sound broke the silence, except the occasional snapping of a twig under our feet, as we moved forward. Yet, in spite of the quietness, I had a horrible feeling that we were not alone; and I kept so close to Tonnison that twice I kicked his heels clumsily, though he said nothing. A minute, and then another, and we reached the confines of the wood coming out at last upon the bare rockiness of the countryside. Only then was I able to shake off the haunting dread that had followed me among the trees.

Once, as we moved away, there seemed to come again a distant sound of wailing, and I said to myself that it was the wind---yet the evening was breathless.

Presently, Tonnison began to talk.

“Look you,” he said with decision, “I would not spend the night in \textit{that} place for all the wealth that the world holds. There is something unholy---diabolical---about it. It came to me all in a moment, just after you spoke. It seemed to me that the woods were full of vile things---you know!”

“Yes,” I answered, and looked back toward the place; but it was hidden from us by a rise in the ground.

“There’s the book,” I said, and I put my hand into the satchel.

“You’ve got it safely?” he questioned, with a sudden access of anxiety.

“Yes,” I replied.

“Perhaps,” he continued, “we shall learn something from it when we get back to the tent. We had better hurry, too; we’re a long way off still, and I don’t fancy, now, being caught out here in the dark.”

It was two hours later when we reached the tent; and, without delay, we set to work to prepare a meal; for we had eaten nothing since our lunch at midday.

Supper over, we cleared the things out of the way, and lit our pipes. Then Tonnison asked me to get the manuscript out of my satchel. This I did, and then, as we could not both read from it at the same time, he suggested that I should read the thing out loud. “And mind,” he cautioned, knowing my propensities, “don’t go skipping half the book.”

Yet, had he but known what it contained, he would have realized how needless such advice was, for once at least. And there seated in the opening of our little tent, I began the strange tale of \textit{The House on the Borderland} (for such was the title of the MS.); this is told in the following pages.

\null\null
{\centering\textit{Begin Manuscript...}\par}

\clearpage % the following material begins with \cleartorecto

\input{borderland-components/borderland-01.tex}
% File borderland-02.txt
% Version 2017/09/18
% In the original, this was Chapter III.

\clearpage
\label{ch:02}

\begin{ChapterStart}
\null\null
\ChapterTitle{2. The House in the Arena}
\end{ChapterStart}

And so, after a time, I came to the mountains. Then, the course of my journey was altered, and I began to move along their bases, until, all at once, I saw that I had come opposite to a vast rift, opening into the mountains. Through this, I was borne, moving at no great speed. On either side of me, huge, scarped walls of rocklike substance rose sheer. Far overhead, I discerned a thin ribbon of red, where the mouth of the chasm opened, among inaccessible peaks. Within, was gloom, deep and somber, and chilly silence. For a while, I went onward steadily, and then, at last, I saw, ahead, a deep, red glow, that told me I was near upon the further opening of the gorge.

A minute came and went, and I was at the exit of the chasm, staring out upon an enormous amphitheatre of mountains. Yet, of the mountains, and the terrible grandeur of the place, I recked nothing; for I was confounded with amazement to behold, at a distance of several miles and occupying the center of the arena, a stupendous structure built apparently of green jade. Yet, in itself, it was not the discovery of the building that had so astonished me; but the fact, which became every moment more apparent, that in no particular, save in color and its enormous size, did the lonely structure vary from this house in which I live.

For a while, I continued to stare, fixedly. Even then, I could scarcely believe that I saw aright. In my mind, a question formed, reiterating incessantly: ‘What does it mean?’ ‘What does it mean?’ and I was unable to make answer, even out of the depths of my imagination. I seemed capable only of wonder and fear. For a time longer, I gazed, noting continually some fresh point of resemblance that attracted me. At last, wearied and sorely puzzled, I turned from it, to view the rest of the strange place on to which I had intruded.

Hitherto, I had been so engrossed in my scrutiny of the House, that I had given only a cursory glance ’round. Now, as I looked, I began to realize upon what sort of a place I had come. The arena, for so I have termed it, appeared a perfect circle of about ten to twelve miles in diameter, the House, as I have mentioned before, standing in the center. The surface of the place, like to that of the Plain, had a peculiar, misty appearance, that was yet not mist.

From a rapid survey, my glance passed quickly upward along the slopes of the circling mountains. How silent they were. I think that this same abominable stillness was more trying to me than anything that I had so far seen or imagined. I was looking up, now, at the great crags, towering so loftily. Up there, the impalpable redness gave a blurred appearance to everything.

And then, as I peered, curiously, a new terror came to me; for away up among the dim peaks to my right, I had descried a vast shape of blackness, giantlike. It grew upon my sight. It had an enormous equine head, with gigantic ears, and seemed to peer steadfastly down into the arena. There was that about the pose that gave me the impression of an eternal watchfulness---of having warded that dismal place, through unknown eternities. Slowly, the monster became plainer to me; and then, suddenly, my gaze sprang from it to something further off and higher among the crags. For a long minute, I gazed, fearfully. I was strangely conscious of something not altogether unfamiliar---as though something stirred in the back of my mind. The thing was black, and had four grotesque arms. The features showed indistinctly, ’round the neck, I made out several light-colored objects. Slowly, the details came to me, and I realized, coldly, that they were skulls. Further down the body was another circling belt, showing less dark against the black trunk. Then, even as I puzzled to know what the thing was, a memory slid into my mind, and straightway, I knew that I was looking at a monstrous representation of Kali, the Hindu goddess of death.

Other remembrances of my old student days drifted into my thoughts. My glance fell back upon the huge beast-headed Thing. Simultaneously, I recognized it for the ancient Egyptian god Set, or Seth, the Destroyer of Souls. With the knowledge, there came a great sweep of questioning---‘Two of the---!’ I stopped, and endeavored to think. Things beyond my imagination peered into my frightened mind. I saw, obscurely. ‘The old gods of mythology!’ I tried to comprehend to what it was all pointing. My gaze dwelt, flickeringly, between the two. ‘If---’

An idea came swiftly, and I turned, and glanced rapidly upward, searching the gloomy crags, away to my left. Something loomed out under a great peak, a shape of greyness. I wondered I had not seen it earlier, and then remembered I had not yet viewed that portion. I saw it more plainly now. It was, as I have said, grey. It had a tremendous head; but no eyes. That part of its face was blank.

Now, I saw that there were other things up among the mountains. Further off, reclining on a lofty ledge, I made out a livid mass, irregular and ghoulish. It seemed without form, save for an unclean, half-animal face, that looked out, vilely, from somewhere about its middle. And then I saw others---there were hundreds of them. They seemed to grow out of the shadows. Several I recognized almost immediately as mythological deities; others were strange to me, utterly strange, beyond the power of a human mind to conceive.

On each side, I looked, and saw more, continually. The mountains were full of strange things---Beast-gods, and Horrors so atrocious and bestial that possibility and decency deny any further attempt to describe them. And I---I was filled with a terrible sense of overwhelming horror and fear and repugnance; yet, spite of these, I wondered exceedingly. Was there then, after all, something in the old heathen worship, something more than the mere deifying of men, animals, and elements? The thought gripped me---was there?

What were they, those Beast-gods, and the others? At first, they had appeared to me just sculptured Monsters placed indiscriminately among the inaccessible peaks and precipices of the surrounding mountains. Now, as I scrutinized them with greater intentness, my mind began to reach out to fresh conclusions. There was something about them, an indescribable sort of silent vitality that suggested, to my broadening consciousness, a state of life-in-death---a something that was by no means life, as we understand it; but rather an inhuman form of existence, that well might be likened to a deathless trance---a condition in which it was possible to imagine their continuing, eternally. ‘Immortal!’ the word rose in my thoughts unbidden; and, straightway, I grew to wondering whether this might be the immortality of the gods.

And then, in the midst of my wondering and musing, something happened. Until then, I had been staying just within the shadow of the exit of the great rift. Now, without volition on my part, I drifted out of the semi-darkness and began to move slowly across the arena---toward the House. At this, I gave up all thoughts of those prodigious Shapes above me---and could only stare, frightenedly, at the tremendous structure toward which I was being conveyed so remorselessly. Yet, though I searched earnestly, I could discover nothing that I had not already seen, and so became gradually calmer.

Presently, I had reached a point more than halfway between the House and the gorge. All around was spread the stark loneliness of the place, and the unbroken silence. Steadily, I neared the great building. Then, all at once, something caught my vision, something that came ’round one of the huge buttresses of the House, and so into full view. It was a gigantic thing, and moved with a curious lope, going almost upright, after the manner of a man. It was quite unclothed, and had a remarkable luminous appearance. Yet it was the face that attracted and frightened me the most. It was the face of a swine.

Silently, intently, I watched this horrible creature, and forgot my fear, momentarily, in my interest in its movements. It was making its way, cumbrously ’round the building, stopping as it came to each window to peer in and shake at the bars, with which---as in this house---they were protected; and whenever it came to a door, it would push at it, fingering the fastening stealthily. Evidently, it was searching for an ingress into the House.

I had come now to within less than a quarter of a mile of the great structure, and still I was compelled forward. Abruptly, the Thing turned and gazed hideously in my direction. It opened its mouth, and, for the first time, the stillness of that abominable place was broken, by a deep, booming note that sent an added thrill of apprehension through me. Then, immediately, I became aware that it was coming toward me, swiftly and silently. In an instant, it had covered half the distance that lay between. And still, I was borne helplessly to meet it. Only a hundred yards, and the brutish ferocity of the giant face numbed me with a feeling of unmitigated horror. I could have screamed, in the supremeness of my fear; and then, in the very moment of my extremity and despair, I became conscious that I was looking down upon the arena, from a rapidly increasing height. I was rising, rising. In an inconceivably short while, I had reached an altitude of many hundred feet. Beneath me, the spot that I had just left, was occupied by the foul swine-creature. It had gone down on all fours and was snuffing and rooting, like a veritable hog, at the surface of the arena. A moment and it rose to its feet, clutching upward, with an expression of desire upon its face such as I have never seen in this world.

Continually, I mounted higher. A few minutes, it seemed, and I had risen above the great mountains---floating, alone, afar in the redness. At a tremendous distance below, the arena showed, dimly; with the mighty House looking no larger than a tiny spot of green. The swine-thing was no longer visible.

Presently, I passed over the mountains, out above the huge breadth of the plain. Far away, on its surface, in the direction of the ring-shaped sun, there showed a confused blur. I looked toward it, indifferently. It reminded me, somewhat, of the first glimpse I had caught of the mountain-amphitheatre.

With a sense of weariness, I glanced upward at the immense ring of fire. What a strange thing it was! Then, as I stared, out from the dark center, there spurted a sudden flare of extraordinary vivid fire. Compared with the size of the black center, it was as naught; yet, in itself, stupendous. With awakened interest, I watched it carefully, noting its strange boiling and glowing. Then, in a moment, the whole thing grew dim and unreal, and so passed out of sight. Much amazed, I glanced down to the Plain from which I was still rising. Thus, I received a fresh surprise. The Plain---everything had vanished, and only a sea of red mist was spread far below me. Gradually as I stared this grew remote, and died away into a dim far mystery of red against an unfathomable night. A while, and even this had gone, and I was wrapped in an impalpable, lightless gloom.

\clearpage

\input{borderland-components/borderland-03.tex}
\input{borderland-components/borderland-04.tex}
\input{borderland-components/borderland-05.tex}
\input{borderland-components/borderland-06.tex}
% File borderland-07.txt
% Version 2017/09/18
% In the original, this was Chapter XIII.

\clearpage
\label{ch:07}

\begin{ChapterStart}
\null\null
\ChapterTitle{7. The Trap in the Great Cellar}
\end{ChapterStart}

It was not until a couple of days later, that I managed to get across to the ravine. There, I found that, in my few weeks’ absence, there had been wrought a wondrous change. Instead of the three-parts filled ravine, I looked out upon a great lake, whose placid surface, reflected the light, coldly. The water had risen to within half a dozen feet of the ravine edge. Only in one part was the lake disturbed, and that was above the place where, far down under the silent waters, yawned the entrance to the vast, underground Pit. Here, there was a continuous bubbling; and, occasionally, a curious sort of sobbing gurgle would find its way up from the depth. Beyond these, there was nothing to tell of the things that were hidden beneath. As I stood there, it came to me how wonderfully things had worked out. It was completely shut off and concealed from human curiosity forever.

Is it not possible that it has, all along, held a deeper significance, a hint---could one but have guessed---of the Pit that lies far down in the earth, beneath this old house? Under this house! Even now, the idea is strange and terrible to me. For I have proved, beyond doubt, that the Pit yawns right below the house, which is evidently supported, somewhere above the center of it, upon a tremendous, arched roof, of solid rock.

It happened in this wise, that, having occasion to go down to the cellars, the thought occurred to me to pay a visit to the great vault, where the trap is situated; and see whether everything was as I had left it.

Reaching the place, I walked slowly up the center, until I came to the trap. There it was, with the stones piled upon it, just as I had seen it last. I had a lantern with me, and the idea came to me, that now would be a good time to investigate whatever lay under the great, oak slab. Placing the lantern on the floor, I tumbled the stones off the trap, and, grasping the ring, pulled the door open. As I did so, the cellar became filled with the sound of a murmurous thunder, that rose from far below. At the same time, a damp wind blew up into my face, bringing with it a load of fine spray. Therewith, I dropped the trap, hurriedly, with a half frightened feeling of wonder.

For a moment, I stood puzzled. Then, a sudden thought possessed me, and I raised the ponderous door, with a feeling of excitement. Leaving it standing upon its end, I seized the lantern, and, kneeling down, thrust it into the opening. As I did so, the moist wind and spray drove in my eyes, making me unable to see, for a few moments. Even when my eyes were clear, I could distinguish nothing below me, save darkness, and whirling spray.

Seeing that it was useless to expect to make out anything, with the light so high, I felt in my pockets for a piece of twine, with which to lower it further into the opening. Even as I fumbled, the lantern slipped from my fingers, and hurtled down into the darkness. For a brief instant, I watched its fall, and saw the light shine on a tumult of white foam, some eighty or a hundred feet below me. Then it was gone. My sudden surmise was correct, and now, I knew the cause of the wet and noise. The great cellar was connected with the Pit, by means of the trap, which opened right above it; and the moisture, was the spray, rising from the water, falling into the depths.

In an instant, I had an explanation of certain things, that had hitherto puzzled me. These thoughts flashed through my brain, as I stood in the dark, searching my pockets for matches. I had the box in my hand now, and, striking a light, I stepped to the trap door, and closed it. Then, I piled the stones back upon it; after which, I made my way out from the cellars.

And so, I suppose the water goes on, thundering down into that bottomless hell-pit. Sometimes, I have an inexplicable desire to go down to the great cellar, open the trap, and gaze into the impenetrable, spray-damp darkness. At times, the desire becomes almost overpowering, in its intensity. It is not mere curiosity, that prompts me; but more as though some unexplained influence were at work. Still, I never go; and intend to fight down the strange longing, and crush it; even as I would the unholy thought of self-destruction.

This idea of some intangible force being exerted, may seem reasonless. Yet, my instinct warns me, that it is not so. In these things, reason seems to me less to be trusted than instinct.

One thought there is, in closing, that impresses itself upon me, with ever growing insistence. It is, that I live in a very strange house; a very awful house. And I have begun to wonder whether I am doing wisely in staying here. Yet, if I left, where could I go, and still obtain the solitude, and the sense of her presence, that alone make my old life bearable?\footnote{An apparently unmeaning interpolation. I can find no previous reference in the MS. to this matter. It becomes clearer, however, in the light of succeeding incidents.---\allsmcp{WHH}}

\clearpage

\input{borderland-components/borderland-08.tex}
% File borderland-09.txt
% Version 2017/09/18
% In the original, this was Chapter XV.

\clearpage
\label{ch:09}

\begin{ChapterStart}
\null\null
\ChapterTitle{9. The Noise in the Night}
\end{ChapterStart}

And now, I come to the strangest of all the strange happenings that have befallen me in this house of mysteries. It occurred quite lately---within the month; and I have little doubt but that what I saw was in reality the end of all things. However, to my story.

I do not know how it is; but, up to the present, I have never been able to write these things down, directly they happened. It is as though I have to wait a time, recovering my just balance, and digesting---as it were---the things I have heard or seen. No doubt, this is as it should be; for, by waiting, I see the incidents more truly, and write of them in a calmer and more judicial frame of mind.

It is now the end of November. My story relates to what happened in the first week of the month.

It was night, about eleven o’clock. Pepper and I kept one another company in the study---that great, old room of mine, where I read and work. I was reading, curiously enough, the Bible. I have begun, in these later days, to take a growing interest in that great and ancient book. Suddenly, a distinct tremor shook the house, and there came a faint and distant, whirring buzz, that grew rapidly into a far, muffled screaming. It reminded me, in a queer, gigantic way, of the noise that a clock makes, when the catch is released, and it is allowed to run down. The sound appeared to come from some remote height---somewhere up in the night. There was no repetition of the shock. I looked across at Pepper. He was sleeping peacefully.

Gradually, the whirring noise decreased, and there came a long silence.

All at once, a glow lit up the end window, which protrudes far out from the side of the house, so that, from it, one may look both East and West. I felt puzzled, and, after a moment’s hesitation, walked across the room, and pulled aside the blind. As I did so, I saw the Sun rise, from behind the horizon. It rose with a steady, perceptible movement. I could see it travel upward. In a minute, it seemed, it had reached the tops of the trees, through which I had watched it. Up, up---It was broad daylight now. Behind me, I was conscious of a sharp, mosquito-like buzzing. I glanced ’round, and knew that it came from the clock. Even as I looked, it marked off an hour. The minute hand was moving ’round the dial, faster than an ordinary second-hand. The hour hand moved quickly from space to space. I had a numb sense of astonishment. A moment later, so it seemed, the two candles went out, almost together. I turned swiftly back to the window; for I had seen the shadow of the window-frames, traveling along the floor toward me, as though a great lamp had been carried up past the window.

I saw now, that the sun had risen high into the heavens, and was still visibly moving. It passed above the house, with an extraordinary sailing kind of motion. As the window came into shadow, I saw another extraordinary thing. The fine-weather clouds were not passing, easily, across the sky---they were scampering, as though a hundred-mile-an-hour wind blew. As they passed, they changed their shapes a thousand times a minute, as though writhing with a strange life; and so were gone. And, presently, others came, and whisked away likewise.

To the West, I saw the sun, drop with an incredible, smooth, swift motion. Eastward, the shadows of every seen thing crept toward the coming greyness. And the movement of the shadows was visible to me---a stealthy, writhing creep of the shadows of the wind-stirred trees. It was a strange sight.

Quickly, the room began to darken. The sun slid down to the horizon, and seemed, as it were, to disappear from my sight, almost with a jerk. Through the greyness of the swift evening, I saw the silver crescent of the moon, falling out of the Southern sky, toward the West. The evening seemed to merge into an almost instant night. Above me, the many constellations passed in a strange, ‘noiseless’ circling, Westward. The moon fell through that last thousand fathoms of the night-gulf, and there was only the starlight....

About this time, the buzzing in the corner ceased; telling me that the clock had run down. A few minutes passed, and I saw the Eastward sky lighten. A grey, sullen morning spread through all the darkness, and hid the march of the stars. Overhead, there moved, with a heavy, everlasting rolling, a vast, seamless sky of grey clouds---a cloud-sky that would have seemed motionless, through all the length of an ordinary earth-day. The sun was hidden from me; but, from moment to moment, the world would brighten and darken, brighten and darken, beneath waves of subtle light and shadow....

The light shifted ever Westward, and the night fell upon the earth. A vast rain seemed to come with it, and a wind of a most extraordinary loudness---as though the howling of a nightlong gale, were packed into the space of no more than a minute.

This noise passed, almost immediately, and the clouds broke; so that, once more, I could see the sky. The stars were flying Westward, with astounding speed. It came to me now, for the first time, that, though the noise of the wind had passed, yet a constant ‘blurred’ sound was in my ears. Now that I noticed it, I was aware that it had been with me all the time. It was the world-noise.

And then, even as I grasped at so much comprehension, there came the Eastward light. No more than a few heartbeats, and the sun rose, swiftly. Through the trees, I saw it, and then it was above the trees. Up---up, it soared and all the world was light. It passed, with a swift, steady swing to its highest altitude, and fell thence, Westward. I saw the day roll visibly over my head. A few light clouds flittered Northward, and vanished. The sun went down with one swift, clear plunge, and there was about me, for a few seconds, the darker growing grey of the gloaming.

Southward and Westward, the moon was sinking rapidly. The night had come, already. A minute it seemed, and the moon fell those remaining fathoms of dark sky. Another minute, or so, and the Eastward sky glowed with the coming dawn. The sun leapt upon me with a frightening abruptness, and soared ever more swiftly toward the zenith. Then, suddenly, a fresh thing came to my sight. A black thundercloud rushed up out of the South, and seemed to leap all the arc of the sky, in a single instant. As it came, I saw that its advancing edge flapped, like a monstrous black cloth in the heaven, twirling and undulating rapidly, with a horrid suggestiveness. In an instant, all the air was full of rain, and a hundred lightning flashes seemed to flood downward, as it were in one great shower. In the same second of time, the world-noise was drowned in the roar of the wind, and then my ears ached, under the stunning impact of the thunder.

And, in the midst of this storm, the night came; and then, within the space of another minute, the storm had passed, and there was only the constant ‘blur’ of the world-noise on my hearing. Overhead, the stars were sliding quickly Westward; and something, mayhaps the particular speed to which they had attained, brought home to me, for the first time, a keen realization of the knowledge that it was the world that revolved. I seemed to see, suddenly, the world---a vast, dark mass---revolving visibly against the stars.

The dawn and sun seemed to come together, so greatly had the speed of the world-revolution increased. The sun drove up, in one long, steady curve; passed its highest point, swept down into the Western sky, and disappeared. I was scarcely conscious of evening, so brief was it. Then I was watching the flying constellations, and the Westward hastening moon. In but a space of seconds, so it seemed, it was sliding swiftly downward through the night-blue, and then was gone. And, almost directly, came the morning.

And now there seemed to come a strange acceleration. The sun made one clean, clear sweep through the sky, and disappeared behind the Westward horizon, and the night came and went with a like haste.

As the succeeding day, opened and closed upon the world, I was aware of a sweat of snow, suddenly upon the earth. The night came, and, almost immediately, the day. In the brief leap of the sun, I saw that the snow had vanished; and then, once more, it was night.

Thus matters were; and, even after the many incredible things that I have seen, I experienced all the time a most profound awe. To see the sun rise and set, within a space of time to be measured by seconds; to watch (after a little) the moon leap---a pale, and ever growing orb---up into the night sky, and glide, with a strange swiftness, through the vast arc of blue; and, presently, to see the sun follow, springing out of the Eastern sky, as though in chase; and then again the night, with the swift and ghostly passing of starry constellations, was all too much to view believingly. Yet, so it was---the day slipping from dawn to dusk, and the night sliding swiftly into day, ever rapidly and more rapidly.

The last three passages of the sun had shown me a snow-covered earth, which, at night, had seemed, for a few seconds, incredibly weird under the fast-shifting light of the soaring and falling moon. Now, however, for a little space, the sky was hidden, by a sea of swaying, leaden-white clouds, which lightened and blackened, alternately, with the passage of day and night.

The clouds rippled and vanished, and there was once more before me, the vision of the swiftly leaping sun, and nights that came and went like shadows.

Faster and faster, spun the world. And now each day and night was completed within the space of but a few seconds; and still the speed increased.

It was a little later, that I noticed that the sun had begun to have the suspicion of a trail of fire behind it. This was due, evidently, to the speed at which it, apparently, traversed the heavens. And, as the days sped, each one quicker than the last, the sun began to assume the appearance of a vast, flaming comet\footnote{The Recluse uses this as an illustration, evidently in the sense of the popular conception of a comet.---\allsmcp{WHH}} flaring across the sky at short, periodic intervals. At night, the moon presented, with much greater truth, a comet-like aspect; a pale, and singularly clear, fast traveling shape of fire, trailing streaks of cold flame. The stars showed now, merely as fine hairs of fire against the dark.

Once, I turned from the window, and glanced at Pepper. In the flash of a day, I saw that he slept, quietly, and I moved once more to my watching.

The sun was now bursting up from the Eastern horizon, like a stupendous rocket, seeming to occupy no more than a second or two in hurling from East to West. I could no longer perceive the passage of clouds across the sky, which seemed to have darkened somewhat. The brief nights, appeared to have lost the proper darkness of night; so that the hair-like fire of the flying stars, showed but dimly. As the speed increased, the sun began to sway very slowly in the sky, from South to North, and then, slowly again, from North to South.

So, amid a strange confusion of mind, the hours passed.

All this while had Pepper slept. Presently, feeling lonely and distraught, I called to him, softly; but he took no notice. Again, I called, raising my voice slightly; still he moved not. I walked over to where he lay, and touched him with my foot, to rouse him. At the action, gentle though it was, he fell to pieces. That is what happened; he literally and actually crumbled into a mouldering heap of bones and dust.

For the space of, perhaps a minute, I stared down at the shapeless heap, that had once been Pepper. I stood, feeling stunned. What can have happened? I asked myself; not at once grasping the grim significance of that little hill of ash. Then, as I stirred the heap with my foot, it occurred to me that this could only happen in a great space of time. Years---and years.

Outside, the weaving, fluttering light held the world. Inside, I stood, trying to understand what it meant---what that little pile of dust and dry bones, on the carpet, meant. But I could not think, coherently.

I glanced away, ’round the room, and now, for the first time, noticed how dusty and old the place looked. Dust and dirt everywhere; piled in little heaps in the corners, and spread about upon the furniture. The very carpet, itself, was invisible beneath a coating of the same, all pervading, material. As I walked, little clouds of the stuff rose up from under my footsteps, and assailed my nostrils, with a dry, bitter odor that made me wheeze, huskily.

Suddenly, as my glance fell again upon Pepper’s remains, I stood still, and gave voice to my confusion---questioning, aloud, whether the years were, indeed, passing; whether this, which I had taken to be a form of vision, was, in truth, a reality. I paused. A new thought had struck me. Quickly, but with steps which, for the first time, I noticed, tottered, I went across the room to the great pier-glass, and looked in. It was too covered with grime, to give back any reflection, and, with trembling hands, I began to rub off the dirt. Presently, I could see myself. The thought that had come to me, was confirmed. Instead of the great, hale man, who scarcely looked fifty, I was looking at a bent, decrepit man, whose shoulders stooped, and whose face was wrinkled with the years of a century. The hair---which a few short hours ago had been nearly coal black---was now silvery white. Only the eyes were bright. Gradually, I traced, in that ancient man, a faint resemblance to my self of other days.

I turned away, and tottered to the window. I knew, now, that I was old, and the knowledge seemed to confirm my trembling walk. For a little space, I stared moodily out into the blurred vista of changeful landscape. Even in that short time, a year passed, and, with a petulant gesture, I left the window. As I did so, I noticed that my hand shook with the palsy of old age; and a short sob choked its way through my lips.

For a little while, I paced, tremulously, between the window and the table; my gaze wandering hither and thither, uneasily. How dilapidated the room was. Everywhere lay the thick dust. The fender was a shape of rust. The chains that held the brass clock-weights, had rusted through long ago, and now the weights lay on the floor beneath; themselves two cones of verdigris.

As I glanced about, it seemed to me that I could see the very furniture of the room rotting and decaying before my eyes. Nor was this fancy, on my part; for, all at once, the bookshelf, along the sidewall, collapsed, with a cracking and rending of rotten wood, precipitating its contents upon the floor, and filling the room with a smother of dusty atoms.

How tired I felt. As I walked, it seemed that I could hear my dry joints, creak and crack at every step. All had happened so quickly and suddenly. This must be, indeed, the beginning of the end of all things! It occurred to me, to go to look for her; but I felt too weary. And then, she had been so queer about these happenings, of late. Of late! I repeated the words, and laughed, feebly---mirthlessly, as the realization was borne in upon me that I spoke of a time, half a century gone. Half a century! It might have been twice as long!

I moved slowly to the window, and looked out once more across the world. I can best describe the passage of day and night, at this period, as a sort of gigantic, ponderous flicker. Moment by moment, the acceleration of time continued; so that, at nights now, I saw the moon, only as a swaying trail of palish fire, that varied from a mere line of light to a nebulous path, and then dwindled again, disappearing periodically.

The flicker of the days and nights quickened. The days had grown perceptibly darker, and a queer quality of dusk lay, as it were, in the atmosphere. The nights were so much lighter, that the stars were scarcely to be seen, saving here and there an occasional hair-like line of fire, that seemed to sway a little, with the moon.

Quicker, and ever quicker, ran the flicker of day and night; and, suddenly it seemed, I was aware that the flicker had died out, and, instead, there reigned a comparatively steady light, which was shed upon all the world, from an eternal river of flame that swung up and down, North and South, in stupendous, mighty swings.

The sky was now grown very much darker, and there was in the blue of it a heavy gloom, as though a vast blackness peered through it upon the earth. Yet, there was in it, also, a strange and awful clearness, and emptiness. Periodically, I had glimpses of a ghostly track of fire that swayed thin and darkly toward the sun-stream; vanished and reappeared. It was the scarcely visible moon-stream.

Looking out at the landscape, I was conscious again, of a blurring ‘flitter,’ that came either from the light of the ponderous, swinging sun-stream, or was the result of the incredibly rapid changes of the earth’s surface. And every few moments, so it seemed, the snow would lie suddenly upon the world, and vanish as abruptly, as though an invisible giant ‘flitted’ a white sheet off and on the earth.

Time fled, and the weariness that was mine, grew insupportable. I turned from the window, and walked once across the room, the heavy dust deadening the sound of my footsteps. Each step that I took, seemed a greater effort than the one before. An intolerable ache, knew me in every joint and limb, as I trod my way, with a weary uncertainty.

By the opposite wall, I came to a weak pause, and wondered, dimly, what was my intent. I looked to my left, and saw my old chair. The thought of it brought a faint sense of comfort to my bewildered wretchedness. Yet, because I was so weary and old and tired, I would scarcely brace my mind to do anything but stand, and wish myself past those few yards. I rocked, as I stood. The floor, even, seemed a place for rest; but the dust lay so thick and sleepy and black. I turned, with a great effort of will, and made toward my chair. I reached it, with a groan of thankfulness. I sat down.

Everything about me appeared to be growing dim. It was all so strange and unthought of. Last night, I was a comparatively strong, though elderly man; and now, only a few hours later---! I looked at the little dust-heap that had once been Pepper. Hours! and I laughed, a feeble, bitter laugh; a shrill, cackling laugh, that shocked my dimming senses.

For a while, I must have dozed. Then I opened my eyes, with a start. Somewhere across the room, there had been a muffled noise of something falling. I looked, and saw, vaguely, a cloud of dust hovering above a pile of \textit{débris}. Nearer the door, something else tumbled, with a crash. It was one of the cupboards; but I was tired, and took little notice. I closed my eyes, and sat there in a state of drowsy, semi-unconsciousness. Once or twice---as though coming through thick mists---I heard noises, faintly. Then I must have slept.

\clearpage

\input{borderland-components/borderland-10.tex}
\input{borderland-components/borderland-11.tex}
\input{borderland-components/borderland-12.tex}
\input{borderland-components/borderland-13.tex}
\input{borderland-components/borderland-14.tex}
\input{borderland-components/borderland-15.tex}
\input{borderland-components/borderland-16.tex}
\input{borderland-components/borderland-17.tex}
\input{borderland-components/borderland-18.tex}
\input{borderland-components/borderland-19.tex}
\input{borderland-components/borderland-20.tex}
\input{borderland-components/borderland-21-conclusion.tex}


% No back matter. Indeed, back matter is very rare in faction, or even in nonfiction.
% If this book had endnotes or Appendix, then it would be in main matter,
% with continued Arabic page numbering.


\cleartoend % Ensures final page is blank verso, possibly preceded by blank recto.


\end{document}

